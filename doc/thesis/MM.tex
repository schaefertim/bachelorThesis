\chapter{Theoretische Analyse}\label{Kap-Theorie}

\noindent

\section{Hotellings Gesetz}\label{Sec-Hotelling}

\subsection{Modell}\label{Sec-Hotelling-Modell}

\citet{hotelling1929} untersucht anhand eines einfachen Modells, wie sich bei einem Duopol der räumliche Wettbewerb gestaltet.
Dazu nimmt er an, dass die zwei Wettbewerber auf einer räumlichen Linie ihre Position wählen können.
Abbildung \ref{Fig-Linearer-Wettbewerb} zeigt mögliche Positionen der Wettbewerber.
Dabei sind die Konsumenten gleichmäßig über die gesamte Linie verteilt.

\begin{figure}[htb]
	\centering
	\begin{tikzpicture}
	\coordinate (start) at (0,0) {};
	\coordinate (end) at (10,0) {};
	\coordinate (left) at (2,0) {};
	\coordinate (right) at (7,0) {};
	\draw (start) -- (end);
	\draw ($(start)+(0,5pt)$) -- ($(start)-(0,5pt)$);
	\draw ($(left)+(0,5pt)$) -- ($(left)-(0,5pt)$);
	\draw ($(right)+(0,5pt)$) -- ($(right)-(0,5pt)$);
	\draw ($(end)+(0,5pt)$) -- ($(end)-(0,5pt)$);
	\node at ($(start)+(0,10pt)$) {$0$};
	\node at ($(end)+(0,10pt)$) {$l$};
	\node at ($(left)+(0,10pt)$) {$A$};
	\node at ($(right)+(0,10pt)$) {$B$};
	
	\coordinate (startdown) at ($(start)+(0,-20pt)$) {};
	\coordinate (enddown) at ($(end)+(0,-20pt)$) {};
	\coordinate (leftdown) at ($(left)+(0,-20pt)$) {};
	\coordinate (rightdown) at ($(right)+(0,-20pt)$) {};
	\coordinate (middledown) at (4,-20pt) {};
	
	\draw[|-|] (startdown) -- (leftdown) node[midway, below] {$a$};
	\draw[|-|] (rightdown) -- (enddown) node[midway, below] {$b$};
	\draw[|-|] (leftdown) -- (middledown) node[midway, below] {$x$};
	\draw[|-|] (middledown) -- (rightdown) node[midway, below] {$y$};
	\end{tikzpicture}
	\caption{Positionen der Wettbewerber $A$ und $B$ und Konsumenten auf einer Linie von $0$ bis $l$. Die Konsumenten im Bereich $a$ und $x$ kaufen bei $A$ ein und $y$ und $b$ bei $B$. (Darstellung angelehnt an \cite[Abb.\,1, S.\,45]{hotelling1929})}
	\label{Fig-Linearer-Wettbewerb}
\end{figure}

Eine grundlegende Annahme des Modells ist, dass die Konsumenten lineare Transportkosten in Höhe von $c$ pro Längeneinheit haben.
Die Konsumenten kaufen somit bei demjenigen Duopolisten, bei dem die Kosten, bestehend aus Preis und Transportkosten, am geringsten sind.
Des Weiteren nimmt Hotelling an, dass $A$ zu einem Preis $p_1$ und $B$ zu einem Preis $p_2$ verkauft.
Die Verkaufsmengen sind jeweils $q_1$ und $q_2$. \citep[S.\,45]{hotelling1929}

Die erste wichtige Feststellung ist, dass das Duopol nur dann ein Duopol bleibt und nicht zu einem Monopol kollabiert, wenn der Preisunterschied nicht zu hoch ist.
Genauer gesagt, darf der Preisaufschlag von $A$ nicht höher sein als die Transportkosten von Wettbewerber $B$ nach $A$: \citep[S.\,45-46]{hotelling1929} (eigene Gleichung)
\begin{equation}\label{eqn:Preisannahme}
\begin{split}
p_1 &\leq p_2 + c(l-a-b) \\
p_2 &\leq p_1 + c(l-a-b)
\end{split}
\end{equation}
So lange die Bedingung aus Gleichung \ref{eqn:Preisannahme} erfüllt ist, kaufen alle Konsumenten links von $A$ bei Wettbewerber $A$ und alle rechts von $B$ bei $B$.
Die Konsumenten in der Mitte teilen sich in einen Abschnitt $x$, der bei $A$ kauft, und einen Abschnitt $y$, der bei $B$ kauft.

Daraus ergeben sich zwei Bedingungen. Erstens ergeben die Längen zusammen die gesamte Länge und zweitens soll der Konsument der die beiden Abschnitte $x$ und $y$ trennt indifferent zwischen den beiden Wettbewerbern sein \citep[S.\,46]{hotelling1929}. Dies ist in Gleichung \ref{eqn:Gleichgewichtsbedingungen} formuliert.
\begin{equation}\label{eqn:Gleichgewichtsbedingungen}
\begin{split}
l &= a + x + y + b \\
p_1 + cx &= p_2 + cy
\end{split}
\end{equation}

Wird dieses lineare Gleichungssystem nach $x$ und $y$ aufgelöst, ergibt sich \citep[S.\,46]{hotelling1929}
\begin{equation}
\begin{split}
x &= \tfrac{1}{2} \left(l-a-b+\frac{p_2-p_1}{c}\right)\\
y &= \tfrac{1}{2} \left(l-a-b+\frac{p_1-p_2}{c}\right)
\end{split}
\end{equation}

Ohne Beschränkung der Allgemeinheit werden die Kosten der Wettbewerber Null gesetzt und der Gewinn ist somit \citep[S.\,46]{hotelling1929}
\begin{equation}
\begin{split}
\pi_1 &= p_1 q_1 = p_1 (a+x) = \tfrac{1}{2} p_1 \left(l+a-b\right) -\frac{p_1^2}{c}+\frac{p_1 p_2}{c}\\
\pi_2 &= p_2 q_2 = p_2 (b+y) = \tfrac{1}{2} p_1 \left(l-a+b\right) -\frac{p_2^2}{c}+\frac{p_1 p_2}{c}
\end{split}
\end{equation}

Der gewinnmaximierende Preis, berechnet mithilfe der Gleichungen $\frac{\partial \pi_1}{\partial p_1}=0$ und $\frac{\partial \pi_2}{\partial p_2}=0$, ist \citep[S.\,46]{hotelling1929}
\begin{equation}
\begin{split}
p_1^* &= \left(l+\frac{a-b}{c}\right) \\
p_2^* &= \left(l-\frac{a-b}{c}\right)
\end{split}
\end{equation}
mit den verkauften Mengen
\begin{equation}
\begin{split}
q_1^* &= a + x = \tfrac{1}{2} \left(l + \frac{a-b}{3}\right) \\
q_2^* &= b + y = \tfrac{1}{2} \left(l - \frac{a-b}{3}\right)
.\end{split}
\end{equation}

Der Gewinn der beiden Wettbewerber im Preisgleichgewicht ist dann \citep[S.\,50]{hotelling1929}
\begin{equation}\label{eqn:Gewinnfunktion}
\begin{split}
\pi_1^* &= \frac{c}{2} \left(l+\frac{a-b}{3}\right)^2 \\
\pi_2^* &= \frac{c}{2} \left(l-\frac{a-b}{3}\right)^2
.\end{split}
\end{equation}

Mithilfe dieser analytischen Lösung können nun weitere Berechnungen angestellt werden.
Im Folgenden werden direkte Schlussfolgerungen vorgestellt.

\subsection{Schlussfolgerungen}
\paragraph{Prinzip der minimalen Differenzierung}
Ist einer der Wettbewerber auf seine Position fixiert, so kann der andere Wettbewerber seine Position so anpassen, dass sein Gewinn maximiert wird.
Dies kann nicht durch Differenzierung ermittelt werden, da die Gewinnfunktion kein eindeutiges Maximum hat.
Es gilt jedoch für alle $a<l$, dass $\frac{\partial \pi_1^*}{\partial a} > 0$. Deswegen strebt Wettbewerber $A$, danach sich möglichst nah an der Mitte zu positionieren.
Für Wettbewerber $B$ gilt das symmetrische Argument. \citep[S.\,51-52]{hotelling1929}

Daraus leitet sich ab, dass die Wettbewerber stets zueinander streben und sich minimal differenzieren.
Dieses Prinzip gilt explizit auch für andere räumliche Wettbewerbe.
Dazu zählt \citet{hotelling1929} auch Produktdifferenzierung und explizit auch den politischen Wettbewerb. \citep[S.\,54-55]{hotelling1929}

\paragraph{Wohlfahrt}
Um die Wohlfahrt zu maximieren, sollten die Wettbewerber sich gleichmäßig über den Bereich der Konsumenten verteilen \citep[S.\,53]{hotelling1929}.
Das bedeutet in diesem Fall $a=b=\frac{l}{4}$. Hotelling betrachtet dies, dem Zeitgeist entsprechend, unter dem Titel Sozialismus versus Kapitalismus und schließt, dass Sozialismus in diesem Fall überlegen ist \citep[S.\,52]{hotelling1929}.

\section{Modell von d‘Aspremont}\label{Sec-Aspremont}

\citet{aspremont1979} kritisieren das Modell und das Prinzip der minimalen Differenzierung von \citet{hotelling1929}.
Die Arbeit stützt sich auf zwei Aspekte: Erstens ist im Modell von \citet{hotelling1929} kein Gleichgewichtspreis definiert, wenn die beiden Wettbewerber sich am selben Punkt befinden.
Zweitens stellen die Autoren ein Modell mit einer anderen Transportkostenfunktion vor, das drastisch andere Lösungen hat. \citep[S.\,1145]{aspremont1979}

\subsection{Kritik an Hotelling}
Der Gleichgewichtspreis ergibt sich als Nash-Cournot-Gleichgewicht.
Dabei müssen zwei Fälle unterschieden werden: Erstens wenn sich die Wettbewerber am selben Punkt befinden und zweitens wenn die Wettbewerber auseinanderliegen.

\paragraph{Wettbewerber am selben Punkt}
Im Fall, dass die Wettbewerber sich beide am selben Punkt, beispielsweise in der Mitte, befinden, so herrscht reiner Preiswettbewerb.
Nach \citet{bertrand1883theorie} führt das zu Preisen in Höhe der Stückkosten, in diesem Fall zu einem Preis $p_1=p_2=0$. \citep[S.\,1146-1147]{aspremont1979}

\paragraph{Wettbewerber an verschiedenen Punkten}
Liegen die Wettbewerber auseinander, so stimmen \citet{aspremont1979} mit \citet{hotelling1929} überein und präzisieren das Marktgleichgewicht.
Der Gleichgewichtspreis ist dann
\begin{equation}
\begin{split}
p_1^* &= c \left(l+\frac{a-b}{3}\right) \\
p_2^* &= c \left(l-\frac{a-b}{3}\right)
\end{split}
\end{equation}
unter der Bedingung
\begin{equation}
\begin{split}
\left(l+\frac{a-b}{3}\right)^2 &\geq \tfrac{4}{3} l (a+2b) \\
\left(l-\frac{a-b}{3}\right)^2 &\geq \tfrac{4}{3} l (2a+b)
.\end{split}
\end{equation}
Ist diese Bedingung nicht erfüllt, so kollabiert das Duopol zu einem Monopol. \citep[S.\,1146]{aspremont1979}

\paragraph{Das Gleichgewicht tendiert zur Mitte und kollabiert.}
Da wie oben beschrieben $\frac{\partial \pi_1^*}{\partial a}>0$ und $\frac{\partial \pi_2^*}{\partial b}>0$ gilt, bewegen sich beide Wettbewerber zur Gewinnoptimierung zur Mitte hin.
Allerdings existiert in der Mitte nur noch das Gleichgewicht mit Nullpreisen.
Da dies kein wirkliches Marktgleichgewicht ist, ist somit laut \citet{aspremont1979} die Lösung von \citet{hotelling1929} invalidiert. \citep[S.\,1147-1148]{aspremont1979} 

\subsection{Alternatives Modell}\label{Sec-Aspremont-Modell}
\citet{aspremont1979} schlagen stattdessen ein alternatives Modell mit einer veränderten Transportkostenfunktion vor.
Diese ist dann quadratisch im Abstand $d$, also $cd^2$ statt $cd$.
Das Gleichgewicht berechnet sich identisch zum Fall oben.
Die Gleichgewichtspreise sind dann  \citep[S.\,1149]{aspremont1979}
\begin{equation} \label{eqn:aspremont-prices}
\begin{split}
p_1^* &= c (l-a-b) \left(l+\frac{a-b}{b}\right) \\
p_2^* &= c (l-a-b) \left(l-\frac{a-b}{b}\right)
\end{split}
\end{equation}

Die Gleichgewichtsmenge ist gegeben als \citep[S.\,1148]{aspremont1979}
\begin{equation} \label{eqn:aspremont-quantities}
q_1(p_1,p_2) = \begin{cases}
	a+ \frac{p_2-p_1}{2c(l-a-b)} + \frac{l-a-b}{2} & 0\leq a+ \frac{p_2-p_1}{2c(l-a-b)} + \frac{l-a-b}{2} \leq l\\
	l & a+ \frac{p_2-p_1}{2c(l-a-b)} + \frac{l-a-b}{2}>l\\
	0 & a+ \frac{p_2-p_1}{2c(l-a-b)} + \frac{l-a-b}{2}<0
\end{cases}
\end{equation}

Setzt man die Gleichgewichtspreise aus Gleichungen \ref{eqn:aspremont-prices} in die Gleichgewichtsmengen aus Gleichung \ref{eqn:aspremont-quantities} ein, so kann man daraus den Gewinn im Gleichgewicht $\pi_1^*=p_1^* q_1(p_1^*, p_2^*)$ berechnen.
Der Gewinn im Gleichgewicht ist dann (Quelle: eigene Berechnung)
\begin{equation}
\begin{split}
\pi_1^* &= \frac{c}{2} (l-a-b) \left(l + \frac{a-b}{2} \right)^2\\
\pi_2^* &= \frac{c}{2} (l-a-b) \left(l - \frac{a-b}{2} \right)^2
.
\end{split}
\end{equation}
Daraus lassen sich wiederum einige Schlussfolgerungen ableiten.

\subsection{Schlussfolgerungen}

\paragraph{Prinzip der maximalen Differenzierung}
Daraus lassen sich die Ableitungen der Gewinnfunktion im Gleichgewicht berechnen.
Sie sind (Quelle: eigene Berechnung)
\begin{equation}
\frac{\partial \pi_1^*}{\partial a} = -\frac{c}{2}\left(l+\frac{a-b}{3}\right)^2 + \frac{c}{3} (l-a-b) \left( l+\frac{a-b}{3} \right)
.
\end{equation}
Dieser Wert ist strikt negativ, da der erste Term negativ und der zweite strikt positiv ist, der erste Term im Absolutwert jedoch stets dominiert.
Der erste Term ist im Absolutwert strikt größer als der zweite, weil $\frac{c}{2} > \frac{c}{3}$ und $\left(l+\frac{a-b}{3}\right) \geq (l-a-b)$. (Quelle: eigene Berechnung)

Die Ableitungen $\frac{\partial \pi_1^*}{\partial a}<0$ und $\frac{\partial \pi_2^*}{\partial b}<0$ sind also negativ \citep[S.\,1149]{aspremont1979}.
Somit bewegen sich die beiden Wettbewerber so weit wie möglich voneinander weg. Das ist das Prinzip der maximalen Differenzierung.

\section{Agentenbasierte Ansätze}\label{Sec-ABM}

Agentenbasierte Ansätze sind in der Lage eine größere Komplexität an Modellen abzubilden.
In Kapitel \ref{sec:ABM-Komplexität} wird dargelegt, dass in der Politik viele verschiedene Aspekte zu berücksichtigen sind.
Dadurch werden die Modelle größer und komplexer und lassen sich unter Umständen nicht mehr analytisch lösen.

Diese komplexen Modelle sollen im besten Fall sowohl mathematisch gelöst als auch computersimuliert werden. Dabei soll sich das Modell auf seine wesentlichen Elemente konzentrieren. Erst wenn sich das Modell trotzdem als zu komplex darstellt soll das Modell ausschließlich als agentenbasiertes Modell formuliert und gelöst werden. \citep[S.\,13]{marchi2014ABMs}

\subsection{Komplexität von Politik} \label{sec:ABM-Komplexität}
\paragraph{Interpretation Hotelling}
Die bisher vorgestellten Modelle bilden einen eindimensionalen räumlichen Preiswettbewerb ab.
Das kann so interpretiert werden, dass die Parteien eine politische Position anbieten und außerdem einen Preis festlegen, beispielsweise in Form von Steuererhöhungen- oder senkungen.
Im Modell wird die politische Position als räumlicher Wettbewerb abgebildet.
Der Geldanreiz wird als Preiswettbewerb dargestellt.

\paragraph{Andere Aspekte}
Politik ist real sehr viel komplexer.
Die möglichen Einflussfaktoren sind quasi endlos.
Im Folgenden werden folgende ausgewählte Aspekte erläutert, die eine Rolle spielen können:
\begin{itemize}
	\item Wahlrecht
	\item Koalitionen
	\item Dynamik
\end{itemize}

\paragraph{Wahlrecht}
Das Wahlrecht spielt eine wesentliche Rolle für die Strategien der Parteien, da es gewissermaßen die Spielregeln der politischen Machtverteilung definiert.
Es wird im Wesentlichen zwischen Mehrheitswahlrecht und Verhältniswahlrecht unterschieden.
Beim Mehrheitswahlrecht werden jeweils diejenigen Kandidaten gewählt, die in ihrem Wahlbezirk die meisten Stimmen bekommen.
Aus parteipolitischer Perspektive degeneriert das politische System meistens zu einem Zweiparteiensystem, beispielsweise gibt es in Großbritannien und USA im Wesentlichen nur zwei Parteien \citep[S.\,257-258]{schofield1998germany}.
Dagegen gilt in Deutschland hauptsächlich das Verhältniswahlrecht.
Dabei spiegeln die Anteile der Parlamentarier im Parlament die Wahlstimmenanteile wider.
Dass Parteien auf diese Art und Weise beliebig klein werden können, wird in Deutschland durch eine 5\%-Hürde begrenzt.
Das sorgt für eine völlig eigene Dynamik rund um die 5\%-Hürde.

\paragraph{Koalitionen}
Eine Regierungskoalition stellt in einer parlamentarischen Demokratie zugleich die Regierung und übt somit politische Macht aus.
Nur so können eigene Standpunkte und Ideen durchgesetzt werden. 
Genau diese Idee steht bei \citet{laver1990coalitions} im Mittelpunkt des Modells.
In diesem Modell stellen die Koalitionspartner aus ihren Reihen die Exekutive, das heißt sie besetzen die Ministerämter. Die Autoren nehmen dabei an, dass eine Partei, die ein Ministeramt innehat, die volle Kontrolle über diesen Fachbereich hat und innerhalb dieses Fachbereichs die eigene Parteiposition ohne Rücksicht auf Koalitionspartner durchsetzt. Dies beschränkt die glaubwürdigen Koalitionen auf diskrete Kombinationen der Standpunkte der Parteien auf festgelegten Politikfeldern. Insbesondere schließt das Koalitionen aus, bei denen Kompromisse innerhalb eines Politikfelds geschlossen werden. \citep[S.\,873-875]{laver1990coalitions}

Eine Regierungskoalition stellt den status quo dar. Damit der status quo durch eine andere Koalition abgelöst werden kann müssen mehrere Voraussetzungen erfüllt sein. Erstens muss jeder einzelne Koalitionspartner in der neuen Koalition die vorgeschlagene Ministerialverteilung gegenüber der alten Ministerialverteilung präferieren. Zweitens unterstellen \citet{laver1990coalitions} den Parteien die Rationalität das komplexe Spiel von Gegenvorschlägen durchrechnen zu können. Dadurch kann eine Koalition nur abgelöst werden, wenn alle beteiligten Parteien diese Dynamiken gegenüber dem status quo präferieren. \citep[S.\,877-878]{laver1990coalitions}

Dieses Modell wird auf die politische Situation in Island 1967 und 1971 angewendet. Dabei werden zwei politische Dimensionen mit vier (1967) beziehungsweise fünf (1971) Parteien betrachtet \citep[S.\,882-885]{laver1990coalitions}. \citet{laver1990coalitions} bezeichnen ihr Modell berechtigterweise nicht als agentenbasiertes Modell. Dennoch eignet es sich hervorragend als agentenbasiertes Modell, da es die Dynamik von neuen Koalitionen zwischen Wahlen gut beschreiben kann und die Aktuere klare Verhaltensmuster haben. Dabei können verschiedene Szenarien bezüglich der Rationalität der Parteien durchgespielt werden. Beispielsweise können die Parteien wie im Modell als vollständig rational betrachtet werden. Jedoch kann alternativ auch angenommen werden, dass die Parteien jeglicher Koalition zustimmen, die sie dem status quo gegenüber präferieren.

Aufgrund der vielfältigen Kombinationsmöglichkeiten der Parteien werden diese Modelle sehr schnell sehr komplex, insbesondere bei mehr als drei Parteien.
Das hier beschriebene Modell setzt außerdem voraus, dass die politischen Dimensionen, in denen die Positionen der Parteien festgelegt werden, klar definiert und abgegrenzt sind.
Die in dieser Arbeit verwendeten Daten vermischen jedoch politische Dimensionen und sind daher nicht geeignet um Koalitionen auf diese Art und Weise zu analysieren.

\paragraph{Dynamik}
Zuletzt ist Politik keinesfalls statisch sondern dynamisch. Einfache Modelle betrachten Politik als rundenbasiert, zum Beispiel zwischen den Wahlen, wohingegen komplexere Modelle die Phasen zwischen den Wahlen abbilden \citep[S.\,263-264]{laver2005policy}.
Das Modell von \citet{laver2005policy} wird in Kapitel \ref{sec:ABM-Dynamik} genauer betrachtet und in Kapitel \ref{Sec:Dynamik-Anwendung} auf reale Daten angewendet.

\subsection{Einführung in Agentenbasierte Modelle}

Agentenbasierte Modelle zeichnen sich dadurch aus, dass die Modellbeschreibung sich auf einzelne Bestandteile, die Agenten, beziehen. Zusammen bilden die Agentenbeschreibung und ihre Interaktionen das Modell, anstatt eine globale Beschreibung des Modells zu haben. \citet{marchi2014ABMs} besprechen eingehend agentenbasierende Modelle. Sie beschreiben die Vielfalt an Modellen und erklären ihre Anwendungsfelder.

\paragraph{Agenten}
Im Zentrum der Modelle stehen die Agenten mit ihren Eigenschaften und Verhaltensregeln.
Die Anzahl der Agenten schwankt normalerweise zwischen 2 und 10000.
Die Agenten haben eine vorher festgelegte Anzahl an Eigenschaften.
Diese können sich über die Zeit ändern.
Das bedeutet zu jedem Zeitpunkt können die Eigenschaften der Agenten durch einen Vektor $a_j^t = (x_{j1}^t, x_{j2}^t,..., x_{jM}^t)$ beschrieben werden.
Diese Eigenschaften bieten eine große Flexibilität des Modells, da sie fast alles codieren können. \citep[S.\,7]{marchi2014ABMs}

Im Kontext politischer Modelle spielen insbesondere die Eigenschaften von Wählern und Parteien eine Rolle.
Für ein räumliches politisches Modell codieren die Eigenschaften also die Positionen der Parteien.
Für die Wähler kann dort codiert sein, welche Partei sie momentan präferieren beziehungsweise welche Partei sie in der Vergangenheit gewählt haben.

\paragraph{Verhaltensregeln}
Die Verhaltensregeln beschreiben wie sich die Eigenschaften der Agenten von einem Zeitpunkt zum nächsten verändern. In einfachen Modellen hängt das ausschließlich von den Eigenschaften des vorherigen Zeitpunkts ab. In komplizierteren Modellen kann dies auch von weiteren vorherigen Zeitpunkten abhängen. Diese Verhaltensregeln beschreiben insbesondere die Rationalität der Agenten. Agenten mit beschränkter Rationalität betrachten nur einen kleinen Ausschnitt der Informationen und ziehen mögliche Nebenwirkungen ihrer Entscheidungen nicht in Betracht. Dagegen nutzen Agenten mit vollkommener Rationalität alle verfügbaren Informationen und Berücksichtigen bei ihrer Entscheidung die Reaktionen anderer Akteure. Auf diesem Spektrum der Rationalität können Modelle berücksichtigen, dass Agenten lernfähig sein können. \citep[S.\,8]{marchi2014ABMs}

\paragraph{Äußere Regeln}
Zuletzt muss dass Modell genau festlegen in welcher Reihenfolge die Akteure agieren beziehungsweise wie ein gleichzeitiges Handeln der Agenten behandelt werden soll.
Eine solche Reihenfolge kann zufällig, aufgrund von Eigenschaften, oder endogen im Modell entschieden werden \citep[S.\,8]{marchi2014ABMs}.
Beispielsweise können die Entscheidungen zweier Agenten A und B gegenseitig von der Position des anderen abhängen. Wenn zuerst Agent A agiert und dann Agent B die neue Position von A berücksichtigt, so führt das zu einem anderen Ergebnis, als wenn zuerst Agent A agiert und dann Agent B auf die Position von A reagiert. 

\paragraph{Beispiele agentenbasierter Modelle}

Im Prinzip können die meisten analytisch lösbaren Modelle auch agentenbasiert gelöst werden. Beispielsweise können die Modelle von \citet{hotelling1929} und \citet{aspremont1979} agentenbasiert gelöst werden.
Das Koalitionsmodell von \citet{laver1990coalitions} wird zwar analytisch gelöst, kann jedoch genauso auch agentenbasiert betrachtet werden.

Die Modelle von \citet{kollman1992adaptive} und \citet{kollman1998political} stellen als Zweiparteienmodelle die Dynamik der Parteien ins Zentrum.
Sie werden im nächsten Absatz erläutert.
Darauf baut das Mehrparteienmodell von \citet{laver2005policy} auf. Dieses Modell wird separat in Kapitel \ref{sec:ABM-Dynamik} beschrieben, da es Teil der empirischen Untersuchung in Kapitel \ref{Sec:Dynamik-Anwendung} ist.

\paragraph{Dynamisches Zweiparteienmodell}

\citet{kollman1992adaptive} entwickeln ein Modell mit zwei Parteien, das klären soll, ob es unter verschiedenen Annahmen ein Gleichgewicht gibt, zu dem die Parteien streben. In dem agentenbasierten Modell versuchen die Parteien wahlweise mit einem Wahlprogramm möglichst nah an ihrer Idealposition zu gewinnen oder nur zu gewinnen. Dabei wissen die Parteien nicht genau über die Wähler Bescheid, sondern können nur auf ihre Umfragewerte reagieren und ihre Position entsprechend anpassen. Die Präferenz der Wähler drückt sich in einer Optimalposition im mehrdimensionalen Positionsraum und eine Gewichtung der Politikfelder aus. 
\citep[S.\,930-931]{kollman1992adaptive}

Für jede Wahl beziehungsweise jeden Zeitschritt bleibt die Regierungspartei statisch und die Oppositionspartei versucht zu gewinnen. Dabei stellen die Autoren fest, dass bei zwei Parteien, die lediglich nach Wahlgewinn streben, der Herausforderer nach nur zwölf Wahlen nur noch eine Chance von 40 Prozent hat, die Regierung zurückzugewinnen. Dagegen wechselt die Regierung bei zwei ideologische Parteien hin und her und auch nach mehreren Wahlen bleiben die Chancen gut, dass die Oppositionspartei gewinnt. \citep[S.\,934-5]{kollman1992adaptive}

\citet{kollman1998political} untersuchen dieses Modell mit seinen adaptiven Parteien genauer hinsichtlich der Wählerverteilung. Nun gibt es Stellen mit hohen und geringen Wählerdichten. Dabei stellt sich heraus, dass die Parteien zwar immer noch moderate Positionen finden, die Wahlen gewinnen können, es ist jedoch deutlich erschwert. Das bedeutet, dass der Vorteil der Regierungspartei noch größer ist und dass es je nach Unregelmäßigkeit der Wählerverteilung länger braucht, bis ein Optimum gefunden wird. \citep{kollman1998political}

\subsection{Dynamik von Parteien} \label{sec:ABM-Dynamik}

Als Weiterentwicklung der Modelle von \citet{kollman1992adaptive} und \citet{kollman1998political} baut \citet{laver2005policy} sein dynamisches Modell mit Parteitypen auf.
Dieser dynamische Ansatz stelle die Realität gut dar, in der die Parteien kontinuierlich ihre Positionen anpassen, wie schwankende Umfragewerte der Parteien belegen würden \citep[S.\,263-264]{laver2005policy}.

Das Modell von \citet{laver2005policy} vereinigt dabei Ansätze der klassischen räumlichen Modelle mit den dynamischen Ansätzen verschieder Parteitypen. Die Parteitypen haben gemeinsam, dass sie ihre Position auf die aktuelle politische Situation anpassen. Sie unterscheiden sich jedoch darin, wie sie diese Anpassung vornehmen. Laver unterscheidet dabei die vier Parteitypen Aggregator, Hunter, Predator und Sticker \citep[S.\,266-267]{laver2005policy}. In dieser Arbeit werden die original englischen Begriffe beibehalten. Im Folgenden werden die Parteitypen genauer definiert.

\paragraph{Aggregator} Der Aggregator sammelt die Meinungen seiner momentanen Wähler. Die neue Position ist die Durchschnittsposition dieser momentanen Wähler \citep[S.\,267]{laver2005policy}. Dies entspricht einer Partei die sich basisdemokratisch organisiert.

\paragraph{Hunter} Der Hunter versucht möglichst viele Wähler hinzuzugewinnen. Dazu verwendet er eine beschränkt rationale Strategie die auf dem Erfolg der vorhergegangenen Positionsanpassungen basiert. Hat die vorherige Positionsanpassung Wähler hinzugewonnen, so wird eine weiterer Schritt in die selbe Richtung gegangen. Ansonsten wird ein Schritt in die umgekehrte Richtung gegangen, mit einer zufälligen Komponente von bis zu 90 Grad \citep[S.\,267]{laver2005policy}. Dieses Verhalten entspricht einer Partei, in der die Parteiführung völlig freie Hand hat bei der Positionswahl und danach strebt, möglichst viele Wähler zu gewinnen.

\paragraph{Predator} Der Predator passt seine Position nicht aufgrund der Wähler an, sondern aufgrund der Positionen der anderen Parteien und deren Zustimmungswerten. Ist der Predator selbst die größte Partei, so behält er seine Position. Ansonsten bewegt er sich in Richtung der Position der größten Partei. Dies entspricht einer Partei mit einer starken Führung, die versucht möglichst viele Wähler zu gewinnen. Im Unterschied zum Hunter nimmt sie nicht die Zustimmungsänderung als Indikator, sondern die Partei mit der meisten Zustimmung als Indikator, wo die meisten Wähler zu gewinnen sind \citep[S.\,267]{laver2005policy}.

\paragraph{Sticker} Der Sticker ändert seine Position nicht. Diese Regel kann einer ideologischen Partei entsprechen, die auf ihren Positionen festgefahren ist. Diese Partei ist an sich nicht interessant, kann jedoch spannende Wechselwirkungen mit anderen Parteitypen haben \citep[S.\,267]{laver2005policy}.

\paragraph{Ergebnisse}
Laut \citet{laver2005policy} ist das bemerkenswerteste Ergebenis, dass der Parteityp Hunter sehr erfolgreich ist. Besteht das Modell ausschließlich aus Huntern so bewegen sie sich zum Zentrum hin. Bei genau zwei Parteien bewegen sich beide sehr nah zum Zentrum, ähnlich wie bei \citet{hotelling1929}. Bei mehr Huntern halten sie einen größeren Abstand zum Zentrum, da sie sonst Wähler an die anderen Hunter verlieren würden. \citep[S.\,267-270]{laver2005policy}

Sind dagegen nur Parteien vom Typ Aggregator im System, so lassen sich die Positionen der Parteien nur schwer vorhersagen. Die Parteien verteilen sich relativ gleichmäßig über den Bereich der Wähler. Das liegt daran, dass sich die Aggregatoren die Wähler nicht bewusst streitig machen, indem sie sich auf die andere Partei zubewegen würden. \citep[S.\,270-271]{laver2005policy}

Die Fälle mit ausschließlich Stickern und ausschließlich Predatoren sind trivial. Das Modell mit nur Stickern bleibt statisch. Im Modell mit nur Predatoren sammeln sich alle im Zentrum, ähnlich der Situation in \citet{hotelling1929}.

Beinhaltet das Modell verschiedene Parteitypen werden die Interaktionen komplexer. Beinhaltet das Modell mehrere Aggregatoren und einzelne Hunter oder Predatoren, so konkurrieren die Hunter und Predatoren in der Mitte des Positionsraums. Sind sie alleine, so sind sie dort die größte Partei. Gibt es jedoch noch weiter Hunter oder Predatoren schneiden sie weniger gut ab. Die Aggregatoren haben zwar geringere Wähleranteile, jedoch sind sie sehr robust und werden kaum marginalisiert, da sie stets ihre Wählerbasis bedienen.
Für ein Modell mit einem Predator und mehreren Huntern stellt der Autor fest, dass die Hunter erfolgreicher sind, da der Predator ständig zwischen den Positionen wechselt und somit keine aussichtsreiche Position festigt. \citep[S.\,271-274]{laver2005policy}

\paragraph{Weiterführende Modelle}
Laver hat dieses Modell in weiteren Arbeiten erweitert und verfeinert.
Im Modell von \citet{laver2007endogenousParties} wird die Anzahl der Parteien nicht mehr zu Simulationsbeginn exogen festgelegt, sondern ist ein endogener Bestandteil des Modells.
Manche Parteien sterben und es kommen neue Parteien hinzu.
Im Buch \citet{laver2011party} geben die Autoren einen umfassenden Überblick über die Modelle von Laver. Außerdem werden viele Details und Weiterentwicklungen der Modelle vorgestellt.