\chapter{Theoretische Analyse}\label{Kap-Theorie}

\noindent
Im Folgenden werden unterschiedliche Modelle des räumlichen Wettbewerbs vorgestellt. Jedes Modell wird zunächst in seiner ursprünglichen Form gezeigt. Anschließend wird analysiert inwiefern sich das Modell auf politische Gegebenheiten anwenden lässt und welche Vorhersagen sich daraus ergeben.

\section{Hotellings Gesetz}\label{Sec-Hotelling}

\subsection{Modell}\label{Sec-Hotelling-Modell}

Hotelling \citep{hotelling1929} untersucht anhand eines einfachen Modells wie sich bei einem Duopol der räumliche Wettbewerb gestaltet. Dazu nimmt er an, dass die zwei Wettbewerber auf einer räumlichen Linie ihre Position wählen können. Abbildung \ref{Fig-Linearer-Wettbewerb} zeigt mögliche Positionen der Wettbewerber. Dabei sind die Konsumenten gleichmäßig über die gesamte Linie verteilt.

\begin{figure}[htb]
	\centering
	\begin{tikzpicture}
	\coordinate (start) at (0,0) {};
	\coordinate (end) at (10,0) {};
	\coordinate (left) at (2,0) {};
	\coordinate (right) at (7,0) {};
	\draw (start) -- (end);
	\draw ($(start)+(0,5pt)$) -- ($(start)-(0,5pt)$);
	\draw ($(left)+(0,5pt)$) -- ($(left)-(0,5pt)$);
	\draw ($(right)+(0,5pt)$) -- ($(right)-(0,5pt)$);
	\draw ($(end)+(0,5pt)$) -- ($(end)-(0,5pt)$);
	\node at ($(start)+(0,10pt)$) {$0$};
	\node at ($(end)+(0,10pt)$) {$l$};
	\node at ($(left)+(0,10pt)$) {$A$};
	\node at ($(right)+(0,10pt)$) {$B$};
	
	\coordinate (startdown) at ($(start)+(0,-20pt)$) {};
	\coordinate (enddown) at ($(end)+(0,-20pt)$) {};
	\coordinate (leftdown) at ($(left)+(0,-20pt)$) {};
	\coordinate (rightdown) at ($(right)+(0,-20pt)$) {};
	\coordinate (middledown) at (4,-20pt) {};
	
	\draw[|-|] (startdown) -- (leftdown) node[midway, below] {$a$};
	\draw[|-|] (rightdown) -- (enddown) node[midway, below] {$b$};
	\draw[|-|] (leftdown) -- (middledown) node[midway, below] {$x$};
	\draw[|-|] (middledown) -- (rightdown) node[midway, below] {$y$};
	\end{tikzpicture}
	\caption{Positionen der Wettbewerber $A$ und $B$ und Konsumenten auf einer Linie von $0$ bis $l$. Die Konsumenten im Bereich $a$ und $x$ kaufen bei $A$ ein und $y$ und $b$ bei $B$. (angelehnt an \cite{hotelling1929})}
	\label{Fig-Linearer-Wettbewerb}
\end{figure}

Eine grundlegende Annahme des Modells ist, dass die Konsumenten lineare Transportkosten in Höhe von $c$ pro Längeneinheit haben. Die Konsumenten kaufen somit bei demjenigen Duopolisten bei dem die Kosten bestehend aus Preis und Transportkosten am geringsten sind.

Nun nimmt Hotelling an, dass $A$ zu einem Preis $p_1$ und $B$ zu einem Preis $p_2$ verkauft. Die Verkaufsmengen sind jeweils $q_1$ und $q_2$.

Die erste wichtige Feststellung ist, dass das Duopol nur dann ein Duopol bleibt und nicht zu einem Monopol kollabiert, wenn der Preisunterschied nicht zu hoch ist. Genauer gesagt, darf der Preisaufschlag von $A$ nicht höher sein als die Transportkosten von Wettbewerber $B$ nach $A$:
\begin{equation}\label{eqn:Preisannahme}
\begin{split}
p_1 &\leq p_2 + c(l-a-b) \\
p_2 &\leq p_1 + c(l-a-b)
\end{split}
\end{equation}
So lange Bedingung \ref{eqn:Preisannahme} erfüllt ist, kaufen alle Konsumenten links bei Wettbewerber $A$ und alle rechts bei $B$. Die Konsumenten in der Mitte teilen sich in einen Abschnitt $x$, der bei $A$ kauft, und einen Abschnitt $y$, der bei $B$ kauft.

Daraus ergeben sich zwei Bedingungen. Erstens ergeben die Längen zusammen die gesamte Länge und zweitens soll der Konsument der die beiden Abschnitte $x$ und $y$ trennt indifferent zwischen den beiden Wettbewerbern sein. Dies ist in Gleichung \ref{eqn:Gleichgewichtsbedingungen} formuliert.
\begin{equation}\label{eqn:Gleichgewichtsbedingungen}
\begin{split}
l &= a + x + y + b \\
p_1 + cx &= p_2 + cy
\end{split}
\end{equation}

Löst man dieses lineare Gleichungssystem nach $x$ und $y$ auf ergibt sich
\begin{equation}
\begin{split}
x &= \tfrac{1}{2} \left(l-a-b+\frac{p_2-p_1}{c}\right)\\
y &= \tfrac{1}{2} \left(l-a-b+\frac{p_1-p_2}{c}\right)
\end{split}
\end{equation}

Ohne Beschränkung der Allgemeinheit werden die Kosten der Wettbewerber Null gesetzt und der Gewinn ist somit
\begin{equation}
\begin{split}
\pi_1 &= p_1 q_1 = p_1 (a+x) = \tfrac{1}{2} p_1 \left(l+a-b\right) -\frac{p_1^2}{c}+\frac{p_1 p_2}{c}\\
\pi_2 &= p_2 q_2 = p_2 (b+y) = \tfrac{1}{2} p_1 \left(l-a+b\right) -\frac{p_2^2}{c}+\frac{p_1 p_2}{c}
\end{split}
\end{equation}

Der gewinnmaximierende Preis, berechnet mithilfe der Gleichungen $\frac{\partial \pi_1}{\partial p_1}=0$ und $\frac{\partial \pi_2}{\partial p_2}=0$, ist
\begin{equation}
\begin{split}
p_1 &= \left(l+\frac{a-b}{c}\right) \\
p_2 &= \left(l-\frac{a-b}{c}\right)
\end{split}
\end{equation}
mit den verkauften Mengen
\begin{equation}
\begin{split}
q_1 &= a + x = \tfrac{1}{2} \left(l + \frac{a-b}{3}\right) \\
q_2 &= b + y = \tfrac{1}{2} \left(l - \frac{a-b}{3}\right)
.\end{split}
\end{equation}

Die resultierende Gewinnfunktion der Wettbewerber ist
\begin{equation}\label{eqn:Gewinnfunktion}
\begin{split}
\pi_1 &= \frac{c}{2} \left(l+\frac{a-b}{3}\right) \\
\pi_2 &= \frac{c}{2} \left(l-\frac{a-b}{3}\right)
.\end{split}
\end{equation}

\subsection{Schlussfolgerungen}

\section{Modell von d‘Aspremont}\label{Sec-Aspremont}

d'Aspremont \citep{aspremont1979} kritisiert das Principle of Minimum Differentiation (TODO: kontrollieren, dass definiert) von Hotelling.
Die Kritik stützt sich auf zwei Aspekte: Erstens ist kein Gleichgewichtspreis definiert, wenn die beiden Wettbewerber sich am selben Punkt befinden. Zweitens hat das Modell drastisch andere Lösungen, wenn die Transportkostenfunktion sich ändert.

\subsection{Kritik an Hotelling}
Der Gleichgewichtspreis ergibt sich als Nash-Cournot-Gleichgewicht. Dabei müssen zwei Fälle unterschieden werden: Erstens wenn sich die Wettbewerber am selben Punkt befinden und zweitens wenn die Wettbewerber auseinanderliegen.

Im Fall, dass die Wettbewerber sich beide am selben Punkt (beispielsweise in der Mitte) befinden, so herrscht reiner Preiswettbewerb. Nach Betrand (TODO: zitieren) führt das zu Preisen in Höhe der Stückkosten, in diesem Fall zu einem Preis $p_1=p_2=0$.

Liegen die Wettbewerber auseinander, so stimmt \cite{aspremont1979} mit \cite{hotelling1929} überein und präzisiert das Marktgleichgewicht. Der Gleichgewichtspreis ist dann
\begin{equation}
\begin{split}
p_1 &= c \left(l+\frac{a-b}{3}\right) \\
p_2 &= c \left(l-\frac{a-b}{3}\right)
\end{split}
\end{equation}
unter der Bedingung
\begin{equation}
\begin{split}
\left(l+\frac{a-b}{3}\right)^2 &\geq \tfrac{4}{3} l (a+2b) \\
\left(l-\frac{a-b}{3}\right)^2 &\geq \tfrac{4}{3} l (2a+b)
\end{split}
\end{equation}

\paragraph{Das Gleichgewicht tendiert zur Mitte und kollabiert.} Da $\frac{\partial \pi_1}{\partial a}>0$ und $\frac{\partial \pi_2}{\partial b}>0$ bewegen sich beide Wettbewerber zur Gewinnoptimierung zur Mitte hin. Allerdings existiert in der Mitte nur noch das Gleichgewicht mit Nullpreisen. Da dies kein wirkliches Marktgleichgewicht ist, ist das laut \cite{aspremont1979} das gesamte Modell invalidiert.

\subsection{Modell}\label{Sec-Aspremont-Modell}
\cite{aspremont1979} schlägt stattdessen ein korrigiertes Modell vor mit einer veränderten Transportkostenfunktion vor. Diese ist dann quadratisch, also $cx^2$ statt $cx$.
Das Gleichgewicht berechnet sich identisch zum Fall oben. Die Gleichgewichtspreise sind dann
\begin{equation}
\begin{split}
p_1 &= c (l-a-b) \left(l+\frac{a-b}{b}\right) \\
p_2 &= c (l-a-b) \left(l-\frac{a-b}{b}\right)
\end{split}
\end{equation}

Die Gewinnfunktionen sind dann
\begin{equation}
TODO: Gewinnfunktionen
\end{equation}

Die Ableitungen $\frac{\partial \pi_1}{\partial a}<0$ und $\frac{\partial \pi_2}{\partial b}<0$ sind also negativ. Somit bewegen sich die beiden Wettbewerber so weit wie möglich voneinander weg. Das ist das Prinzip der maximalen Differenzierung.

\subsection{Schlussfolgerungen}

\section{Agentenbasierte Ansätze}\label{Sec-ABM}
Agentenbasierte Ansätze sind in der Lage eine größere Komplexität an Modellen abzubilden. In Kapitel \ref{sec:ABM-Komplexität} wird dargelegt, dass in der Politik viele verschiedene Aspekte zu berücksichtigen sind. Diese Aspekte lassen sich nicht alle in ein globales Modell (TODO check Begriff) integrieren und analytisch lösen. Daher werden agentenbasierte Ansätze verwendet.

\subsection{Komplexität von Politik} \label{sec:ABM-Komplexität}
Politik umfasst viele Aspekte. Die bisher vorgestellten Modelle bilden einen eindimensionalen räumlichen Preiswettbewerb ab. Politik ist real sehr viel komplexer. Im Folgenden werden einige Aspekte erläutert, die eine Rolle spielen können. Dazu gehören:
\begin{itemize}
	\item Wahlrecht
	\item Koalitionen
	\item Personen
	\item Dynamik
\end{itemize}

\paragraph{Wahlrecht}
Das Wahlrecht spielt eine wesentliche Rolle für die Strategien der Parteien, da es gewissermaßen die Spielregeln der politischen Machtverteilung definiert. Es wird im Wesentlichen zwischen Mehrheitswahlrecht und Verhältniswahlrecht unterschieden (TODO Zitat). Beim Mehrheitswahlrecht werden jeweils diejenigen Kandidaten gewählt, die in ihrem Wahlbezirk die meisten Stimmen bekommen. Aus parteipolitischer Perspektive degeneriert das politische System zu einem Zweiparteiensystem (TODO Zitat). Dagegen gilt in Deutschland hauptsächlich das Verhältniswahlrecht. Dabei spiegeln die Anteile der Parlamterier im Parlament die Wahlstimmenanteile wider. Dass Parteien auf diese Art und Weise beliebig klein werden können, wird in Deutschland durch eine 5\%-Hürde begrenzt. Das sorgt für eine völlig eigene Dynamik rund um die 5\%-Grenze.

\paragraph{Koalitionen}
Politische Macht wird in Demokratien hauptsächlich über Gesetzgebung und Regierung ausgeübt (TODO check und Zitat). Nur so können eigene Standpunkte und Ideen durchgesetzt werden. Da eine Partei in Deutschland selten die absolute Mehrheit erringen kann, müssen die Parteien Koalitionen eingehen. Beispielweise beschäftigen sich Laver und Shepsle (TODO Laver Shepsle 1990) mit möglichen Koalitionsmodellen.

Ich möchte hier die Grundidee der Koalitionsmodelle zitieren. [TODO]

Aufgrund der vielfältigen Kombinationsmöglichkeiten der Parteien werden diese Modelle sehr schnell sehr komplex, insbesondere bei mehr als drei Parteien. Das hier beschriebene Modell setzt außerdem voraus, dass die politischen Dimensionen, über die die Standpunkte der Parteien definiert sind, klar definiert und abgegrenzt sind. Das Modell, das hier später beschrieben wird vermischt jedoch politische Dimensionen und ist daher nicht geeignet um Koalitionen auf diese Art und Weise zu analysieren.

\paragraph{Personen}
Ein wesentlicher Faktor sind außerdem die Personen, die die Parteien jeweils vertreten, beispielsweise die Spitzenkandidaten bei Bundestagswahlen. 
[TODO: Beispiel von Wahl und Einfluss von Spitzenkandidaten]

\paragraph{Dynamik}
Zuletzt ist Politik keinesfalls statisch sondern dynamisch. Einfache Modelle betrachten Politik als rundenbasiert (TODO Zitat), zum Beispiel zwischen den Wahlen, wohingegen komplexere Modelle die Phasen zwischen den Wahlen abbilden (TODO Zitat)

Die Dynamik wird in Kapitel \ref{sec:ABM-Dynamik} genauer betrachtet.

\subsection{Einführung in Agentenbasierte Modelle}

Agentenbasierte Modelle zeichnen sich dadurch aus, dass die Modellbeschreibung sich auf einzelne Bestandteile, die Agenten, beziehen. Zusammen bilden die Agentenbeschreibung und ihre Interaktionen das Modell, anstatt eine globale Beschreibung des Modells zu haben. Marchi und Page \citep{marchi2014ABMs} besprechen eingehend agentenbasierende Modelle. Sie beschreiben die Vielfalt an Modellen und erklären ihre Anwendungsfelder.

Im Zentrum der Modelle stehen die Agentenmit ihren Eigenschaften und Verhaltensregeln. Die Anzahl der Agenten kann je nach Modell stark schwanken. Laut \cite{marchi2014ABMs} variiert die Anzahl der Agenten normalerweise zwischen 2 und 10000. Im Zuge der Covid-19 Studien gibt es agentenbasierte Modelle die bis zu ??? Agenten umfassen (TODO Zitat).

Die Agenten haben eine vorher festgelegte Anzahl an Eigenschaften. Diese können sich über die Zeit ändern. Das bedeutet zu jedem Zeitpunkt können die Eigenschaften der Agenten durch einen Vektor $a_j^t = (x_{j1}^t, x_{j2}^t,..., x_{jM}^t)$ beschrieben werden. Diese Eigenschaften bieten eine große Flexibilität des Modells, da sie fast alles codieren können. Im Kontext politischer Modelle spielen insbesondere die Eigenschaften von Wählern und Parteien eine Rolle. Das bedeutet beispielsweise, dass ideologische Positionen bedeutend sind.

Die Verhaltensregeln beschreiben wie sich die Eigenschaften der Agenten von einem Zeitpunkt zum nächsten verändern. In einfachen Modellen hängt das ausschließlich von den Eigenschaften des vorherigen Zeitpunkts ab. In komplizierteren Modellen kann dies auch von weiteren vorherigen Zeitpunkten abhängen. Diese Verhaltensregeln beschreiben insbesondere die Rationalität der Agenten. Agenten mit beschränkter Rationalität betrachten nur einen kleinen Ausschnitt der Informationen und ziehen mögliche Nebenwirkungen ihrer Entscheidungen nicht in Betracht. Dagegen nutzen Agenten mit vollkommener Rationalität alle verfügbaren Informationen und Berücksichtigen bei ihrer Entscheidung die Reaktionen anderer Akteure. Auf diesem Spektrum der Rationalität können Modelle berücksichtigen, dass Agenten lernfähig sein können. Zuletzt muss dass Modell genau festlegen in welcher Reihenfolge die Akteure agieren beziehungsweise wie ein gleichzeitiges Handeln der Agenten behandelt werden soll. Beispielsweise, können die Entscheidungen zweier Agenten A und B gegenseitig von der Position des anderen abhängen. Wenn zuerst Agent A agiert und dann Agent B die neue Position von A berücksichtig, so führt das zu einem anderen Ergebnis, als wenn zuerst Agent A agiert und dann Agent B auf die Position von A reagiert. Eine solche Reihenfolge kann zufällig, aufgrund von Eigenschaften, oder endogen im Modell entschieden werden[TODO Page 1977 zitieren?].

[TODO Beispiele was mit agentenbasierten Modellen gelöst werden kann]

\subsection{Dynamik von Parteien} \label{sec:ABM-Dynamik}

\cite{laver2005policy}: kontinuierliche Evaluation der Wähler

\cite{laver2007endogenousParties}: Parteitypen
