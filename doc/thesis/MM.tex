\chapter{Theoretische Analyse}\label{Kap-Theorie}

\noindent
Im Folgenden werden unterschiedliche Modelle des räumlichen Wettbewerbs vorgestellt. Jedes Modell wird zunächst in seiner ursprünglichen Form gezeigt. Anschließend wird analysiert inwiefern sich das Modell auf politische Gegebenheiten anwenden lässt und welche Vorhersagen sich daraus ergeben.

\section{Hotellings Gesetz}\label{Sec-Hotelling}

\subsection{Modell}\label{Sec-Hotelling-Modell}

Hotelling \citep{hotelling1929} untersucht anhand eines einfachen Modells wie sich bei einem Duopol der räumliche Wettbewerb gestaltet. Dazu nimmt er an, dass die zwei Wettbewerber auf einer räumlichen Linie ihre Position wählen können. Abbildung \ref{Fig-Linearer-Wettbewerb} zeigt mögliche Positionen der Wettbewerber. Dabei sind die Konsumenten gleichmäßig über die gesamte Linie verteilt.

\begin{figure}[htb]
	\centering
	\begin{tikzpicture}
	\coordinate (start) at (0,0) {};
	\coordinate (end) at (10,0) {};
	\coordinate (left) at (2,0) {};
	\coordinate (right) at (7,0) {};
	\draw (start) -- (end);
	\draw ($(start)+(0,5pt)$) -- ($(start)-(0,5pt)$);
	\draw ($(left)+(0,5pt)$) -- ($(left)-(0,5pt)$);
	\draw ($(right)+(0,5pt)$) -- ($(right)-(0,5pt)$);
	\draw ($(end)+(0,5pt)$) -- ($(end)-(0,5pt)$);
	\node at ($(start)+(0,10pt)$) {$0$};
	\node at ($(end)+(0,10pt)$) {$l$};
	\node at ($(left)+(0,10pt)$) {$A$};
	\node at ($(right)+(0,10pt)$) {$B$};
	
	\coordinate (startdown) at ($(start)+(0,-20pt)$) {};
	\coordinate (enddown) at ($(end)+(0,-20pt)$) {};
	\coordinate (leftdown) at ($(left)+(0,-20pt)$) {};
	\coordinate (rightdown) at ($(right)+(0,-20pt)$) {};
	\coordinate (middledown) at (4,-20pt) {};
	
	\draw[|-|] (startdown) -- (leftdown) node[midway, below] {$a$};
	\draw[|-|] (rightdown) -- (enddown) node[midway, below] {$b$};
	\draw[|-|] (leftdown) -- (middledown) node[midway, below] {$x$};
	\draw[|-|] (middledown) -- (rightdown) node[midway, below] {$y$};
	\end{tikzpicture}
	\caption{Positionen der Wettbewerber $A$ und $B$ und Konsumenten auf einer Linie von $0$ bis $l$. Die Konsumenten im Bereich $a$ und $x$ kaufen bei $A$ ein und $y$ und $b$ bei $B$. (eigene Darstellung angelehnt an \citet{hotelling1929})}
	\label{Fig-Linearer-Wettbewerb}
\end{figure}

Eine grundlegende Annahme des Modells ist, dass die Konsumenten lineare Transportkosten in Höhe von $c$ pro Längeneinheit haben. Die Konsumenten kaufen somit bei demjenigen Duopolisten bei dem die Kosten bestehend aus Preis und Transportkosten am geringsten sind.

Nun nimmt Hotelling an, dass $A$ zu einem Preis $p_1$ und $B$ zu einem Preis $p_2$ verkauft. Die Verkaufsmengen sind jeweils $q_1$ und $q_2$.

Die erste wichtige Feststellung ist, dass das Duopol nur dann ein Duopol bleibt und nicht zu einem Monopol kollabiert, wenn der Preisunterschied nicht zu hoch ist. Genauer gesagt, darf der Preisaufschlag von $A$ nicht höher sein als die Transportkosten von Wettbewerber $B$ nach $A$:
\begin{equation}\label{eqn:Preisannahme}
\begin{split}
p_1 &\leq p_2 + c(l-a-b) \\
p_2 &\leq p_1 + c(l-a-b)
\end{split}
\end{equation}
So lange Bedingung \ref{eqn:Preisannahme} erfüllt ist, kaufen alle Konsumenten links bei Wettbewerber $A$ und alle rechts bei $B$. Die Konsumenten in der Mitte teilen sich in einen Abschnitt $x$, der bei $A$ kauft, und einen Abschnitt $y$, der bei $B$ kauft.

Daraus ergeben sich zwei Bedingungen. Erstens ergeben die Längen zusammen die gesamte Länge und zweitens soll der Konsument der die beiden Abschnitte $x$ und $y$ trennt indifferent zwischen den beiden Wettbewerbern sein. Dies ist in Gleichung \ref{eqn:Gleichgewichtsbedingungen} formuliert.
\begin{equation}\label{eqn:Gleichgewichtsbedingungen}
\begin{split}
l &= a + x + y + b \\
p_1 + cx &= p_2 + cy
\end{split}
\end{equation}

Löst man dieses lineare Gleichungssystem nach $x$ und $y$ auf ergibt sich
\begin{equation}
\begin{split}
x &= \tfrac{1}{2} \left(l-a-b+\frac{p_2-p_1}{c}\right)\\
y &= \tfrac{1}{2} \left(l-a-b+\frac{p_1-p_2}{c}\right)
\end{split}
\end{equation}

Ohne Beschränkung der Allgemeinheit werden die Kosten der Wettbewerber Null gesetzt und der Gewinn ist somit
\begin{equation}
\begin{split}
\pi_1 &= p_1 q_1 = p_1 (a+x) = \tfrac{1}{2} p_1 \left(l+a-b\right) -\frac{p_1^2}{c}+\frac{p_1 p_2}{c}\\
\pi_2 &= p_2 q_2 = p_2 (b+y) = \tfrac{1}{2} p_1 \left(l-a+b\right) -\frac{p_2^2}{c}+\frac{p_1 p_2}{c}
\end{split}
\end{equation}

Der gewinnmaximierende Preis, berechnet mithilfe der Gleichungen $\frac{\partial \pi_1}{\partial p_1}=0$ und $\frac{\partial \pi_2}{\partial p_2}=0$, ist
\begin{equation}
\begin{split}
p_1 &= \left(l+\frac{a-b}{c}\right) \\
p_2 &= \left(l-\frac{a-b}{c}\right)
\end{split}
\end{equation}
mit den verkauften Mengen
\begin{equation}
\begin{split}
q_1 &= a + x = \tfrac{1}{2} \left(l + \frac{a-b}{3}\right) \\
q_2 &= b + y = \tfrac{1}{2} \left(l - \frac{a-b}{3}\right)
.\end{split}
\end{equation}

Die resultierende Gewinnfunktion der Wettbewerber ist
\begin{equation}\label{eqn:Gewinnfunktion}
\begin{split}
\pi_1 &= \frac{c}{2} \left(l+\frac{a-b}{3}\right)^2 \\
\pi_2 &= \frac{c}{2} \left(l-\frac{a-b}{3}\right)^2
.\end{split}
\end{equation}

\subsection{Schlussfolgerungen}
\paragraph{Prinzip der minimalen Differenzierung}
Ist einer der Wettbewerber auf seine Position fixiert, so kann der andere Wettbewerber seine Position so anpassen, dass sein Gewinn maximiert wird. Dies kann nicht durch Differenzierung ermittelt werden, da die Gewinnfunktion kein eindeutiges Maximum hat. Es gilt jedoch für alle $a<l$, dass $\frac{\partial \pi_1}{\partial a} > 0$. Deswegen strebt Wettbewerber $A$, danach sich möglichst nah an der Mitte zu positionieren. Für Wettbewerber $B$ gilt das symmetrische Argument. \citep[S.\,51-2]{hotelling1929}

Daraus leitet sich ab, dass die Wettbewerber stets zueinander streben und sich minimal differenzieren.

\section{Modell von d‘Aspremont}\label{Sec-Aspremont}

d'Aspremont \citep{aspremont1979} kritisieren das Prinzip der minimalen Differenzierung von \citet{hotelling1929}.
Die Kritik stützt sich auf zwei Aspekte: Erstens ist kein Gleichgewichtspreis definiert, wenn die beiden Wettbewerber sich am selben Punkt befinden. Zweitens hat das Modell drastisch andere Lösungen, wenn die Transportkostenfunktion sich ändert. \citep[S.\,1145]{aspremont1979}

\subsection{Kritik an Hotelling}
Der Gleichgewichtspreis ergibt sich als Nash-Cournot-Gleichgewicht. Dabei müssen zwei Fälle unterschieden werden: Erstens wenn sich die Wettbewerber am selben Punkt befinden und zweitens wenn die Wettbewerber auseinanderliegen.

Im Fall, dass die Wettbewerber sich beide am selben Punkt (beispielsweise in der Mitte) befinden, so herrscht reiner Preiswettbewerb. Nach Betrand (TODO: zitieren) führt das zu Preisen in Höhe der Stückkosten, in diesem Fall zu einem Preis $p_1=p_2=0$.

Liegen die Wettbewerber auseinander, so stimmt \citet{aspremont1979} mit \citet{hotelling1929} überein und präzisiert das Marktgleichgewicht. Der Gleichgewichtspreis ist dann
\begin{equation}
\begin{split}
p_1 &= c \left(l+\frac{a-b}{3}\right) \\
p_2 &= c \left(l-\frac{a-b}{3}\right)
\end{split}
\end{equation}
unter der Bedingung
\begin{equation}
\begin{split}
\left(l+\frac{a-b}{3}\right)^2 &\geq \tfrac{4}{3} l (a+2b) \\
\left(l-\frac{a-b}{3}\right)^2 &\geq \tfrac{4}{3} l (2a+b)
\end{split}
\end{equation}

\paragraph{Das Gleichgewicht tendiert zur Mitte und kollabiert.} Da $\frac{\partial \pi_1}{\partial a}>0$ und $\frac{\partial \pi_2}{\partial b}>0$ bewegen sich beide Wettbewerber zur Gewinnoptimierung zur Mitte hin. Allerdings existiert in der Mitte nur noch das Gleichgewicht mit Nullpreisen. Da dies kein wirkliches Marktgleichgewicht ist, ist das laut \citet{aspremont1979} das gesamte Modell invalidiert.

\subsection{Alternatives Modell}\label{Sec-Aspremont-Modell}
\citet{aspremont1979} schlagen stattdessen ein alternatives Modell mit einer veränderten Transportkostenfunktion vor. Diese ist dann quadratisch im Abstand, also $cx^2$ statt $cx$.
Das Gleichgewicht berechnet sich identisch zum Fall oben. Die Gleichgewichtspreise sind dann
\begin{equation} \label{eqn:aspremont-prices}
\begin{split}
p_1 &= c (l-a-b) \left(l+\frac{a-b}{b}\right) \\
p_2 &= c (l-a-b) \left(l-\frac{a-b}{b}\right)
\end{split}
\end{equation}

Die Gleichgewichtsmenge ist dann \citep[S.\,1148]{aspremont1979}
\begin{equation} \label{eqn:aspremont-quantities}
q_1(p_1,p_2) = \begin{cases}
	a+ \frac{p_2-p_1}{2c(l-a-b)} + \frac{l-a-b}{2} & 0\leq a+ \frac{p_2-p_1}{2c(l-a-b)} + \frac{l-a-b}{2} \leq l\\
	l & a+ \frac{p_2-p_1}{2c(l-a-b)} + \frac{l-a-b}{2}>l\\
	0 & a+ \frac{p_2-p_1}{2c(l-a-b)} + \frac{l-a-b}{2}<0
\end{cases}
\end{equation}

Setzt man die Gleichgewichtspreise aus Gleichungen \ref{eqn:aspremont-prices} in die Gleichgewichtsmengen aus Gleichung \ref{eqn:aspremont-quantities} ein, so kann man daraus den Gewinn im Gleichgewicht $\pi_1^*=p_1^* q_1(p_1^*, p_2^*)$ berechnen.
Der Gewinn im Gleichgewicht ist dann (Quelle: eigene Berechnung)
\begin{equation}
\pi_1^* = \frac{c}{2} (l-a-b) \left(l+ \frac{a-b}{2} \right)^2
\end{equation}

\subsection{Schlussfolgerungen}

\paragraph{Prinzip der maximalen Differenzierung}
Daraus lassen sich die Ableitungen der Gewinnfunktion im Gleichgewicht berechnen. Sie sind (Quelle: eigene Berechnung)
\begin{equation}
\frac{\partial \pi_1^*}{\partial a} = -\frac{c}{2}\left(l+\frac{a-b}{3}\right)^2 + \frac{c}{3} (l-a-b) \left( l+\frac{a-b}{3} \right)
\end{equation}
Dieser Wert ist strikt negativ, da der erste Term negativ und der zweite strikt positiv ist, der erste Term im Absolutwert jedoch stets dominiert. Der erste Term ist im Absolutwert größer als der zweite, weil $\frac{c}{2} > \frac{c}{3}$ und $\left(l+\frac{a-b}{3}\right) \geq (l-a-b)$ (Quelle: eigene Berechnung).

Die Ableitungen $\frac{\partial \pi_1^*}{\partial a}<0$ und $\frac{\partial \pi_2^*}{\partial b}<0$ sind also negativ \citep[S.\,1149]{aspremont1979}.
Somit bewegen sich die beiden Wettbewerber so weit wie möglich voneinander weg. Das ist das Prinzip der maximalen Differenzierung.

\section{Agentenbasierte Ansätze}\label{Sec-ABM}
Agentenbasierte Ansätze sind in der Lage eine größere Komplexität an Modellen abzubilden. In Kapitel \ref{sec:ABM-Komplexität} wird dargelegt, dass in der Politik viele verschiedene Aspekte zu berücksichtigen sind. Diese Aspekte lassen sich nicht alle in ein globales Modell (TODO check Begriff) integrieren und analytisch lösen. Daher werden agentenbasierte Ansätze verwendet.

\subsection{Komplexität von Politik} \label{sec:ABM-Komplexität}
Politik umfasst viele Aspekte. Die bisher vorgestellten Modelle bilden einen eindimensionalen räumlichen Preiswettbewerb ab. Politik ist real sehr viel komplexer. Im Folgenden werden einige Aspekte erläutert, die eine Rolle spielen können. Dazu gehören:
\begin{itemize}
	\item Wahlrecht
	\item Koalitionen
	\item Personen
	\item Dynamik
\end{itemize}

\paragraph{Wahlrecht}
Das Wahlrecht spielt eine wesentliche Rolle für die Strategien der Parteien, da es gewissermaßen die Spielregeln der politischen Machtverteilung definiert. Es wird im Wesentlichen zwischen Mehrheitswahlrecht und Verhältniswahlrecht unterschieden (TODO Zitat). Beim Mehrheitswahlrecht werden jeweils diejenigen Kandidaten gewählt, die in ihrem Wahlbezirk die meisten Stimmen bekommen. Aus parteipolitischer Perspektive degeneriert das politische System zu einem Zweiparteiensystem (TODO Zitat). Dagegen gilt in Deutschland hauptsächlich das Verhältniswahlrecht. Dabei spiegeln die Anteile der Parlamterier im Parlament die Wahlstimmenanteile wider. Dass Parteien auf diese Art und Weise beliebig klein werden können, wird in Deutschland durch eine 5\%-Hürde begrenzt. Das sorgt für eine völlig eigene Dynamik rund um die 5\%-Grenze.

\paragraph{Koalitionen}
Politische Macht wird in Demokratien hauptsächlich über Gesetzgebung und Regierung ausgeübt (TODO check und Zitat). Nur so können eigene Standpunkte und Ideen durchgesetzt werden. Da eine Partei in Deutschland selten die absolute Mehrheit erringen kann, müssen die Parteien Koalitionen eingehen. Beispielweise beschäftigen sich Laver und Shepsle (TODO Laver Shepsle 1990) mit möglichen Koalitionsmodellen.

Ich möchte hier die Grundidee der Koalitionsmodelle zitieren. [TODO]

Aufgrund der vielfältigen Kombinationsmöglichkeiten der Parteien werden diese Modelle sehr schnell sehr komplex, insbesondere bei mehr als drei Parteien. Das hier beschriebene Modell setzt außerdem voraus, dass die politischen Dimensionen, über die die Standpunkte der Parteien definiert sind, klar definiert und abgegrenzt sind. Das Modell, das hier später beschrieben wird vermischt jedoch politische Dimensionen und ist daher nicht geeignet um Koalitionen auf diese Art und Weise zu analysieren.

\paragraph{Personen}
Ein wesentlicher Faktor sind außerdem die Personen, die die Parteien jeweils vertreten, beispielsweise die Spitzenkandidaten bei Bundestagswahlen. 
[TODO: Beispiel von Wahl und Einfluss von Spitzenkandidaten]

\paragraph{Dynamik}
Zuletzt ist Politik keinesfalls statisch sondern dynamisch. Einfache Modelle betrachten Politik als rundenbasiert (TODO Zitat), zum Beispiel zwischen den Wahlen, wohingegen komplexere Modelle die Phasen zwischen den Wahlen abbilden (TODO Zitat)

Die Dynamik wird in Kapitel \ref{sec:ABM-Dynamik} genauer betrachtet.

\subsection{Einführung in Agentenbasierte Modelle}

Agentenbasierte Modelle zeichnen sich dadurch aus, dass die Modellbeschreibung sich auf einzelne Bestandteile, die Agenten, beziehen. Zusammen bilden die Agentenbeschreibung und ihre Interaktionen das Modell, anstatt eine globale Beschreibung des Modells zu haben. Marchi und Page \citep{marchi2014ABMs} besprechen eingehend agentenbasierende Modelle. Sie beschreiben die Vielfalt an Modellen und erklären ihre Anwendungsfelder.

Im Zentrum der Modelle stehen die Agentenmit ihren Eigenschaften und Verhaltensregeln. Die Anzahl der Agenten kann je nach Modell stark schwanken. Laut \citet{marchi2014ABMs} variiert die Anzahl der Agenten normalerweise zwischen 2 und 10000. Im Zuge der Covid-19 Studien gibt es agentenbasierte Modelle die bis zu ??? Agenten umfassen (TODO Zitat).

Die Agenten haben eine vorher festgelegte Anzahl an Eigenschaften. Diese können sich über die Zeit ändern. Das bedeutet zu jedem Zeitpunkt können die Eigenschaften der Agenten durch einen Vektor $a_j^t = (x_{j1}^t, x_{j2}^t,..., x_{jM}^t)$ beschrieben werden. Diese Eigenschaften bieten eine große Flexibilität des Modells, da sie fast alles codieren können. Im Kontext politischer Modelle spielen insbesondere die Eigenschaften von Wählern und Parteien eine Rolle. Das bedeutet beispielsweise, dass ideologische Positionen bedeutend sind.

Die Verhaltensregeln beschreiben wie sich die Eigenschaften der Agenten von einem Zeitpunkt zum nächsten verändern. In einfachen Modellen hängt das ausschließlich von den Eigenschaften des vorherigen Zeitpunkts ab. In komplizierteren Modellen kann dies auch von weiteren vorherigen Zeitpunkten abhängen. Diese Verhaltensregeln beschreiben insbesondere die Rationalität der Agenten. Agenten mit beschränkter Rationalität betrachten nur einen kleinen Ausschnitt der Informationen und ziehen mögliche Nebenwirkungen ihrer Entscheidungen nicht in Betracht. Dagegen nutzen Agenten mit vollkommener Rationalität alle verfügbaren Informationen und Berücksichtigen bei ihrer Entscheidung die Reaktionen anderer Akteure. Auf diesem Spektrum der Rationalität können Modelle berücksichtigen, dass Agenten lernfähig sein können. Zuletzt muss dass Modell genau festlegen in welcher Reihenfolge die Akteure agieren beziehungsweise wie ein gleichzeitiges Handeln der Agenten behandelt werden soll. Beispielsweise, können die Entscheidungen zweier Agenten A und B gegenseitig von der Position des anderen abhängen. Wenn zuerst Agent A agiert und dann Agent B die neue Position von A berücksichtig, so führt das zu einem anderen Ergebnis, als wenn zuerst Agent A agiert und dann Agent B auf die Position von A reagiert. Eine solche Reihenfolge kann zufällig, aufgrund von Eigenschaften, oder endogen im Modell entschieden werden[TODO Page 1977 zitieren?].

\paragraph{Beispiele agentenbasierter Modelle}

Im Folgenden sollen drei räumliche agentenbasierte politische Modelle genauer erläutert werden. Zunächst wird das Koalitionsmodell von \citet{laver1990coalitions} erläutert. Dagegen stellt das Modell von Kollmann, Miller und Page (TODO, cite 1992, 1998) als Zweiparteienmodell die Bewegungen der Parteien ins Zentrum. Das dritte Modell ist das Mehrparteienmodell von \citet{laver2005policy}. Dieses Modell wird separat in Kapitel (TODO 2.3.4 Dynamik von Parteien) beschrieben, da es teil der empirischen Untersuchung in Kapitel (TODO 3.4 Dynamik agentenbasiert modelliert) ist.

[TODO check Buch von Laver ob noch andere/sinnvollere Modelle]

\paragraph{Koalitionsmodell}
Die Idee für das Koalitionsmodell von \citet{laver1990coalitions} basiert darauf, dass eine Koalition in einer parlamentarischen Demokratie zugleich die Regierung stellt. Das bedeutet, dass aus den Reihen der Koalitionspartner die Exekutive, das bedeutet die Ministerämter besetzt werden. Die Autoren nehmen dabei an, dass eine Partei, die ein Ministeramt innehat, die volle Kontrolle über diesen Fachbereich hat und innerhalb dieses Fachbereichs die eigene Parteiposition ohne Rücksicht auf Koalitionspartner durchsetzt. Dies beschränkt die glaubwürdigen Koalitionen auf diskrete Kombinationen der Standpunkte der Parteien auf festgelegten Politikfeldern. Insbesondere schließt das Koalitionen aus, bei denen Kompromisse innerhalb eines Politikfelds geschlossen werden. \citep[S.\,873-5]{laver1990coalitions}

Eine Regierungskoalition stellt den status quo dar. Damit der status quo durch eine andere Koalition abgelöst werden kann müssen mehrere Voraussetzungen erfüllt sein. Erstens muss jeder einzelne Koalitionspartner in der neuen Koalition die vorgeschlagene Ministerialverteilung gegenüber der alten Ministerialverteilung präferieren. Zweitens unterstellen \citet{laver1990coalitions} den Parteien die Rationalität das komplexe Spiel von Gegenvorschlägen durchrechnen zu können. Dadurch kann eine Koalition nur abgelöst werden, wenn alle beteiligten Parteien diese Dynamiken gegenüber dem status quo präferieren. \citep[S.\,877-8]{laver1990coalitions}

Dieses Modell wird auf die politische Situation in Island 1967 und 1971 angewendet. Dabei werden zwei politische Dimensionen mit vier (1967) beziehungsweise fünf (1971) Parteien betrachtet \citep[S.\,882-5]{laver1990coalitions}. \citet{laver1990coalitions} bezeichnen ihr Modell berechtigterweise nicht als agentenbasiertes Modell. Dennoch eignet es sich hervorragend als agentenbasiertes Modell, da es die Dynamik von neuen Koalitionen zwischen Wahlen gut beschreiben kann und die Aktuere klare Verhaltensmuster haben. Dabei können verschiedene Szenarien bezüglich der Rationalität der Parteien durchgespielt werden. Beispielsweise können die Parteien wie im Modell als vollständig rational betrachtet werden. Jedoch kann alternativ auch angenommen werden, dass die Parteien jeglicher Koalition zustimmen, die sie dem status quo gegenüber präferieren.

\paragraph{Dynamisches Zweiparteienmodell}

[TODO Check: Hotelling kann als agentenbasiertes Modell betrachtet werden.]

\citet{kollman1992adaptive} entwickeln ein Modell mit zwei Parteien, das klären soll, ob es unter verschiedenen Annahmen ein Gleichgewicht gibt, zu dem die Parteien streben. In dem agentenbasierten Modell versuchen die Parteien wahlweise mit einem Wahlprogramm möglichst nah an ihrer Idealposition zu gewinnen oder nur zu gewinnen. Dabei wissen die Parteien nicht genau über die Wähler Bescheid, sondern können nur auf ihre Umfragewerte reagieren und ihre Position entsprechend anpassen. Die Präferenz der Wähler drückt sich in einer Optimalposition im mehrdimensionalen Positionsraum und eine Gewichtung der Politikfelder aus. 
\citep[S.\,930-1]{kollman1992adaptive}

Für jede Wahl beziehungsweise jeden Zeitschritt bleibt die Regierungspartei statisch und die Oppositionspartei versucht zu gewinnen. Dabei stellen die Autoren fest, dass bei zwei Parteien, die lediglich nach Wahlgewinn streben, der Herausforderer nach nur zwölf Wahlen nur noch eine Chance von 40 Prozent hat, die Regierung zurückzugewinnen. Dagegen wechselt die Regierung bei zwei ideologische Parteien hin und her und auch nach mehreren Wahlen bleiben die Chancen gut, dass die Oppositionspartei gewinnt. \citep[S.\,934-5]{kollman1992adaptive}

\citet{kollman1998political} untersuchen dieses Modell mit seinen adaptiven Parteien genauer hinsichtlich der Wählerverteilung. Nun gibt es Stellen mit hohen und geringen Wählerdichten. Dabei stellt sich heraus, dass die Parteien zwar immer noch moderate Positionen finden, die Wahlen gewinnen können, es ist jedoch deutlich erschwert. Das bedeutet, dass der Vorteil der Regierungspartei noch größer ist und dass je nach Unregelmäßigkeit der Wählerverteilung es länger braucht, bis ein Optimum gefunden wird. \citep{kollman1998political}

\paragraph{TODO verschiebe diesen Paragraphen}
\citet[S.\,13ff]{marchi2014ABMs}: Ein Modell sollte sowohl mathematisch gelöst als auch computersimuliert werden. Das Modell sollte dabei zunächst so einfach wie möglich gehalten werden. Erst wenn sich das Modell trotzdem als zu komplex darstellt sollte das Modell ausschließlich als agentenbasiertes Modell formuliert werden.




\subsection{Dynamik von Parteien} \label{sec:ABM-Dynamik}

\citet{laver2005policy}: kontinuierliche Evaluation der Wähler

Viele politische Modelle sind statisch hinsichtlich der Parteipositionen, wohingegen Laver die Parteipositionen als dynamisch modelliert. Dies stelle die Realität besser dar, in der die Parteien kontinuierlich ihre Positionen anpassen, wie schwankende Umfragewerte der Parteien belegen würden \citep[S.\,263-4]{laver2005policy}.

[TODO evtl Absatz über KMP]

Das Modell von \citet{laver2005policy} vereinigt Ansätze der klassischen räumlichen Modelle mit den dynamischen Ansätzen verschieder Parteitypen. Die Parteitypen haben gemeinsam, dass sie ihre Position auf die aktuelle politische Situation anpassen. Sie unterscheiden sich jedoch darin, wie sie diese Anpassung vornehmen. Laver unterscheidet dabei die vier Parteitypen Aggregator, Hunter, Predator und Sticker \citep[S.\,266-7]{laver2005policy}. In dieser Arbeit werden die original englischen Begriffe beibehalten. Im Folgenden werden die Parteitypen genauer definiert.

\paragraph{Aggregator} Der Aggregator, was im Deutschen die gleiche Bedeutung hat, sammelt die Meinungen seiner momentanen Wähler. Die neue Position ist die Durchschnittsposition dieser momentanen Wähler \citep[S.\,267]{laver2005policy}. Dies entspricht einer Partei die sich basisdemokratisch organisiert.

\paragraph{Hunter} Der Hunter, auf deutsch der Jäger, versucht möglichst viele Wähler hinzuzugewinnen. Dazu verwendet er eine beschränkt rationale Strategie die auf dem Erfolg der vorhergegangenen Positionsanpassungen basiert. Hat die vorherige Positionsanpassung Wähler hinzugewonnen, so wird eine weiterer Schritt in die selbe Richtung gegangen. Ansonsten wird ein Schritt in die umgekehrte Richtung gegangen, mit einer zufälligen Komponente von bis zu 90 Grad \citep[S.\,267]{laver2005policy}. Dieses Verhalten entspricht einer Partei, in der die Parteiführung völlig freie Hand hat bei der Positionswahl und danach strebt, möglichst viele Wähler zu gewinnen.

\paragraph{Predator} Der Predator, was soviel heißt wie Raubtier, passt seine Position nicht aufgrund der Wähler an, sondern aufgrund der Positionen der anderen Parteien und deren Zustimmungswerten. Ist der Predator selbst die größte Partei, so behält er seine Position. Ansonsten bewegt er sich in Richtung der Position der größten Partei. Dies entspricht einer Partei mit einer starken Führung, die versucht möglichst viele Wähler zu gewinnen. Im Unterschied zum Hunter nimmt sie nicht die Zustimmungsänderung als Indikator, sondern die Partei mit der meisten Zustimmung als Indikator, wo die meisten Wähler zu gewinnen sind \citep[S.\,267]{laver2005policy}.

\paragraph{Sticker} Der Sticker, was auf deutsch dem Festkleber entspricht, ändert seine Position nicht. Diese Regel kann einer ideologischen Partei entsprechen, die auf ihren Positionen festgefahren ist. Diese Partei ist an sich nicht interessant, kann jedoch spannende Wechselwirkungen mit anderen Parteitypen haben \citep[S.\,267]{laver2005policy}.

TODO Ergebnisse der Experimente von \citep{laver2005policy}

\citet{laver2007endogenousParties}: Parteitypen % TODO
