\chapter{Theoretische Analyse}\label{Kap-Theorie}

\noindent
Im Folgenden werden unterschiedliche Modelle des räumlichen Wettbewerbs vorgestellt. Jedes Modell wird zunächst in seiner ursprünglichen Form gezeigt. Anschließend wird analysiert inwiefern sich das Modell auf politische Gegebenheiten anwenden lässt und welche Vorhersagen sich daraus ergeben.

\section{Hotellings Gesetz}\label{Sec-Hotelling}

\subsection{Modell}\label{Sec-Hotelling-Modell}

Hotelling \citep{hotelling1929} untersucht anhand eines einfachen Modells wie sich bei einem Duopol der räumliche Wettbewerb gestaltet. Dazu nimmt er an, dass die zwei Wettbewerber auf einer räumlichen Linie ihre Position wählen können. Abbildung \ref{Fig-Linearer-Wettbewerb} zeigt mögliche Positionen der Wettbewerber. Dabei sind die Konsumenten gleichmäßig über die gesamte Linie verteilt.

\begin{figure}[htb]
	\centering
	\begin{tikzpicture}
	\coordinate (start) at (0,0) {};
	\coordinate (end) at (10,0) {};
	\coordinate (left) at (2,0) {};
	\coordinate (right) at (7,0) {};
	\draw (start) -- (end);
	\draw ($(start)+(0,5pt)$) -- ($(start)-(0,5pt)$);
	\draw ($(left)+(0,5pt)$) -- ($(left)-(0,5pt)$);
	\draw ($(right)+(0,5pt)$) -- ($(right)-(0,5pt)$);
	\draw ($(end)+(0,5pt)$) -- ($(end)-(0,5pt)$);
	\node at ($(start)+(0,10pt)$) {$0$};
	\node at ($(end)+(0,10pt)$) {$l$};
	\node at ($(left)+(0,10pt)$) {$A$};
	\node at ($(right)+(0,10pt)$) {$B$};
	
	\coordinate (startdown) at ($(start)+(0,-20pt)$) {};
	\coordinate (enddown) at ($(end)+(0,-20pt)$) {};
	\coordinate (leftdown) at ($(left)+(0,-20pt)$) {};
	\coordinate (rightdown) at ($(right)+(0,-20pt)$) {};
	
	\draw[|-|] (startdown) -- (leftdown) node[midway, below] {$a$};
	\draw[|-|] (rightdown) -- (enddown) node[midway, below] {$b$};
	\end{tikzpicture}
	\caption{Positionen der Wettbewerber $A$ und $B$ und Konsumenten auf einer Linie von $0$ bis $l$.}
	\label{Fig-Linearer-Wettbewerb}
\end{figure}

Eine grundlegende Annahme des Modells ist, dass die Konsumenten lineare Transportkosten in Höhe von $c$ pro Längeneinheit haben. Die Konsumenten kaufen somit bei demjenigen Duopolisten bei dem die Kosten bestehend aus Preis und Transportkosten am geringsten sind.

Nun nimmt Hotelling an, dass $A$ zu einem Preis $p_1$ und $B$ zu einem Preis $p_2$ verkauft. Die Verkaufsmengen sind jeweils $q_1$ und $q_2$.

Die erste wichtige Feststellung ist, dass das Duopol nur dann ein Duopol bleibt und nicht zu einem Monopol kollabiert, wenn der Preisunterschied nicht zu hoch ist. Genauer gesagt, darf der Preisaufschlag von $A$ nicht höher sein als die Transportkosten von Wettbewerber $B$ nach $A$:
\begin{equation}\label{eqn:Preisannahme}
\begin{split}
p_1 &\leq p_2 + c(l-a-b) \\
p_2 &\leq p_1 + c(l-a-b)
\end{split}
\end{equation}
So lange Bedingung \ref{eqn:Preisannahme} erfüllt ist, kaufen alle Konsumenten links bei Wettbewerber $A$ und alle rechts bei $B$. Die Konsumenten in der Mitte teilen sich in einen Abschnitt $x$, der bei $A$ kauft, und einen Abschnitt $y$, der bei $B$ kauft.

Daraus ergeben sich zwei Bedingungen. Erstens ergeben die Längen zusammen die gesamte Länge und zweitens soll der Konsument der die beiden Abschnitte $x$ und $y$ trennt indifferent zwischen den beiden Wettbewerbern sein. Dies ist in Gleichung \ref{eqn:Gleichgewichtsbedingungen} formuliert.
\begin{equation}\label{eqn:Gleichgewichtsbedingungen}
\begin{split}
l &= a + x + y + b \\
p_1 + cx &= p_2 + cy
\end{split}
\end{equation}

Löst man dieses lineare Gleichungssystem nach $x$ und $y$ auf ergibt sich
\begin{equation}
\begin{split}
x &= \tfrac{1}{2} \left(l-a-b+\frac{p_2-p_1}{c}\right)\\
y &= \tfrac{1}{2} \left(l-a-b+\frac{p_1-p_2}{c}\right)
\end{split}
\end{equation}

Ohne Beschränkung der Allgemeinheit werden die Kosten der Wettbewerber Null gesetzt und der Gewinn ist somit
\begin{equation}
\begin{split}
\pi_1 &= p_1 q_1 = p_1 (a+x) = \tfrac{1}{2} p_1 \left(l+a-b\right) -\frac{p_1^2}{c}+\frac{p_1 p_2}{c}\\
\pi_2 &= p_2 q_2 = p_2 (b+y) = \tfrac{1}{2} p_1 \left(l-a+b\right) -\frac{p_2^2}{c}+\frac{p_1 p_2}{c}
\end{split}
\end{equation}

Der gewinnmaximierende Preis, berechnet mithilfe der Gleichungen $\frac{\partial \pi_1}{\partial p_1}=0$ und $\frac{\partial \pi_2}{\partial p_2}=0$, ist
\begin{equation}
\begin{split}
p_1 &= \left(l+\frac{a-b}{c}\right) \\
p_2 &= \left(l-\frac{a-b}{c}\right)
\end{split}
\end{equation}
mit den verkauften Mengen
\begin{equation}
\begin{split}
q_1 &= a + x = \tfrac{1}{2} \left(l + \frac{a-b}{3}\right) \\
q_2 &= b + y = \tfrac{1}{2} \left(l - \frac{a-b}{3}\right)
.\end{split}
\end{equation}

Die resultierende Gewinnfunktion der Wettbewerber ist
\begin{equation}\label{eqn:Gewinnfunktion}
\begin{split}
\pi_1 &= \frac{c}{2} \left(l+\frac{a-b}{3}\right) \\
\pi_2 &= \frac{c}{2} \left(l-\frac{a-b}{3}\right)
.\end{split}
\end{equation}

\subsection{Schlussfolgerungen}

\subsection{Hypothesen}

\section{Modell von d‘Aspremont}\label{Sec-Aspremont}

d'Aspremont \citep{aspremont1979} kritisiert das Principle of Minimum Differentiation (TODO: kontrollieren, dass definiert) von Hotelling.
Die Kritik stützt sich auf zwei Aspekte: Erstens ist kein Gleichgewichtspreis definiert, wenn die beiden Wettbewerber sich am selben Punkt befinden. Zweitens hat das Modell drastisch andere Lösungen, wenn die Transportkostenfunktion sich ändert.

\subsection{Kritik an Hotelling}
Der Gleichgewichtspreis ergibt sich als Nash-Cournot-Gleichgewicht. Dabei müssen zwei Fälle unterschieden werden: Erstens wenn sich die Wettbewerber am selben Punkt befinden und zweitens wenn die Wettbewerber auseinanderliegen.

Im Fall, dass die Wettbewerber sich beide am selben Punkt (beispielsweise in der Mitte) befinden, so herrscht reiner Preiswettbewerb. Nach Betrand (TODO: zitieren) führt das zu Preisen in Höhe der Stückkosten, in diesem Fall zu einem Preis $p_1=p_2=0$.

Liegen die Wettbewerber auseinander, so stimmt \cite{aspremont1979} mit \cite{hotelling1929} überein und präzisiert das Marktgleichgewicht. Der Gleichgewichtspreis ist dann
\begin{equation}
\begin{split}
p_1 &= c \left(l+\frac{a-b}{3}\right) \\
p_2 &= c \left(l-\frac{a-b}{3}\right)
\end{split}
\end{equation}
unter der Bedingung
\begin{equation}
\begin{split}
\left(l+\frac{a-b}{3}\right)^2 &\geq \tfrac{4}{3} l (a+2b) \\
\left(l-\frac{a-b}{3}\right)^2 &\geq \tfrac{4}{3} l (2a+b)
\end{split}
\end{equation}

\paragraph{Das Gleichgewicht tendiert zur Mitte und kollabiert.} Da $\frac{\partial \pi_1}{\partial a}>0$ und $\frac{\partial \pi_2}{\partial b}>0$ bewegen sich beide Wettbewerber zur Gewinnoptimierung zur Mitte hin. Allerdings existiert in der Mitte nur noch das Gleichgewicht mit Nullpreisen. Da dies kein wirkliches Marktgleichgewicht ist, ist das laut \cite{aspremont1979} das gesamte Modell invalidiert.

\subsection{Modell}\label{Sec-Aspremont-Modell}
\cite{aspremont1979} schlägt stattdessen ein korrigiertes Modell vor mit einer veränderten Transportkostenfunktion vor. Diese ist dann quadratisch, also $cx^2$ statt $cx$.
Das Gleichgewicht berechnet sich identisch zum Fall oben. Die Gleichgewichtspreise sind dann
\begin{equation}
\begin{split}
p_1 &= c (l-a-b) \left(l+\frac{a-b}{b}\right) \\
p_2 &= c (l-a-b) \left(l-\frac{a-b}{b}\right)
\end{split}
\end{equation}

Die Gewinnfunktionen sind dann
\begin{equation}
TODO: Gewinnfunktionen
\end{equation}

Die Ableitungen $\frac{\partial \pi_1}{\partial a}<0$ und $\frac{\partial \pi_2}{\partial b}<0$ sind also negativ. Somit bewegen sich die beiden Wettbewerber so weit wie möglich voneinander weg. Das ist das Prinzip der maximalen Differenzierung.

\subsection{Schlussfolgerungen}

\subsection{Hypothesen}

\section{Agentenbasierte Ansätze}\label{Sec-ABM}

\section{Situation in Deutschland}\label{Sec-Deutschland}

\section{Vorhersagen aus Modellen}\label{Sec-Vorhersagen}
