\chapter{Theoretische Analyse}\label{Kap-Theorie}

\noindent
Text...

\section{Hotellings Gesetz}\label{Sec-Hotelling}

Hotelling \citep{hotelling1929} untersucht anhand eines einfachen Modells wie sich bei einem Duopol der räumliche Wettbewerb gestaltet. Dazu nimmt er an, dass die zwei Wettbewerber auf einer räumlichen Linie ihre Position wählen können. Abbildung \ref{Fig-Linearer-Wettbewerb} zeigt mögliche Positionen der Wettbewerber. Dabei sind die Konsumenten gleichmäßig über die gesamte Linie verteilt.

\begin{figure}[htb]
	\centering
	\begin{tikzpicture}
	\coordinate (start) at (0,0) {};
	\coordinate (end) at (10,0) {};
	\coordinate (left) at (2,0) {};
	\coordinate (right) at (7,0) {};
	\draw (start) -- (end);
	\draw ($(start)+(0,5pt)$) -- ($(start)-(0,5pt)$);
	\draw ($(left)+(0,5pt)$) -- ($(left)-(0,5pt)$);
	\draw ($(right)+(0,5pt)$) -- ($(right)-(0,5pt)$);
	\draw ($(end)+(0,5pt)$) -- ($(end)-(0,5pt)$);
	\node at ($(start)+(0,10pt)$) {$0$};
	\node at ($(end)+(0,10pt)$) {$l$};
	\node at ($(left)+(0,10pt)$) {$A$};
	\node at ($(right)+(0,10pt)$) {$B$};
	\end{tikzpicture}
	\caption{Positionen der Wettbewerber $A$ und $B$ und Konsumenten auf einer Linie von $0$ bis $l$.}
	\label{Fig-Linearer-Wettbewerb}
\end{figure}

Die Konsumenten kaufen bei demjenigen Duopolisten, der ihnen den größten Nutzen gibt. Die Nutzenfunktion der Konsumenten ist dabei
\begin{equation}
U_{Hotelling} = 
\end{equation}

\section{Modell von d‘Aspremont}\label{Sec-Aspremont}

\section{Agentenbasierte Ansätze}\label{Sec-ABM}

\section{Situation in Deutschland}\label{Sec-Deutschland}

\section{Vorhersagen aus Modellen}\label{Sec-Vorhersagen}
