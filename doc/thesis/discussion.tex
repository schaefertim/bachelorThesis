%%%%%%%%%%%%%%%%%%%%%%%%%%%%%%%%%%%%%%%%%%%%%%%%%%%%%%%%%%%%%%%%%%%%
% Diskussion und Ausblick
%%%%%%%%%%%%%%%%%%%%%%%%%%%%%%%%%%%%%%%%%%%%%%%%%%%%%%%%%%%%%%%%%%%%

\chapter{Diskussion}\label{Kap-Diskussion}

\paragraph{Zusammenfassung}
In Kapitel \ref{Kap-Theorie} wurden mehrere politische Modelle vorgestellt. Das Modell von \citet{laver2005policy} wurde dann auf die politische Situation in Deutschland angewendet. Die Wahl-O-Mat-Daten der Parteien bildeten die Grundlage für die zweidimensionalen Positionen der Parteien und Wähler. Um auf die 38 Positionen der Parteien auf zwei Dimensionen zu reduzieren wurde eine Hauptkomponentenanalyse angewendet. Auf Grundlage der projizierten Parteipositionen und den Umfragedaten des Politbarometers wurden die Wähler positioniert. Dabei ergab sich ein konsistentes Bild mit den Parteipositionen. Abschließend wurden diese Partei- und Wählerdaten in das Modell von \citet{laver2005policy} eingespeist und als intitiale Position verwendet. Die Ergebnisse über die Interaktionen der Parteitypen ließen sich im Großen und Ganzen bestätigen und reproduzieren. Allerdings sind durch die unregelmäßige Verteilung der Wähler, die Hunter erfolgreicher als bei \citet{laver2005policy}. Beim Vergleich der verschiedenen Parteiinteraktionen stellt sich heraus, dass ein Modell mit ausschließlich Aggregatoren am nächsten an den realen Parteipositionen liegt. Dies kann den trivialen Grund haben, dass durch die Modellkonstruktion dort die höchsten Wählerdichten zu finden sind. Andererseits könnte es auch daran liegen, dass dies die reale parteipolitische Situation am besten abbildet.

\paragraph{Evaluierung der Modelle}

\paragraph{Datengrundlage}
Ein wesentliches Element dieser Arbeit ist die Verwendung von Daten. Dabei stützt sich die Arbeit zur Festlegung der politischen Positionen auf die Angaben der Parteiem beim Wahl-O-Maten. Dies steht im Kontrast zum Vorgehen von \citet{schofield1998germany}, die die Wähler als Grundlage nehmen und daraufhin die Parteipositionen bestimmen. Da ultimativ die Wähler bestimmen, welche Partei sie wählen, ist es der bessere Ansatz, den Positionsraum über die Wähler festzulegen. Leider war das in dieser Arbeit nicht möglich, weil die Politbarometer-Daten keine einfache Projektion auf politische Dimensionen zulassen. Die perfekte Lösung wäre ein Datensatz, bei dem sowohl Partei- als auch Wählerpositionen in den selben Dimensionen erfasst werden. Dies wäre mit dem Wahl-O-Mat möglich, da dort ohnehin die Parteipositionen erfasst werden. Dabei wird der Wahl-O-Mat von Millionen Menschen genutzt. Diese Daten werden momentan aus Datenschutzgründen nicht gespeichert. Durch eine entsprechende Abfrage des Einverständnis der Nutzer sollte eine Speicherung und wissenschaftliche Nutzung der Daten jedoch möglich sein.

\paragraph{Zukünftige Untersuchungen}
Die vorgestellten Modelle und Ansätze bieten eine gute Grundlage für weitere Forschung. Der Ansatz, die Positionen der Parteien mit einer Hauptkomponentenanalyse zu reduzieren, bietet noch andere Möglichkeiten. So kann ganz grundsätzlich untersucht werden, ob sich das Muster wiederholt, dass Parteien, die für Umweltschutz eintreten, gleichzeitig für soziale Umverteilung eintreten. Dazu können andere Wahljahre in Deutschland herangezogen werden. Es können jedoch auch andere Datengrundlagen oder andere Länder betrachtet werden.
Ist einmal ein hochwertiges Verfahren zur Festlegung der Partei- und Wählerpositionen etabliert, so tun sich auch noch mehr Möglichkeiten hinsichtlich der dynamischen Modelle auf. In dieser Arbeit wurden die Modelle nur qualitativ untersucht. Diese Ideen können in nachfolgenden Arbeiten rigoroser und quantitativ betrachtet werden.
Außerdem können alternative und konkurrierende Modelle verwendet werden. Diese Modelle können dann auf ihre reale Vorhersagekraft überprüft und miteinander verglichen werden. So können einerseits sehr einfache Modelle wichtige Aspekte in politischen Modellen ausleuchten.
Andererseits ist Politik ein sehr komplexer Prozess, der viele Akteure umfasst. Daher ist es durchaus auch berechtigt, wenn Modelle auch groß und komplex werden. Allerdings muss diese Komplexität mit großer Vorhersagekraft gerechtfertigt werden.

\cleardoublepage