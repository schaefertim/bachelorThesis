%%%%%%%%%%%%%%%%%%%%%%%%%%%%%%%%%%%%%%%%%%%%%%%%%%%%%%%%%%%%%%%%%%%%
% Diskussion und Ausblick
%%%%%%%%%%%%%%%%%%%%%%%%%%%%%%%%%%%%%%%%%%%%%%%%%%%%%%%%%%%%%%%%%%%%

\chapter{Diskussion}\label{Kap-Diskussion}

\paragraph{Zusammenfassung}
In Kapitel \ref{Kap-Theorie} wurden mehrere politische Modelle vorgestellt. Das Modell von \citet{laver2005policy} wurde dann auf die politische Situation in Deutschland angewendet. Die Wahl-O-Mat-Daten der Parteien bildeten die Grundlage für die zweidimensionalen Positionen der Parteien und Wähler. Um auf die 38 Positionen der Parteien auf zwei Dimensionen zu reduzieren wurde eine Hauptkomponentenanalyse angewendet. Auf Grundlage der projizierten Parteipositionen und den Umfragedaten des Politbarometers wurden die Wähler positioniert. Dabei ergab sich ein konsistentes Bild mit den Parteipositionen. Abschließend wurden diese Partei- und Wählerdaten in das Modell von \citet{laver2005policy} eingespeist und als intitiale Position verwendet. Die Ergebnisse über die Interaktionen der Parteitypen ließen sich im Großen und Ganzen bestätigen und reproduzieren. Allerdings sind durch die unregelmäßige Verteilung der Wähler, die Hunter erfolgreicher als bei \citet{laver2005policy}. Beim Vergleich der verschiedenen Parteiinteraktionen stellt sich heraus, dass ein Modell mit ausschließlich Aggregatoren am nächsten an den realen Parteipositionen liegt. Dies kann den trivialen Grund haben, dass durch die Modellkonstruktion dort die höchsten Wählerdichten zu finden sind. Andererseits könnte es auch daran liegen, dass dies die reale parteipolitische Situation am besten abbildet.

\paragraph{Datengrundlage}
Ein wesentliches Element dieser Arbeit ist die Verwendung von Daten. Dabei stützt sich die Arbeit zur Festlegung der politischen Positionen auf die Angaben der Parteiem beim Wahl-O-Maten. Dies steht im Kontrast zum Vorgehen von \citet{schofield1998germany}, die die Wähler als Grundlage nehmen und daraufhin die Parteipositionen bestimmen. Da ultimativ die Wähler bestimmen, welche Partei sie wählen, ist es der bessere Ansatz, den Positionsraum über die Wähler festzulegen. Leider war das in dieser Arbeit nicht möglich, weil die Politbarometer-Daten keine einfache Projektion auf politische Dimensionen zulassen. Die perfekte Lösung wäre ein Datensatz, bei dem sowohl Partei- als auch Wählerpositionen in den selben Dimensionen erfasst werden. Dies wäre mit dem Wahl-O-Mat möglich, da dort ohnehin die Parteipositionen erfasst werden. Dabei wird der Wahl-O-Mat von Millionen Menschen genutzt. Diese Daten werden momentan aus Datenschutzgründen nicht gespeichert. Durch eine entsprechende Abfrage des Einverständnis der Nutzer sollte eine Speicherung und wissenschaftliche Nutzung der Daten jedoch möglich sein.

\paragraph{Hotelling versus d'Aspremont}
Die Vorhersagen von \citet{hotelling1929} und \citet{aspremont1979} sind genau gegensätzlich. Die Arbeit hat gezeigt, dass das Modell von \citet{hotelling1929} auf Deutschland nicht zutrifft. Hier liegen die Parteien wie in Kapitel \ref{Sec-Parteienpositionen} gezeigt deutlich auseinander. Das Modell von \citet{aspremont1979} lässt sich nicht eindeutig bewerten, da unklar ist, wo die Grenzen der politischen Dimensionen liegen. Daher lässt sich nicht einschätzen, ob die Extrempunkte erreicht sind. Angesichts der allgemeinen Koalitionsfähigkeit der Parteien miteinander lässt sich jedoch annehmen, dass die Parteien keine Extrempositionen einnehmen. Somit beschreiben diese Modelle die tatsächlichen Gegebenheiten nicht vollständig. Daher bedarf es anderer beziehungsweise komplexerer Modelle.

\paragraph{Bewertung Laver}

Die Ergebnisse aus Kapitel \ref{Sec:Dynamik-Anwendung} stimmen im Großen und Ganzen mit \citet{laver2005policy} überein. Es ergeben sich jedoch Unterschiede aus der anderen Verteilung der Wähler. Insbesondere die unsymmetrische Verteilung sorgt dafür, dass manche Gleichgewichte nicht zustande kommen und die Hunter und Predatoren erfolgreicher sind.
Die meisten Konstellationen an Parteitypen unterscheiden sich nach 1000 Zyklen wesentlich von der Ausgangssituation. Eine Ausnahme davon bildet die Situation mit ausschließlich Aggregatoren. In dieser Situation bleiben die Parteien relativ stabil in der Nähe ihrer Ausgangsposition. Daher lässt sich schließen, dass im Rahmen dieses Modells der Parteityp des Aggregators als beste Beschreibung für die Parteien in Deutschland dient. Dieser Parteityp zeichnet sich dadurch aus, dass er als Position die durchschnittliche Position seiner Wähler annimmt, also demokratisch organisiert ist.

\paragraph{Forschungsfrage}
Die Arbeit sollte klären, ob Parteien ihre Standpunkte unter strategischen Gesichtspunkten auswählen. Im Rahmen der hier gezeigten empirischen Forschung mithilfe des Modells von \citet{laver2005policy} lässt sich sagen, dass die Parteien in Deutschland ihre Position nicht strategisch bestimmen, sondern demokratisch von ihren Wählern bestimmen lassen. Allerdings zeigen die vorgestellten theoretischen Arbeiten, dass strategische Verhaltensweisen der Parteien einen großen Raum in der Forschung einnehmen. Dies kann zwei Gründe haben.
Einerseits ist denkbar, dass viele theoretische Arbeiten Aspekte beleuchten, die in der Praxis keine Rolle spielen und Parteien nicht strategisch handeln.
Andererseits ist denkbar, dass andere empirische Arbeiten, wie beispielsweise \citet{schofield1998germany} durchaus zu dem Schluss kommen können, dass strategische Überlegungen eine Rolle spielen. Dabei sind jedoch komplexere Prozesse wie beispielsweise die Koalitionsbildung zu berücksichtigen sind.

\paragraph{Zukünftige Untersuchungen}
Die vorgestellten Modelle und Ansätze bieten eine gute Grundlage für weitere Forschung. Der Ansatz, die Positionen der Parteien mit einer Hauptkomponentenanalyse zu reduzieren, bietet noch andere Möglichkeiten. So kann ganz grundsätzlich untersucht werden, ob sich das Muster wiederholt, dass Parteien, die für Umweltschutz eintreten, gleichzeitig für soziale Umverteilung eintreten. Dazu können andere Wahljahre in Deutschland herangezogen werden. Es können außerdem auch andere Datengrundlagen oder andere Länder betrachtet werden.
Ist einmal ein hochwertiges Verfahren zur Festlegung der Partei- und Wählerpositionen etabliert, so tun sich noch mehr Möglichkeiten hinsichtlich der dynamischen Modelle auf. In dieser Arbeit wurde ein einfaches Modell ausschließlich qualitativ untersucht. In zukünftigen Arbeiten können weitere Modelle betrachtet werden oder untersuchte Modelle rigoroser und quantitativ untersucht werden.
Insbesondere können alternative und konkurrierende Modelle gegeneinander getestet werden. Dabei können die Modelle auf ihre reale Vorhersagekraft überprüft und miteinander verglichen werden.
Dabei ist stets zu bednken, dass kleine Modelle wichtige Aspekte in politischen Modellen ausleuchten.
Andererseits ist Politik ein sehr komplexer Prozess, der viele Akteure umfasst. Daher ist es durchaus auch berechtigt, wenn Modelle groß und komplex werden. Allerdings muss diese Komplexität mit großer Vorhersagekraft gerechtfertigt werden.

\cleardoublepage