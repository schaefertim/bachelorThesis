%%%%%%%%%%%%%%%%%%%%%%%%%%%%%%%%%%%%%%%%%%%%%%%%%%%%%%%%%%%%%%%%%%%%
% Einleitung
%%%%%%%%%%%%%%%%%%%%%%%%%%%%%%%%%%%%%%%%%%%%%%%%%%%%%%%%%%%%%%%%%%%%

\chapter{Einleitung}\label{Kap-Einleitung}
Politische Partien müssen zu einer Vielfalt von Themen ihre Standpunkte bestimmen. Abhängig von ihren Standpunkten können sie mehr oder weniger Wähler und Wählerinnen überzeugen. Daher ist es naheliegend, dass eine rational agierende Partei ihre Standpunkte unter strategischen Gesichtspunkten auswählt. Andererseits sind Parteien auch idelogische Konstrukte, die ihre Standpunkte aus Überzeugung auswählen. Diese Arbeit soll die Hypothese überprüfen, dass politische Parteien ihre Standpunkte unter strategischen Gesichtspunkten wählen. Dabei zeigen bisherige Arbeiten, dass Parteien unter strategischen Gesichtspunkte ihre Wählerpotentiale nicht ausschöpfen \citep{schofield1998germany}.

Im ersten Teil der Arbeit sollen theoretische Modelle vorgestellt werden. Dazu werden  zunächst einfache Modelle des räumlichen Wettbewerbs betrachtet. Hotellings Gesetz besagt, dass Wettbewerber minimale Differenzierung betreiben \citep{hotelling1929}.

Aspremont et al. \citep{aspremont1979} kritisieren dieses Modell. Sie nehmen an, dass die Transportkosten anstatt linear quadratisch im Abstand sind. Das führt dazu, dass Wettbewerber zur Gewinnmaximierung zu maximaler Differenzierung neigen, also genau das Gegenteil von Hotellings Gesetz. Auch im politischen Kontext ist es denkbar, dass die Nutzenfunktion quadratisch im Abstand der politischen Position ist.

Moderne Ansätze sind häufig agentenbasiert. Diese Ansätze sind in der Regel dynamisch, das heißt es finden mehrere Perioden statt. Marchi und Page \citep{marchi2014ABMs} geben einen Übrblick über solche agentenbasierten Modell. Dabei gibt es sowohl Modelle die die Zahl der Parteien exogen festlegen, als auch Modelle, die die Zahl der Parteien flexibel halten \citep{laver2007endogenousParties}.

Abschließend wird das Wahlsystem in Deutschland vorgestellt. Außerdem sollen aus den Theorien konkrete überprüfbare Vorhersagen für deutsche Parteien und Wähler abgeleitet werden.

Im zweiten Teil der Arbeit sollen die abgeleiteten Hypothesen anhand von Daten getestet werden. 

Um Aussagen über die Verteilung der Parteienpositionen zu testen, werden die Wahl-O-Mat-Daten (Bundeszentrale für politische Bildung, 2021) verwendet. Diese Daten umfassen für alle Parteien eine Aussage (Ablehnung, neutral, Zustimmung) für eine Vielzahl von Positionen.

Um Aussagen über Wählerpositionen zu testen, werden die Daten des Politbarometers (Forschungsgruppe Wahlen, Mannheim, 2021) verwendet. Dies sind Wählerdaten, die generelle politische Präferenzen abfragen und einzelne zusätzliche Positionen erfragen.

Als dritte Überprüfung werden Daten über Wählerwanderung berücksichtigt. Außerdem werden Daten über die Meinung zu Spitzenpolitiker erfragt. Dies erlaubt Rückschlüsse darauf, ob die viele Wähler zwischen mehreren Parteien unentschlossen sind. Das kann entweder bedeuten, dass die Parteipositionen sehr nah beieinander liegen oder es kann bedeuten, dass die Nutzenfunktion der Wähler sehr unempfindlich gegenüber den Parteipositionen ist.

In einer abschließenden Diskussion soll bewertet werden, inwiefern welche Theorie einer praktischen Überprüfung standhält und welche theoretischen Grundlagen eine geeignete Beschreibung der politischen Realität liefern.
