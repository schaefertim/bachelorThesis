%%%%%%%%%%%%%%%%%%%%%%%%%%%%%%%%%%%%%%%%%%%%%%%%%%%%%%%%%%%%%%%%%%%%
% Einleitung
%%%%%%%%%%%%%%%%%%%%%%%%%%%%%%%%%%%%%%%%%%%%%%%%%%%%%%%%%%%%%%%%%%%%

\chapter{Einleitung}\label{Kap-Einleitung}
Politische Partien müssen zu einer Vielfalt von Themen ihre Standpunkte bestimmen. Abhängig von ihren Standpunkten können sie mehr oder weniger Wähler und Wählerinnen überzeugen. Daher ist es naheliegend, dass eine rational agierende Partei ihre Standpunkte unter strategischen Gesichtspunkten auswählt. Andererseits sind Parteien auch idelogische Konstrukte, die ihre Standpunkte aus Überzeugung auswählen. Diese Arbeit soll die Hypothese überprüfen, dass politische Parteien ihre Standpunkte unter strategischen Gesichtspunkten wählen. Dabei zeigen bisherige Arbeiten, dass Parteien unter strategischen Gesichtspunkte ihre Wählerpotentiale nicht ausschöpfen \citep{schofield1998germany}.

Im ersten Teil der Arbeit sollen theoretische Modelle vorgestellt werden. Dazu werden  zunächst einfache Modelle des räumlichen Wettbewerbs betrachtet. Hotellings Gesetz besagt, dass Wettbewerber minimale Differenzierung betreiben \citep{hotelling1929}.

Aspremont et al. \citep{aspremont1979} kritisieren dieses Modell. Sie nehmen an, dass die Transportkosten anstatt linear quadratisch im Abstand sind. Das führt dazu, dass Wettbewerber zur Gewinnmaximierung zu maximaler Differenzierung neigen, also genau das Gegenteil von Hotellings Gesetz. Auch im politischen Kontext ist es denkbar, dass die Kostenfunktion quadratisch im Abstand der politischen Position ist.

Als dritte Perspektive auf politischen Parteienwettbewerb sollen moderne Ansätze betrachtet werden, welche häufig agentenbasiert sind. Diese Ansätze sind in der Regel dynamisch, das heißt es finden mehrere Perioden statt. Marchi und Page \citep{marchi2014ABMs} geben einen Überblick über solche agentenbasierten Modell. Dabei gibt es sowohl Modelle die die Zahl der Parteien exogen festlegen, als auch Modelle, die die Zahl der Parteien flexibel halten \citep{laver2007endogenousParties}.

Im zweiten Teil der Arbeit sollen die abgeleiteten Hypothesen anhand von Daten getestet werden. Dazu wird die Bundestagswahl 2017 ausgewählt und das Modell von \citet{laver2005policy} darauf angewendet.

Um Aussagen über die Verteilung der Parteienpositionen zu testen, werden die Wahl-O-Mat-Daten \citep{WahlOMat} verwendet. Diese Daten umfassen für alle Parteien eine Aussage (Ablehnung, neutral, Zustimmung) für eine Vielzahl von Positionen. Um daraus einen zweidimensionalen Positionsraum der Parteien zu erhalten, wird eine Hauptkomponentenanalyse angewandt. Die Hauptkomponenten werden außerdem auf ihre politische Entsprechung hin untersucht.

Um Aussagen über Wählerpositionen zu testen, werden die Daten des Politbarometers \citep{politbarometer} verwendet. Dies sind Wählerdaten, die generelle politische Präferenzen abfragen und einzelne zusätzliche Positionen erfragen. Mithilfe dieser Parteipräferenzen werden die Wähler im Positionsraum der Parteien positioniert.

Auf Grundlage dieser Partei- und Wählerpositionen wird das Modell von \citet{laver2005policy} angewandt. Dabei wird untersucht, ob die Ergebnisse auf Grundlage der Daten der Bundestagswahl 2017 mit den Ergebnissen in \citet{laver2005policy} mit Modelldaten übereinstimmen.

In einer abschließenden Diskussion wird bewertet, ob die vorgestellten Modelle und Daten eher für ein strategisches Handeln der Parteien oder dagegen sprechen. Außerdem wird bewertet wie realitätsnah die verwendeten Modelle und Methoden sind. Zusätzlich wird diskutiert wie die erarbeiteten Modelle und Ergebnisse in zukünftiger Forschung verwendet werden können.

\paragraph{Programmcode}
Ein wesentlicher Teil dieser Arbeit beinhaltet Programmcode. Dieser ist mit ??? Zeilen [TODO Zahl] relativ umfangreich. Dieser Programmcode ist in der digitalen Abgabe enthalten und wird außerdem auf Anfrage zur Verfügung gestellt. Diese Art der Bereitstellung erleichtert außerdem die weitere Verwendung dieser Codebasis.
