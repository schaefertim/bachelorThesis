%%%%%%%%%%%%%%%%%%%%%%%%%%%%%%%%%%%%%%%%%%%%%%%%%%%%%%%%%%%%%%%%%%%%
% Ergebnisse
%%%%%%%%%%%%%%%%%%%%%%%%%%%%%%%%%%%%%%%%%%%%%%%%%%%%%%%%%%%%%%%%%%%%

\chapter{Empirische Analyse}\label{Kap-Empirische-Analyse}

\section{Beispiel von Schofield et al}

\blindtext[3]

\section{Positionen der Parteien}\label{Sec-Parteienpositionen}

[TODO grundsätzliche Erläuterungen warum Parteipositionen notwendig sind]

\subsection{Wahl-O-Mat Daten}
Wahl-O-Mat \citep{WahlOMat} ist ein Tool, das von der Bundeszentrale für politische Bildung zur Verfügung gestellt wird. Der Wahl-O-Mat wird wissenschaftlich begleitet von Prof.\,Dr.\,Stefan Marschall \citep{MarschallWahlOMat}.

Das Ziel des Wahl-O-Maten bestehe darin, über \glqq wesentliche und unterscheidbare Inhalte der Parteien\grqq zu informieren. Außerdem soll so das politische Interesse insbesondere vor den Wahlen aber auch nach den Wahlen geweckt werden. 
Die Thesen werden von von Schülern, Auszubildenden und Studenten entwickelt unter Begleitung des Wahl-O-Mat Teams. Deren ungefähr 80 Thesen werden den Parteien zur Beantwortung vorgelegt. Die Parteien geben an, ob sie zustimmen, nicht zustimmen oder neutral sind. Unter dem Kriterium der Unterscheidbarkeit der Parteien werden  38 Thesen ausgewählt und bilden den Wahl-O-Mat. %[TODO cite https://www.sozwiss.hhu.de/institut/abteilungen/politikwissenschaft/politik-ii/prof-dr-stefan-marschall/forschungsprojekte/wahl-o-mat-forschung/was-ist-der-wahl-o-mat]

[TODO wissenschaftliche Publikationen]

Die Daten umfassen ausdrücklich keine Daten von Wählern. Eine Anfrage bei Prof.\,Dr.\,Stefan Marschall hat ergeben, dass \glqq die Logfiles direkt nach dem Wahl-O-Mat-Einsatz aus Datenschutzgründen vernichtet [werden]. Sie stehen auch uns [wissenschaftliche Begleitung des Wahl-O-Mats] nicht zur Auswertung zur Verfügung.\grqq

Für diese Arbeit greife ich auf die GitHub-Datenbank Qual-O-Mat von Felix Bolte \citep{Bolte2022QualOMat} zurück. Diese Datenbank sammelt alle verfügbaren Wahl-O-Mat Daten mit den Antworten der Parteien und legt diese in einem strukturierten Format ab. Somit kann auf die Daten leichter zugegriffen werden.

\subsection{Hauptkomponentenanalyse Wahl-O-Mat}
\paragraph{Hauptkomponentenanalyse}
Die Wahl-O-Mat Daten stellen die Positionen der Parteien im Hinblick auf die Thesen dar. Angesichts dessen, dass 38 Thesen abgefragt werden ist der Raum der Positionen hochdimensional. Daher nehme ich eine Dimensionsreduktion vor. Mein Mittel der Wahl ist eine Hauptkomponentenanalyse vorzunehmen.

Die Hauptkomponentenanalyse hat nach Bishop und Nasrabadi \citep{bishop2006pattern} zwei verschiedene Motivationen. Die erste Motivation ist es, die Hauptkomponenten so zu wählen, dass die Varianz entlang der Hauptkomponenten maximiert wird. Das entspricht im vorliegenden Fall, dem Ziel, dass sich die Positionen der Parteien entlang der Hauptkomponenten maximal unterscheiden sollen. Die zweite Motivation ist es, die Länge der Projektion zu minimieren. Im Fall der Parteipositionen bedeutet das, dass die projizierte Position möglichst nah an der tatsächlichen Position liegen soll. Beide Ziele entsprechen auch den Eigenschaften, die eine geeignete Projektion der Parteipositionen darstellt.

\paragraph{Umsetzung}
Für die konkrete Umsetzung sind die Details zu beachten. Zunächst sind die Daten aufzubereiten. Dazu werden die Daten in numerische Werte konvertiert, wobei die Position \glqq stimmt nicht zu\grqq der $0$ entspricht und \glqq stimmt zu\grqq der $1$. Die neutrale Position entspricht der Mitte bei $0,5$.
Eine wichtige Entscheidung ist außerdem welche Parteien für die Hauptkomponentenanalyse verwendet werden sollen. In diesem Fall werden ausschließlich die Bundestagsparteien verwendet. Das schränkt zwar die Daten sehr ein, aber damit konzentriert sich die Analyse auf die Parteien die relevant sind.

Die Berechnung der Hauptkomponentenanalyse nehme ich mit scikit-learn \citep{scikit-learn} vor. Die Bibliothek implementiert eine Vielzahl an Algorithmen des maschinellen Lernens. Das garantiert eine einfach Handhabung und effiziente Implementierung.

Bei der Berechnung der Haupkomponentenanalyse muss außerdem bedacht werden, wie die Hauptkomponenten skaliert werden. In der Theorie stellen sie lediglich eine Richtung dar. Wenn jedoch eine Metrik verwendet wird, spielt die Skalierung eine Rolle. Es gibt im Wesentlichen zwei Optionen. Erstens, können die Hauptkomponenten so skaliert werden, dass die Varianz in jeder Hauptkomponente die gleich ist. Zweitens, können die Hauptkomponenten so skaliert werden, dass das Verhältnis der Varianzen zwischen den Hauptkomponenten erhalten bleibt. Ich entscheide mich dafür das Verhältnis der Varianzen zu erhalten, da somit eine Gewichtung zwischen den politischen Themen erhalten bleibt.

\paragraph{Positionen der Parteien in den Hauptkomponenten}
Die Positionen der Parteien auf die Hauptkomponenten projiziert sind in Abbildung \ref{fig:party-positions-pca} dargestellt. Parteien, die sich auf einer Achse nah beieinander befinden, haben auf dieser Achse tendenziell ähnliche Positionen. Somit haben beispielsweise \glqq DIE LINKE\grqq\ und \glqq Grüne\grqq\ tendenziell ähnliche Positionen in der ersten Hauptkomponente. Dagegen haben in der zweiten Hauptkomponente \glqq AfD\grqq\ und \glqq DIE LINKE\grqq\ und am anderen Ende des Spektrums \glqq CDU/CSU\grqq\ und \glqq SPD\grqq\ ähnliche Positionen. Dass das Zentrum leer ist, liegt an der Konstruktion der Hauptkomponentenanalyse. Da sie darauf abzielt, die Varianz zu maximieren, liegen nur wenige beziehungsweise keine Datenpunkte in der Projektion in der Mitte.

\begin{figure}[htb]
	\centering
	\includegraphics[scale=1.0]{../../fig/party_positions}
	\caption{Parteipositionen im zweidimensionalen Positionsraum: Der Positionsraum ergibt sich aus den 38 Thesen der Wahl-O-Mat-Daten \citep{WahlOMat,Bolte2022QualOMat} projiziert auf die ersten beiden Hauptkomponenten. (Quelle: eigene Darstellung)\\TODO Achsenbeschriftung}
	\label{fig:party-positions-pca}
\end{figure}

\paragraph{Analyse der Hauptkomponenten}
Die theoretischen Arbeiten, die in Kapitel \ref{Sec-ABM} betrachtet wurden, nehmen zumeist an, dass es sich bei den Achsen um politische Themenfelder handelt.
Um den dimensionslosen Hauptkomponenten Bedeutung zu verleihen wird berechnet, wie die Thesen in ihnen gewichtet sind. Der größte Absolutwert im Vektor einer Hauptkomponente repräsentiert gleichzeitig diejenige These, die diese Hauptkomponente ausmacht. Daher werden die Vektorelemente der Hauptkomponenten nach absteigendem Absolutwert sortiert.
In Tabelle \ref{tab:pca1} sind die 10 Thesen aufgelistet, die für die erste Hauptkomponente am bedeutendsten sind. In Tabelle \ref{tab:pca2} sind die 10 bedeutendsten Thesen der zweiten Hauptkomponente aufgelistet.

\begin{table}[htb]
	\centering
	\sisetup{round-mode=places,round-precision=3}
	\csvreader[
		head to column names,
		head to column names prefix=MY,
		tabular				= {|l|c|L{5cm}|L{7cm}|},
		table head			= \hline \bfseries Nr. & \bfseries Wert & \bfseries Titel & \bfseries Text \\\hline,
		late after line 	= \\\hline,
		late after last line=\\\hline,
		filter				= {\value{csvrow}<10},
	]{../../fig/statements_pca0.csv}{}{
		\MYindex & \num{\MYvalue} & \MYlabel & \MYtext
	}
	\caption{Zehn bedeutendsten Thesen der ersten Hauptkomponente der Parteipositionen: Sortiert nach absteigendem Absolutwert, was gleichbedeuted mit absteigender Relevanz ist. Für Parteien die einen positiven Wert in der Hauptkomponente haben gilt tendenziell: Sie lehnen Thesen mit positivem Wert ab und stimmen Thesen mit negativem Wert zu. (Quelle: eigene Darstellung)}
	\label{tab:pca1}
\end{table}

In der Spalte \glqq Wert\grqq ist der Wert im Vektor der Hauptkomponente gelistet. Je größer der Wert, desto größer ist die Bedeutung für diese Richtung. Ist der Wert positiv, so bedeutet das eine Ablehnung der These. Ist der Wert negativ bedeutet das eine Zustimmung zur These.
Zur Erläuterung ein kurzes Beispiel: Die \glqq AfD\grqq hat die beiden bedeutendsten Thesen der ersten Hauptkomponente, die These 29 mit 0 (Zustimmung) und die These 3 mit 1 (Ablehnung) beantwortet. In der Projektion werden diese Werte mit dem Wert in der Hauptkomponente multipliziert, was einen Wert von $0*0.259+1*(-0.259)=-0.259$ ergibt. Deshalb befindet sich die \glqq AfD\grqq eher im negativen Teil der ersten Hauptkomponente, also eher links. Parteien die diese Thesen genau gegenteilig beantwortet haben befinden sich eher im positiven Teil, also eher rechts, siehe dazu auch Abbilung \ref{fig:party-positions-pca}.

\begin{table}%[htb]
	\centering
	\sisetup{round-mode=places,round-precision=3}
	\csvreader[
	head to column names,
	head to column names prefix=MY,
	tabular				= {|l|c|L{5cm}|L{7cm}|},
	table head			= \hline \bfseries Nr. & \bfseries Wert & \bfseries Titel & \bfseries Text \\\hline,
	late after line 	= \\\hline,
	late after last line=\\\hline,
	filter				= {\value{csvrow}<10},
	]{../../fig/statements_pca1.csv}{}{
		\MYindex & \num{\MYvalue} & \MYlabel & \MYtext
	}
	\caption{Zehn bedeutendsten Thesen der zweiten Hauptkomponente der Parteipositionen: Sortiert nach absteigendem Absolutwert, was gleichbedeuted mit absteigender Relevanz ist. Für Parteien die einen positiven Wert in der Hauptkomponente haben gilt tendenziell: Sie lehnen Thesen mit positivem Wert ab und stimmen Thesen mit negativem Wert zu. (Quelle: eigene Darstellung)}
	\label{tab:pca2}
\end{table}

\paragraph{Interpretation der ersten Hauptkomponente}
Die Thesen der ersten Hauptkomponente in Tabelle \ref{tab:pca1} beinhalten mehrere Umweltschutzthemen, wie
\begin{itemize}
	\item Ausbau erneuerbarer Energien
	\item Besteuerung von Pkw-Diesel
	\item Tempolimit
\end{itemize}
Dabei bedeutet im positiven Teil der Hauptkomponente, dass Umweltschutz befürwortet wird.

Das zweite bedeutende Thema in der ersten Hauptkomponente ist Verteilungspolitik. Die zugehörigen Thesen sind
\begin{itemize}
	\item Freibetrag bei der Grunderwerbssteuer
	\item Gesetzliche Krankenversicherung
	\item Schuldenschnitt für Griechenland
	\item Vermögenssteuer
	\item Sachgrundlose Befristung
\end{itemize}
Dabei bedeutet der positive Teil der Hauptkomponente, dass eine soziale Umverteilung befürwortet wird.

Die Thesen \glqq Erhöhung der Verteidigungsausgaben\grqq\ und \glqq Abbau von Staatsschulden\grqq\ lassen sich nicht einwandfrei zuordnen.

Zusammenfassend gilt, dass die erste Hauptkomponente das Spektrum von \glqq rechts\grqq\ und \glqq braun\grqq\ nach \glqq links\grqq\ und \glqq grün\grqq\ abbildet.

\paragraph{Interpretation der zweiten Hauptkomponente}
Die zehn bedeutendsten Thesen der zweiten Hauptkomponente, die in Tabelle \ref{tab:pca2} zu sehen sind, sind nur schwer zuzuordnen. Beispielsweise stimmen Parteien, die sich im positiven Teil der zweiten Komponente befinden, also eher oben, der These \glqq Vorgezogener Renteneintritt\grqq\ zu, lehnen jedoch die These \glqq Sozialer Wohnungsbau\grqq\ ab. Dies sind beides linke Themen, die aber völlig entgegengesetzt beantwortet werden. Deshalb ist eine eindeutige Zuordnung nur schwer möglich.

\section{Positionen der Wähler}\label{Sec-Wählerpositionen}

\subsection{Politbarometer Daten}
Die Daten des Politbarometers \citep{politbarometer} werden von der Forschungsgruppe Wahlen Mannheim erhoben im Auftrag des ZDF. Die Erhebung erfolgt telefonisch und ist repräsentativ für das gesamte Bundesgebiet. Die Daten umfassen eine lange Historie von 1977 bis 2020. Für den Zweck in dieser Arbeit, werden diejenigen Daten ausgewählt, die am besten mit dem Zeitpunkt des Wahl-O-Maten übereinstimmen.

Der Daten des Politbarometers sind sehr umfangreich. Es werden viele verschiedene politische Ansichten erfasst. Für diese Arbeit werden folgende Variablen verwendet:
\begin{itemize}
	\item V6 Parteienwahl Absicht\\
	Die Frage lautete von 2010 bis 2020 (in anderen Jahren ähnlich): \glqq Und welche Partei würden Sie wählen?\grqq.\\
	Diese Antwort verwende ich um die Umfrageteilnehmer einer Partei zuzuordnen. Die Angaben dieser Frage sind außerdem zum jeweiligen Erhebungszeitpunkt aktueller als die Angaben bei der Variable \glqq V7 Wahl: Rückerinnerung\grqq.

	\item V8 bis V14 Skalometer Parteien\\
	Die Frage lautete von 1989 bis 2010 (in anderen Jahren ähnlich): \glqq Und nun noch etwas genauer zu den Parteien. Stellen Sie sich bitte einmal ein Thermometer vor, das von plus 5 bis minus 5 geht, mit einem Nullpunkt dazwischen. Sagen Sie mir mit diesem Thermometer, was Sie von den einzelnen Parteien halten. +5 bedeutet, dass Sie sehr viel von der Partei halten. -5 bedeutet, dass Sie überhaupt nichts von ihr halten. Mit den Werten dazwischen können Sie Ihre Meinung abgestuft sagen.\grqq\\
	Zu den konkreten Parteien wurde dann gefragt: \glqq Was halten Sie von der [Parteiname]?\grqq, wobei SPD, CDU, CSU, FDP, Grüne, AfD und Die Linken abgefragt werden.\\
	TODO Erläuterung Akkumulation CDU und CSU\\
	Diese Variable wird in dieser Arbeit verwendet, um die Wählerpositionen abzuleiten. Dabei werden die vorher berechneten Parteipositionen und die Bewertung der einzelnen Parteien berücksichtigt.

	\item V22 Links-Rechts-Kontinuum\\
	Die Frage lautete ab 2010: \glqq Wenn von Politik die Rede ist, hört man immer wieder die Begriffe 'links' und 'rechts'. Wir hätten gerne von Ihnen gewusst, ob Sie sich selbst eher links oder eher rechts einstufen. Stellen Sie sich	dazu bitte noch einmal ein Thermometer vor, das diesmal aber nur von 0 bis 10 geht. 0 bedeutet sehr links, 10 bedeutet sehr rechts. Mit den Werten dazwischen können Sie Ihre Meinung abgestuft sagen. Wo würden Sie sich einstufen?\grqq\\
	Diese Variable wird verwendet um die Qualität der Wählerpositionierung  und die Interpretation der Hauptkomponentenanalyse zu beurteilen.
\end{itemize}

[TODO Datenaufbereitung, Datenauswahl]

\subsection{Wählerpositionierung mithilfe der Parteipositionen}
\paragraph{Metrik}
Hier werden die Wähler mithilfe des Skalometers positioniert. Zu jeder Partei steht eine Bewertung des Wählers zwischen -5 und +5 zur Verfügung. Dies ist eine Metrik der Wähler, wobei ein hoher Wert einen geringen Abstand zur Partei und ein niedriger Wert einen großen Abstand zur Parteiposition bedeutet.
Fraglich ist jedoch wie diese Metrik umgesetzt werden kann. In dieser Arbeit wird eine exponentielle Gewichtung der Parteipositionen vorgenommen. Somit ist die Wählerposition $x_i$:
\begin{equation}
	x_i = \frac{\sum_j e^{r_{ij}} p_j}{\sum_j e^{r_{ij}}}
\end{equation}
berechnet aus den Parteipositionen $p_j$, und den Bewertungen des Wählers $r_{ij}$.
Diese Entscheidung hat Vor- und Nachteile:
\begin{itemize}
	\item Berechenbarkeit: Die Formel ist immer berechenbar, da der Nenner stets positiv ist.
	\item Positive Bewertungen werden stärker gewichtet. Aufgrund der Exponentialfunktion spielen fast ausschließlich die besten Bewertungen eine Rolle. Dies ist positiv in dem Sinn, dass bei einer Wahl der Wähler nur eine Stimme hat und es daher wichtiger ist, welche Partei der Wähler mag, als welche er nicht mag.
	\item Negative Bewertungen werden als positiv berechnet: Andererseits wird eine stark negative Gewichtung zwar als quasi Null gewichtet, aber immer noch positiv gewichtet was nicht der Realtität entspricht.
	\item Konvexkombination: Es handelt sich bei der Berechnung um eine Konvexkombination. Das bedeutet, dass Wählerpositionen außerhalb der konvexen Hülle der Parteipositionen gar nicht möglich sind. Dagegen ist es in der Realität durchaus wahrscheinlich, dass Wählerpositionen extremer sind als jede Parteiposition.
	\item Robustheit gegen Verschiebung: Es denkbar, dass Wähler unter einer genauen Zahl als Bewertung, zum Beispiel +3, etwas anderes verstehen. Die hier angewendete Formel hat den Vorteil, dass das Ergebnis invariant ist gegenüber einer pauschal bessereren oder schlechtereren Bewertung aller Parteien. Der einzig entscheidende Faktor ist der Abstand zwischen den Bewertungen.
	\item Nichtlinearität: TODO
\end{itemize}

TODO alternative Metriken

Das Ergebnis der Wählergewichtungen ist in Abbildung \ref{fig:voter-positions-pca} dargestellt.

\begin{figure}[htb]
	\centering
	\includegraphics[scale=1.0]{../../fig/voter_distribution}
	\caption{TODO caption (Quelle: eigene Darstellung)}
	\label{fig:voter-positions-pca}
\end{figure}

\paragraph{Analyse Rechts-Links}
Im Politbarometer stehen auch Daten zur Verfügung, wie sich die Befragten auf einer links-rechts-Skala einschätzen. Die Antworten sind in Abbildung \ref{fig:voter-positions-pca-left-right} veranschaulicht.

\begin{figure}[htb]
	\centering
	\includegraphics[scale=1.0]{../../fig/voter_distribution_left_right}
	\caption{TODO caption (Quelle: eigene Darstellung)}
	\label{fig:voter-positions-pca-left-right}
\end{figure}

Nun stellt sich die Frage, ob dieses Ergebnis mit der Interpretation der ersten Hauptkomponente als links-rechts-Skala übereinstimmt. Tatsächlich ist es so, dass die Wähler im linken Teil des Schaubilds sich tendenziell als links identifizieren und diejenigen im rechten Teil des Schaubilds, sich als neutral beziehungsweise rechts identifizieren. Dies bestätigt also die Interpretation der Haupkomponente.

Dieses Muster ist zwar eindeutig aber eindeutig nicht homogen. Das liegt schlichtweg daran, dass Wähler nicht homogen sind und somit Abweichungen erwartbar sind.

In der Richtung der zweiten Hauptkomponente lässt sich hingegen kein eindeutiger Gradient feststellen.

\paragraph{Parteipräferenz}

Ein weiterer Indikator für die Qualität der Partei- und Wählerverteilung ist die Identifikation mit Parteien. Da die Wählerpositionen so gewählt sind, dass eine positive Bewertung der Partei auch als positionelle Nähe zur Partei gewertet wird, ist zu erwarten, dass die Wähler auch diese Partei wählen würden. Die Wahlabsicht ist in Abbildung \ref{fig:voter-positions-pca-party-affiliation} dargestellt.

\begin{figure}[htb]
	\centering
	\includegraphics[scale=1.0]{../../fig/voter_distribution_party_affiliation}
	\caption{TODO caption (Quelle: eigene Darstellung)}
	\label{fig:voter-positions-pca-party-affiliation}
\end{figure}

Wie erwartet, zeigt sich bei jeder Partei ein eindeutiger Bereich im näheren Umfeld, in dem fast alle Wähler diese Partei wählen würden. Außerdem ein bisschen weiterer Bereich, in dem eine gewisse Wahrscheinlichkeit besteht, dass die Partei gewählt wird. Eine Außnahme dabei bilden die Wähler der AfD. Es gibt einige AfD-Wähler die sich unmittelbar auf der Position der CDU/CSU befinden. Das kann zwei Ursachen haben. Einerseits könnte es sein, dass die Wähler die CDU/CSU gut bewerten aber AfD wählen. Wahrscheinlicher ist jedoch, dass sie FDP und AFD beide gut bewerten und AfD wählen. Das hier abgebildete Modell kann das nicht erklären. Somit gibt es hier Wähler in positioneller Nähe der CDU/CSU, die jedoch nicht diese Partei wählen. Insgesamt wird die Parteizuordnung also in dieser Hauptkomponentenanalyse mit dieser Wählerpositionszuordnung gut abgebildet. Allerdings gibt es auch Fälle, die nicht gut abgebildet werden.

\paragraph{Gesamtbeurteilung Wählerpositionierung}
TODO

\section{Dynamik agentenbasiert modelliert}

Die in Kapitel [TODO] und Kapitel [TODO] erstellten Daten der Partei- und Wählerpositionen dienen nun als Grundlage für ein dynamisches Modell. Das Modell von \citet{laver2005policy} das in Kapitel [TODO] vorgestellt wurde wird hier umgesetzt.

\subsection{Umsetzung des Modells}

Zur Umsetzung des agentenbasierten Modells wird hier Mesa \citep{mesa2020} verwendet. Das Mesa Projekt ist ein open-source Projekt das auf github zu finden ist [TODO mesa repository]. 
Mesa bietet ein Gerüst innerhalb dessen ein agentenbasiertes Modell einfach und effizient umgesetzt werden kann. Es bietet Werkzeuge zum Aufbau, der Analyse und der Visualisierung dieser Modelle.
% TODO cite https://mesa.readthedocs.io/en/latest/overview.html

Zunächst wird das Modell in das allgemeine Modell und seine Agenten eingeteilt. In diesem Fall sind die Parteien die Agenten, da sie ihre Position dynamisch ändern. Die Wähler könnten auch als Agenten modelliert werden. Allerdings ist es hier einfacher sie als Teil des äußeren Modells zu betrachten, da sie lediglich ihre Parteipräferenz ändern. So können die Berechnungen effizienter durchgeführt werden, als wenn alle Wähler einzelne Agenten wären.

\paragraph{Implementierung}Das Modell übernimmt die Berechnung der Wählerzuordnung und der daraus abgeleiteten Größen wie zum Beispiel den Wähleranteil einer Partei. Das Modell führt die Zeitschritte aus. Innerhalb eines Zeitschritts hat Parteie als Agente die Aufgabe, ihre neue Position für den nächsten Zeitschritt zu berechnen. Dabei kann sie auf die Ressourcen des Modells zurückgreifen und die Art und Weise hängt vom Typ der Partei ab. Wie in Kapitel [TODO] beschrieben teilt \citet{laver2005policy} die Parteien in die Typen Aggregator, Hunter, Predator und Sticker ein. Bei programmierten Modellen stecken wichtige Elemente auch in Implementierungsdetails. Im Folgenden wird insbesondere auf folgende Punkte eingegangen:
\begin{itemize}
\item Einheitslänge
\item Hunter mit stagnierendem Stimmenanteil
\item Ausführungsreihenfolge
\item Anzahl Zyklen
\end{itemize}

\paragraph{Einheitslänge}
\citet{laver2005policy} verwendet im Modell, insbesondere für Hunter und Predator eine Einheitslänge als Bewegung für den nächsten Zeitschritt. Allerdings ist nicht eindeutig geklärt wie diese Einheitslänge definiert ist.
[TODO Vermutung anhand Quelle äußern]
Das die Länge nicht eindeutig geklärt ist, ist ein Problem insofern, dass sich die Ergebnisse je nach Einheitslänge erheblich unterscheiden können. Bei einer zu großen Einheitslänge sind die Positionsanpassungen der Parteien zu groß. Dadurch verlieren die Positionsanpassungen ihren Sinn und das Modell kann sich nicht stabilieren. Ist die Einheitslänge dagegen zu klein, braucht das Modell sehr viele Zyklen um sich zu ändern beziehungsweise zu stabilisieren. Im schlimmsten Fall funktioniert das Modell gar nicht, weil beispielsweise der Hunter aufgrund der Diskretität der Wähler mit einer kleinen Schrittlänge gar keine Stimmenanteiländerung feststellt.
[TODO Ich lege es als $0,1$ fest]

\paragraph{Hunter mit stagnierendem Stimmenanteil}
Direkt mit der Einheitslänge verbunden ist die Problematik was der Hunter tut, wenn sein Stimmenanteil gleich bleibt. Wird die momentane Richtung beibehalten, dann verhält sich der Hunter eher explorativ. Dagegen kann diese Strategie katastrophal scheitern, wenn beispielsweise der Stimmenanteil bei Null liegt und der Hunter sich immer weiter von den Wählern entfernt. Die andere Möglichkeit ist, dass der Hunter seine Richtung wechselt. Dann ist die Strategie des Hunters stabiler. Allerdings besteht die Gefahr, dass zu exploriert wird. Diese Eigenschaft muss also direkt mit der Einheitslänge abgestimmt werden.
Laver entscheidet sich für den Richtungswechsel, weshalb diese Arbeit diesen Fall ebenso handhabt \citep[S.\,280]{laver2005policy}.

\paragraph{Ausführungsreihenfolge}
Die Ausführungsreihenfolge zwischen den Parteien kann einen kleinen Unterschied machen. Da in dieser Version die Wählerzuordnungen erst aktualisiert werden, wenn alle Parteien ihre Positionen festgelegt haben, macht es für die meisten Parteitypen keinen Unterschied. Einzig der Predator macht seine Position von der Position der anderen Parteien abhängig. Da es im Normalfall keinen großen Unterschied macht, wird die Reihenfolge als zufällig gewählt.
[TODO im Code umsetzen]

\paragraph{Anzahl Zyklen}
Laver zeigt in seiner Arbeit, dass ein Modell mit bis zu 10 Parteien des Typs Aggregator mit bis zu 1000 Wählern nach spätestens 55 Zyklen statisch wird \citep[S.\,271-2]{laver2005policy}. Das dient als Orientierung, um abzuschätzen, wann ein Modell statisch wird.

\paragraph{Ausführungsdauer}
Die resultierende Implementierung ist effizient. So kann beispielsweise ein System mit ??? Wählern und ??? Parteien ??? Zyklen innerhalb von ??? Sekunden berechnen [TODO]. Kleinere Systeme haben eine vernachlässigbare Ausführungsdauer. Somit sind alle Systeme die, die auf den Daten dieser Arbeit basieren problemlos berechenbar.

\subsection{Ergebnisse}

\blindtext[4]
