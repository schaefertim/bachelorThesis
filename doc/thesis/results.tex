%%%%%%%%%%%%%%%%%%%%%%%%%%%%%%%%%%%%%%%%%%%%%%%%%%%%%%%%%%%%%%%%%%%%
% Ergebnisse
%%%%%%%%%%%%%%%%%%%%%%%%%%%%%%%%%%%%%%%%%%%%%%%%%%%%%%%%%%%%%%%%%%%%

\chapter{Empirische Analyse}\label{Kap-Empirische-Analyse}

\section{Positionen der Parteien}\label{Sec-Parteienpositionen}

[TODO grundsätzliche Erläuterungen warum Parteipositionen notwendig sind]

\subsection{Wahl-O-Mat Daten}
Wahl-O-Mat \citep{WahlOMat} ist ein Tool, das von der Bundeszentrale für politische Bildung zur Verfügung gestellt wird. Der Wahl-O-Mat wird wissenschaftlich begleitet von Prof.\,Dr.\,Stefan Marschall \citep{MarschallWahlOMat}.

Das Ziel des Wahl-O-Maten bestehe darin, über \glqq wesentliche und unterscheidbare Inhalte der Parteien\grqq zu informieren. Außerdem soll so das politische Interesse insbesondere vor den Wahlen aber auch nach den Wahlen geweckt werden. 
Die Thesen werden von von Schülern, Auszubildenden und Studenten entwickelt unter Begleitung des Wahl-O-Mat Teams. Deren ungefähr 80 Thesen werden den Parteien zur Beantwortung vorgelegt. Die Parteien geben an, ob sie zustimmen, nicht zustimmen oder neutral sind. Unter dem Kriterium der Unterscheidbarkeit der Parteien werden  38 Thesen ausgewählt und bilden den Wahl-O-Mat. %[TODO cite https://www.sozwiss.hhu.de/institut/abteilungen/politikwissenschaft/politik-ii/prof-dr-stefan-marschall/forschungsprojekte/wahl-o-mat-forschung/was-ist-der-wahl-o-mat]

[TODO wissenschaftliche Publikationen]

Die Daten umfassen ausdrücklich keine Daten von Wählern. Eine Anfrage bei Prof.\,Dr.\,Stefan Marschall hat ergeben, dass \glqq die Logfiles direkt nach dem Wahl-O-Mat-Einsatz aus Datenschutzgründen vernichtet [werden]. Sie stehen auch uns [wissenschaftliche Begleitung des Wahl-O-Mats] nicht zur Auswertung zur Verfügung.\grqq

Für diese Arbeit greife ich auf die GitHub-Datenbank Qual-O-Mat von Felix Bolte \citep{Bolte2022QualOMat} zurück. Diese Datenbank sammelt alle verfügbaren Wahl-O-Mat Daten mit den Antworten der Parteien und legt diese in einem strukturierten Format ab. Somit kann auf die Daten leichter zugegriffen werden.

\subsection{PCA Wahl-O-Mat}
\paragraph{Principal Component Analysis}
Die Wahl-O-Mat Daten stellen die Positionen der Parteien im Hinblick auf die Thesen dar. Angesichts dessen, dass viele Thesen abgefragt werden ist der Raum der Positionen hochdimensional. Daher nehme ich eine Dimensionsreduktion vor. Mein Mittel der Wahl ist eine Principal Component Analysis (PCA) vorzunehmen.
[TODO Erläuterung PCA allgemein]

\paragraph{Umsetzung}
[TODO Erklärung Code]
[TODO Konertieren Antworten [0, 1], Abwägung Skalierung, Richtung, Parteien usw.]
Die Positionen der Parteien auf die PCs projiziert sind in Abbildung \ref{fig:party-positions-pca} dargestellt.

\begin{figure}[htb]
	\centering
	\includegraphics[scale=1.0]{../../fig/party_positions}
	\caption{TODO caption}
	\label{fig:party-positions-pca}
\end{figure}

\paragraph{Analyse der Principal Components}
Um den dimensionslosen PCs Bedeutung zu verleihen analysiere ich aus welchen Thesen sie sich zusammensetzen. Der größte Absolutwert im Vektor einer PC repräsentiert gleichzeitig diejenige These, die diese PC ausmacht. Daher sortiere ich die PCs nach absteigendem Absolutwert und sehe mir an welche Thesen am wichtigsten sind. In Tabelle [TODO Tabelle einbinden] ...
[TODO Tabelle statements]

[TODO grobe Interpretation]

\section{Positionen der Wähler}\label{Sec-Wählerpositionen}

\subsection{Politbarometer Daten}
[TODO Wer?]
[Datenerhebung]
[Datenbeschreibung]

\subsection{Wählerpositionierung mithilfe der Parteipositionen}
\paragraph{Metrik}
[TODO Diskussion Metrik]
[Entscheidung für exponentielle Gewichtung]
\begin{figure}[htb]
	\centering
	\includegraphics[scale=1.0]{../../fig/voter_distribution}
	\caption{TODO caption}
	\label{fig:voter-positions-pca}
\end{figure}

\paragraph{Analyse Rechts-Links}
\begin{figure}[htb]
	\centering
	\includegraphics[scale=1.0]{../../fig/voter_distribution_left_right}
	\caption{TODO caption}
	\label{fig:voter-positions-pca-left-right}
\end{figure}
[Vergleich mit Interpretation PCs]

\paragraph{Parteipräferenz}
\begin{figure}[htb]
	\centering
	\includegraphics[scale=1.0]{../../fig/voter_distribution_party_affiliation}
	\caption{TODO caption}
	\label{fig:voter-positions-pca-party-affiliation}
\end{figure}
[Analyse]

\paragraph{Gesamtbeurteilung Wählerpositionierung}


\section{Weitere Daten}\label{Sec-Weitere-Daten}

\subsection{Wählerwanderung}\label{Sec-Wählerwanderung}

\subsection{Politisches Spitzenpersonal}\label{Sec-Spitzenpersonal}
