%%%%%%%%%%%%%%%%%%%%%%%%%%%%%%%%%%%%%%%%%%%%%%%%%%%%%%%%%%%%%%%%%%%%
% Ergebnisse
%%%%%%%%%%%%%%%%%%%%%%%%%%%%%%%%%%%%%%%%%%%%%%%%%%%%%%%%%%%%%%%%%%%%

\chapter{Empirische Analyse}\label{Kap-Empirische-Analyse}

Zunächst wird die Arbeit von \citet{schofield1998germany} vorgestellt.
Diese Arbeit gibt eine Orientierung, wie empirisch gearbeitet werden kann.
Anschließend werden Daten über Partei- und Wählerpositionen gesammelt und in einen gemeinsamen Positionsraum berechnet. Auf diese Positionen wird dann das agentenbasierte Modell von \citet{laver2005policy} angewendet.

\paragraph{Programmcode}
Ein wesentlicher Teil dieser Arbeit beinhaltet Programmcode. Dieser ist mit ungefähr 900 Zeilen\footnote{Gezählt mit dem Kommandozeilenbefehl \texttt{find . -name '*.py' | xargs wc -l}} relativ umfangreich.
Dieser Programmcode ist in der digitalen Abgabe enthalten und wird außerdem auf Anfrage zur Verfügung gestellt.
Diese Art der Bereitstellung erleichtert außerdem die weitere Verwendung dieser Codebasis.

\section{Schofield et al. als Vorbild}

\citet{schofield1998germany} führen eine empirische Analyse der politischen Situation in Deutschland und Niederlande 1979 durch. Darauf wenden sie spieltheoretische Modelle an, um zu analysieren, ob die Parteien rational handeln. Diese Studie dient als ein Vorbild für diese Arbeit.

\paragraph{Daten}
\citet{schofield1998germany} nutzen Umfragedaten von Wählern als Grundlage. Die Umfragedaten beinhalten sowohl Antworten von Wählern, als auch Antworten der Parteieliten auf die selben politischen Thesen. Daraus konstruieren die Autoren die Wähler- und Parteipositionen in einem zweidimensionalen Raum.
\citep[S.\,268-269]{schofield1998germany}

\paragraph{Wählerpotentiale}
Die Autoren untersuchen nun, ob die Parteien mehr Wähler gewinnen könnten, wenn sie ihre Position ändern.
Dabei wird angenommen, dass die anderen Parteien ihre Position beibehalten.
Dabei stellt sich heraus, dass einzelne Parteien erhebliche Wählerpotential hätten, wenn sie ihre Position anpassen.
Dies geschehe jedoch nicht, da andere Faktoren wie die tatsächliche Politik nach der Wahl eine erhebliche Rolle spielen.
\citep[S.\,276-277]{schofield1998germany}

\paragraph{Koalitionen}
Wie die Politik nach der Wahl ist, hängt davon ab, welche Regierungskoalition sich bildet. Beispielsweise kann im Modell eine Partei quasi sicher sein, dass sie Teil einer Regierung ist, wenn sie sich Richtung Medianwähler bewegt. Allerdings ist diese Position unter Umständen sehr weit weg von der ursprünglichen, echten Präferenz der Parteielite.
Ebenso kann eine Partei eine extremere Position einnehmen, um mehr Verhandlungsspielraum in einer Koalition zu besitzen.
In den untersuchten Fällen bot sich nur einer Partei, die Möglichkeit eventuell eine extremere Position einzunehmen. \citep[S.\,277-281]{schofield1998germany}

\paragraph{Schlussfolgerung}
Somit sind sowohl das Erschließen neuer Wähler, als auch das Positionieren für eine Regierungskoalition also strategische Möglichkeiten. Dies widerspricht dem Ansatz von \citet{downs1957economic}, stets die Wähleranteile zu maximieren. \citep[S.\,282-283]{schofield1998germany}

\section{Positionen der Parteien}\label{Sec-Parteienpositionen}

Es werden Partei- und Wählerpositionen zur Analyse benötigt. Diese Arbeit benutzt keinen gemeinsamen Datensatz für diese Positionen, sondern es werden zunächst die Parteipositionen bestimmt und basierend darauf die Wählerpositionen.

\subsection{Wahl-O-Mat Daten}
Wahl-O-Mat \citep{WahlOMat} ist ein Tool, das von der Bundeszentrale für politische Bildung zur Verfügung gestellt wird. Der Wahl-O-Mat wird wissenschaftlich begleitet von Prof.\,Dr.\,Stefan Marschall \citep{MarschallWahlOMat}.

Das Ziel des Wahl-O-Maten bestehe darin, über \glqq wesentliche und unterscheidbare Inhalte der Parteien\grqq\ zu informieren. Außerdem soll so das politische Interesse insbesondere vor den Wahlen aber auch nach den Wahlen geweckt werden. 
Die Thesen werden von von Schülern, Auszubildenden und Studenten entwickelt unter Begleitung des Wahl-O-Mat Teams. Deren ungefähr 80 Thesen werden den Parteien zur Beantwortung vorgelegt. Die Parteien geben an, ob sie zustimmen, nicht zustimmen oder neutral sind. Unter dem Kriterium der Unterscheidbarkeit der Parteien werden  38 Thesen ausgewählt und bilden den Wahl-O-Mat.
\citep{marschall2022wahlomatFAQ}%[https://www.sozwiss.hhu.de/institut/abteilungen/politikwissenschaft/politik-ii/prof-dr-stefan-marschall/forschungsprojekte/wahl-o-mat-forschung/was-ist-der-wahl-o-mat]

Beispielhaft für die wissenschaftliche Begleitung zeigen \citet{heinsohn2019effects}, dass der Wahl-O-Mat das allgemeine politische Interesse und das Interesse an Wahlkampagnen steigert. In dieser Studie konnte allerdings kein positiver Effekt auf die politische Bildung der Bürger nachgewiesen werden. \citep[S.\,257-258]{heinsohn2019effects}

Die Daten umfassen ausdrücklich keine Daten von Wählern. Eine Anfrage bei Prof.\,Dr.\,Stefan Marschall hat ergeben, dass
\begin{itemize}
	\item[] \glqq die Logfiles direkt nach dem Wahl-O-Mat-Einsatz aus Datenschutzgründen vernichtet [werden]. Sie stehen auch uns [wissenschaftliche Begleitung des Wahl-O-Mats] nicht zur Auswertung zur Verfügung.\grqq
\end{itemize}
Das ist auch der Grund, warum die Partei- und Wählerpositionen aus zwei verschiedenen Datensätzen berechnet werden.

Für diese Arbeit werden nicht direkt die Daten des Wahl-O-Mats verwendet, sondern auf die GitHub-Datenbank Qual-O-Mat von \citet{Bolte2022QualOMat} zugegriffen. Diese Datenbank sammelt alle verfügbaren Wahl-O-Mat Daten mit den Antworten der Parteien und legt diese in einem strukturierten Format ab.
Somit können die Daten besser verarbeitet werden.

\subsection{Hauptkomponentenanalyse Wahl-O-Mat}\label{sec:Hauptkomponentenanalyse}

Die Wahl-O-Mat Daten stellen die Positionen der Parteien im Hinblick auf die Thesen dar. Angesichts dessen, dass 38 Thesen abgefragt werden ist der Raum der Positionen hochdimensional. Daher wird eine Dimensionsreduktion vorgenommen. Mein Mittel der Wahl ist eine Hauptkomponentenanalyse vorzunehmen.

\paragraph{Hauptkomponentenanalyse}
Die Hauptkomponentenanalyse hat nach \citet{bishop2006pattern} zwei verschiedene Motivationen. Die erste Motivation ist es, die Hauptkomponenten so zu wählen, dass die Varianz entlang der Hauptkomponenten maximiert wird.
Die zweite Motivation ist es, die Länge der Projektion zu minimieren. \citep[Kap.\,12.1, S.\,561-570]{bishop2006pattern}

Die Varianz zu maximieren entspricht im vorliegenden Fall, dem Ziel, dass sich die Positionen der Parteien entlang der Hauptkomponenten maximal unterscheiden sollen.
Den Projektionsabstand zu minimieren bedeutet, dass die projizierte Parteiposition möglichst nah an der ursprünglichen Parteiposition liegen soll.
Beide Ziele entsprechen auch den Eigenschaften, die eine geeignete Projektion der Parteipositionen darstellt.

\paragraph{Datenauswahl}
Für die konkrete Umsetzung sind die Details zu beachten.
Zunächst sind die Daten aufzubereiten. Dazu werden die Daten in numerische Werte konvertiert, wobei die Position \glqq stimmt nicht zu\grqq\ der $0$ entspricht und \glqq stimmt zu\grqq\ der $1$.
Die neutrale Position entspricht der Mitte bei $0,5$.
Eine wichtige Entscheidung ist außerdem, welche Parteien für die Hauptkomponentenanalyse verwendet werden sollen.
In diesem Fall werden ausschließlich die Bundestagsparteien verwendet.
Das schränkt zwar die Daten sehr ein, aber damit konzentriert sich die Analyse auf die Parteien die relevant sind.

\paragraph{Berechnung}
Die Berechnung der Hauptkomponentenanalyse wird mit scikit-learn \citep{scikit-learn} umgesetzt. Die Bibliothek implementiert eine Vielzahl an Algorithmen des maschinellen Lernens. Das garantiert eine einfach Handhabung und effiziente Implementierung.

Bei der Berechnung der Haupkomponentenanalyse muss außerdem bedacht werden, wie die Hauptkomponenten skaliert werden. In der Theorie stellen sie lediglich eine Richtung dar. Wenn jedoch eine Metrik verwendet wird, spielt die Skalierung eine Rolle. Es gibt im Wesentlichen zwei Optionen. Erstens, können die Hauptkomponenten so skaliert werden, dass die Varianz in jeder Hauptkomponente die gleiche ist. Zweitens, können die Hauptkomponenten so skaliert werden, dass das Verhältnis der Varianzen zwischen den Hauptkomponenten erhalten bleibt. Diese Arbeit erhält das Verhältnis der Varianzen, da somit eine gewisse Gewichtung zwischen den politischen Themen erhalten bleibt.

\paragraph{Positionen der Parteien in den Hauptkomponenten}
Die Positionen der Parteien auf die Hauptkomponenten projiziert sind in Abbildung \ref{fig:party-positions-pca} dargestellt. Parteien, die sich auf einer Achse nah beieinander befinden, haben auf dieser Achse tendenziell ähnliche Positionen. Somit haben beispielsweise \glqq DIE LINKE\grqq\ und \glqq Grüne\grqq\ tendenziell ähnliche Positionen in der ersten Hauptkomponente. Dagegen haben in der zweiten Hauptkomponente \glqq AfD\grqq\ und \glqq DIE LINKE\grqq\ und am anderen Ende des Spektrums \glqq CDU/CSU\grqq\ und \glqq SPD\grqq\ ähnliche Positionen. Dass das Zentrum leer ist, liegt an der Konstruktion der Hauptkomponentenanalyse. Da sie darauf abzielt, die Varianz zu maximieren, liegen nur wenige beziehungsweise keine Datenpunkte in der Projektion in der Mitte.

\begin{figure}[htb]
	\centering
	\includegraphics[scale=1.0]{../../fig/party_positions}
	\caption{Parteipositionen im zweidimensionalen Positionsraum: Der Positionsraum ergibt sich aus den 38 Thesen der Wahl-O-Mat-Daten \citep{WahlOMat,Bolte2022QualOMat} projiziert auf die ersten beiden Hauptkomponenten. (Quelle: eigene Darstellung)}
	\label{fig:party-positions-pca}
\end{figure}

\paragraph{Analyse der Hauptkomponenten}
Die theoretischen Arbeiten, die in Kapitel \ref{Sec-ABM} betrachtet wurden, nehmen zumeist an, dass es sich bei den Achsen um politische Themenfelder handelt.
Um den dimensionslosen Hauptkomponenten Bedeutung zu verleihen wird berechnet, wie die Thesen des Wahl-O-Mats in ihnen gewichtet sind. Der größte Absolutwert im Vektor einer Hauptkomponente repräsentiert gleichzeitig diejenige These, die diese Hauptkomponente ausmacht. Daher werden die Vektorelemente der Hauptkomponenten nach absteigendem Absolutwert sortiert.
In Tabelle \ref{tab:pca1} sind die 10 Thesen aufgelistet, die für die erste Hauptkomponente am bedeutendsten sind. In Tabelle \ref{tab:pca2} sind die 10 bedeutendsten Thesen der zweiten Hauptkomponente aufgelistet.

\begin{table}[htb]
	\centering
	\sisetup{round-mode=places,round-precision=3}
	\csvreader[
		head to column names,
		head to column names prefix=MY,
		tabular				= {|l|c|L{5cm}|L{7cm}|},
		table head			= \hline \bfseries Nr. & \bfseries Wert & \bfseries Titel & \bfseries Text \\\hline,
		late after line 	= \\\hline,
		late after last line=\\\hline,
		filter				= {\value{csvrow}<10},
	]{../../fig/statements_pca0.csv}{}{
		\MYindex & \num{\MYvalue} & \MYlabel & \MYtext
	}
	\caption{Zehn bedeutendsten Thesen der ersten Hauptkomponente der Parteipositionen: Sortiert nach absteigendem Absolutwert, was gleichbedeuted mit absteigender Relevanz ist. Für Parteien die einen positiven Wert in der Hauptkomponente haben gilt tendenziell: Sie lehnen Thesen mit positivem Wert ab und stimmen Thesen mit negativem Wert zu. (Quelle: eigene Darstellung)}
	\label{tab:pca1}
\end{table}

In der Spalte \glqq Wert\grqq\ ist der Wert im Vektor der Hauptkomponente gelistet. Je größer der Wert, desto größer ist die Bedeutung für diese Richtung. Ist der Wert positiv, so bedeutet das eine Ablehnung der These. Ist der Wert negativ bedeutet das eine Zustimmung zur These.
Zur Erläuterung ein kurzes Beispiel: Die \glqq AfD\grqq\ hat die beiden bedeutendsten Thesen der ersten Hauptkomponente, die These 29 mit 0 (Zustimmung) und die These 3 mit 1 (Ablehnung) beantwortet. In der Projektion werden diese Werte mit dem Wert in der Hauptkomponente multipliziert, was einen Wert von $0*0.259+1*(-0.259)=-0.259$ ergibt. Deshalb befindet sich die \glqq AfD\grqq\ eher im negativen Teil der ersten Hauptkomponente, also eher links. Parteien, die diese Thesen genau gegenteilig beantwortet haben, wie beispielsweise \glqq DIE LINKE\grqq , befinden sich eher im positiven Teil, also eher rechts, siehe dazu auch Abbilung \ref{fig:party-positions-pca}.

\begin{table}%[htb]
	\centering
	\sisetup{round-mode=places,round-precision=3}
	\csvreader[
	head to column names,
	head to column names prefix=MY,
	tabular				= {|l|c|L{5cm}|L{7cm}|},
	table head			= \hline \bfseries Nr. & \bfseries Wert & \bfseries Titel & \bfseries Text \\\hline,
	late after line 	= \\\hline,
	late after last line=\\\hline,
	filter				= {\value{csvrow}<10},
	]{../../fig/statements_pca1.csv}{}{
		\MYindex & \num{\MYvalue} & \MYlabel & \MYtext
	}
	\caption{Zehn bedeutendsten Thesen der zweiten Hauptkomponente der Parteipositionen: Sortiert nach absteigendem Absolutwert, was gleichbedeuted mit absteigender Relevanz ist. Für Parteien die einen positiven Wert in der Hauptkomponente haben gilt tendenziell: Sie lehnen Thesen mit positivem Wert ab und stimmen Thesen mit negativem Wert zu. (Quelle: eigene Darstellung)}
	\label{tab:pca2}
\end{table}

\paragraph{Interpretation der ersten Hauptkomponente}
Die Thesen der ersten Hauptkomponente in Tabelle \ref{tab:pca1} beinhalten mehrere Umweltschutzthemen, wie
\begin{itemize}
	\item Ausbau erneuerbarer Energien
	\item Besteuerung von Pkw-Diesel
	\item Tempolimit
\end{itemize}
Alle drei Thesen haben dabei einen negativen Wert. Das bedeutet, dass Parteien, die sich im positiven Teil der Hauptkomponente befinden, Umweltschutz befürwortet.

Das zweite bedeutende Thema in der ersten Hauptkomponente ist Verteilungspolitik. Die zugehörigen Thesen sind
\begin{itemize}
	\item Freibetrag bei der Grunderwerbssteuer
	\item Gesetzliche Krankenversicherung
	\item Schuldenschnitt für Griechenland
	\item Vermögenssteuer
	\item Sachgrundlose Befristung
\end{itemize}
Wieder zeigen die Werte ein einheitliches Bild. Hier bedeutet der positive Teil der Hauptkomponente, dass eine soziale Umverteilung befürwortet wird.

Die Thesen \glqq Erhöhung der Verteidigungsausgaben\grqq\ und \glqq Abbau von Staatsschulden\grqq\ lassen sich nicht einwandfrei zuordnen oder zu einem größeren Thema zusammenfassen.

Zusammenfassend gilt, dass die erste Hauptkomponente beim Umweltschutz das Spektrum von \glqq rechts\grqq\ nach \glqq links\grqq\ und bei der Verteilungspolitik von \glqq braun\grqq\ nach \glqq grün\grqq\ abbildet.

\paragraph{Interpretation der zweiten Hauptkomponente}
Die zehn bedeutendsten Thesen der zweiten Hauptkomponente, die in Tabelle \ref{tab:pca2} zu sehen sind, sind nur schwer zu verallgemeinern. Beispielsweise stimmen Parteien, die sich im positiven Teil der zweiten Komponente befinden, also eher oben, der These \glqq Vorgezogener Renteneintritt\grqq\ zu, lehnen jedoch die These \glqq Sozialer Wohnungsbau\grqq\ ab. Dies sind beides Themen der Verteilungspolitik, die aber, wie an den Werten zu sehen ist, völlig entgegengesetzt beantwortet werden. Deshalb ist eine eindeutige Zuordnung nur schwer möglich.
Damit hat die zweite Hauptkomponente keine eindeutige Interpretation.

\section{Positionen der Wähler}\label{Sec-Wählerpositionen}

Nun werden die Positionen der Wähler bestimmt. In dieser Arbeit werden dazu die Bewertungen der Wähler gegenüber den Parteien herangezogen. Die Positionen der Wähler werden dann relativ zu den Positionen der Parteien bestimmt.

\subsection{Politbarometer Daten}
Die Daten des Politbarometers \citep{politbarometer} werden von der Forschungsgruppe Wahlen Mannheim erhoben im Auftrag des ZDF. Die Erhebung erfolgt telefonisch und ist repräsentativ für das gesamte Bundesgebiet. Die Daten umfassen eine lange Historie von 1977 bis 2020.

Die Daten des Politbarometers sind sehr umfangreich, denn es werden viele verschiedene politische Ansichten erfasst.
Die für diese Arbeit wichtigsten Variablen werden im Folgenden vorgestellt.

\paragraph{V6 Parteienwahl Absicht}
Die Frage lautete von 2010 bis 2020 (in anderen Jahren ähnlich): \glqq Und welche Partei würden Sie wählen?\grqq. \citep[Variablendokumentation, S.\,12]{politbarometer}\\
Diese Antwort wird verwendet, um die Umfrageteilnehmer einer Partei zuzuordnen. Die Angaben dieser Frage sind außerdem zum jeweiligen Erhebungszeitpunkt aktueller als die Angaben bei der Variable \glqq V7 Wahl: Rückerinnerung\grqq.

\paragraph{V8 bis V14 Skalometer Parteien}
Die Frage lautete von 1989 bis 2010 (in anderen Jahren ähnlich): \glqq Und nun noch etwas genauer zu den Parteien. Stellen Sie sich bitte einmal ein Thermometer vor, das von plus 5 bis minus 5 geht, mit einem Nullpunkt dazwischen. Sagen Sie mir mit diesem Thermometer, was Sie von den einzelnen Parteien halten. +5 bedeutet, dass Sie sehr viel von der Partei halten. -5 bedeutet, dass Sie überhaupt nichts von ihr halten. Mit den Werten dazwischen können Sie Ihre Meinung abgestuft sagen.\grqq\\
Zu den konkreten Parteien wurde dann gefragt: \glqq Was halten Sie von der [Parteiname]?\grqq, wobei SPD, CDU, CSU, FDP, Grüne, AfD und Die Linken abgefragt werden. \citep[Variablendokumentation, S.\,28-29]{politbarometer}\\
Diese Variable wird in dieser Arbeit verwendet, um die Wählerpositionen abzuleiten. Dabei werden die vorher berechneten Parteipositionen und die Bewertung der einzelnen Parteien berücksichtigt.

\paragraph{V22 Links-Rechts-Kontinuum}
Die Frage lautete ab 2010: \glqq Wenn von Politik die Rede ist, hört man immer wieder die Begriffe 'links' und 'rechts'. Wir hätten gerne von Ihnen gewusst, ob Sie sich selbst eher links oder eher rechts einstufen. Stellen Sie sich	dazu bitte noch einmal ein Thermometer vor, das diesmal aber nur von 0 bis 10 geht. 0 bedeutet sehr links, 10 bedeutet sehr rechts. Mit den Werten dazwischen können Sie Ihre Meinung abgestuft sagen. Wo würden Sie sich einstufen?\grqq \citep[Variablendokumentation, S.\,61-62]{politbarometer}\\
Diese Variable wird verwendet um die Qualität der Wählerpositionierung  und die Interpretation der Hauptkomponentenanalyse zu beurteilen.

\paragraph{Datenauswahl}
Für den Zweck in dieser Arbeit, werden diejenigen Daten ausgewählt, die am besten mit dem Zeitpunkt des Wahl-O-Maten übereinstimmen.
Die Bundestagswahl fand zum 24.\,September 2017 statt, daher werden vom Politbarometer die gesamten Umfragedaten für August 2017 verwendet.
Außerdem werden alle Umfrageteilnehmer aus dem Datensatz entfernt, welchen eine zur Berechnung wichtige Angabe fehlt.
Beispielsweise werden zur Positionsberechnung diejenigen Wähler entfernt, bei denen mindestens eine Präferenzangabe fehlt.
Weitere wichtige Arbeitsschritte sind unter Anderem die Konvertierung der Angaben in verarbeitbare Formate und die Vereinheitlichung mit dem Wahl-O-Mat-Datensatz.

\subsection{Wählerpositionierung mithilfe der Parteipositionen}

\paragraph{Metrik}
Hier werden die Wähler mithilfe des Skalometers positioniert. Zu jeder Partei steht eine Bewertung des Wählers zwischen -5 und +5 zur Verfügung. Dies ist eine Metrik der Wähler, wobei ein hoher Wert einen geringen Abstand zur Partei und ein niedriger Wert einen großen Abstand zur Parteiposition bedeutet.
Fraglich ist jedoch wie diese Metrik umgesetzt werden kann. In dieser Arbeit wird eine exponentielle Gewichtung der Parteipositionen vorgenommen. Somit ist die Wählerposition $x_i$:
\begin{equation}
	x_i = \frac{\sum_j e^{r_{ij}} p_j}{\sum_j e^{r_{ij}}}
\end{equation}
berechnet aus den Parteipositionen $p_j$ aus Kapitel \ref{Sec-Parteienpositionen}, und den Bewertungen des Wählers $r_{ij}$.
Diese Entscheidung hat Vor- und Nachteile:
\begin{itemize}
	\item Berechenbarkeit: Die Formel ist immer berechenbar, da der Nenner stets positiv ist.
	\item Positive Bewertungen werden stärker gewichtet als negative: Aufgrund der Exponentialfunktion spielen fast ausschließlich die besten Bewertungen eine Rolle. Dies ist positiv in dem Sinn, dass bei einer Wahl der Wähler nur eine Stimme hat und es daher wichtiger ist, welche Partei der Wähler mag, als welche er nicht mag.
	\item Negative Bewertungen werden als marginal positiv berücksichtigt: Eine stark negative Gewichtung zwar als quasi Null gewichtet, aber immer noch positiv gewichtet was nicht der Realtität entspricht.
	\item Konvexkombination: Es handelt sich bei der Berechnung um eine Konvexkombination. Das bedeutet, dass Wählerpositionen außerhalb der konvexen Hülle der Parteipositionen gar nicht möglich sind. Dagegen ist es in der Realität durchaus wahrscheinlich, dass Wählerpositionen extremer sind als jede Parteiposition. Diese Tatsache, dass es somit keine Ausreißer gibt, hat wiederum den Vorteil, dass die Daten leicht beherrschbar sind.
	\item Invarianz gegen Verschiebung: Es ist denkbar, dass Wähler pauschal alle Parteien um +1 besser oder -1 schlechter bewerten. Die hier angewendete Formel hat den Vorteil, dass das Ergebnis invariant ist gegenüber einer pauschal besseren oder schlechteren Bewertung aller Parteien. Der einzig entscheidende Faktor ist die Differenz zwischen den Bewertungen.
\end{itemize}
Das Ergebnis der Wählergewichtungen ist in Abbildung \ref{fig:voter-positions-pca} dargestellt.

\begin{figure}[htb]
	\centering
	\includegraphics[scale=1.0]{../../fig/voter_distribution}
	\caption{Wählerpositionen im zweidimensionalen Raum der Parteipositionen. Positionen sind mithilfe der Politbarometer-Daten \citep{politbarometer} berechnet. (Quelle: eigene Darstellung)}
	\label{fig:voter-positions-pca}
\end{figure}

\paragraph{Analyse Rechts-Links}
Im Politbarometer stehen auch Daten zur Verfügung, wie sich die Befragten auf einer links-rechts-Skala einschätzen. Die Antworten sind in Abbildung \ref{fig:voter-positions-pca-left-right} veranschaulicht.

\begin{figure}[htb]
	\centering
	\includegraphics[scale=1.0]{../../fig/voter_distribution_left_right}
	\caption{Wähler auf einer links-rechts-Skala auf Grundlage der Politbarometer-Daten \citep{politbarometer}. (Quelle: eigene Darstellung)}
	\label{fig:voter-positions-pca-left-right}
\end{figure}

Nun stellt sich die Frage, ob dieses Ergebnis mit der Interpretation der ersten Hauptkomponente als links-rechts-Skala aus Kapitel \ref{sec:Hauptkomponentenanalyse} übereinstimmt.
Tatsächlich ist es so, dass die Wähler im rechten Teil des Schaubilds sich tendenziell als links identifizieren und diejenigen im linken Teil des Schaubilds, sich als neutral beziehungsweise rechts identifizieren.
Dieses Muster ist zwar eindeutig aber eindeutig nicht homogen. Das liegt schlichtweg daran, dass Wähler nicht homogen sind und somit Abweichungen erwartbar sind.
Insgesamt bestätigt die Darstellung die Interpretation der ersten Haupkomponente als links-rechts-Skala.

In der Richtung der zweiten Hauptkomponente lässt sich hingegen kein eindeutiger Gradient feststellen, was mit der Unklarheit der zweiten Hauptkomponente auf der links-rechts-Skala bestätigt.

\paragraph{Parteipräferenz}

Ein weiterer Indikator für die Qualität der Partei- und Wählerverteilung ist die Identifikation mit Parteien. Da die Wählerpositionen so gewählt sind, dass eine positive Bewertung der Partei auch als positionelle Nähe zur Partei gewertet wird, ist zu erwarten, dass die Wähler auch diese Partei wählen würden. Die Wahlabsicht ist in Abbildung \ref{fig:voter-positions-pca-party-affiliation} dargestellt.

\begin{figure}[htb]
	\centering
	\includegraphics[scale=1.0]{../../fig/voter_distribution_party_affiliation}
	\caption{Wahlabsicht der Wähler auf Grundlage der Politbarometer-Daten \citep{politbarometer}. (Quelle: eigene Darstellung)}
	\label{fig:voter-positions-pca-party-affiliation}
\end{figure}

Wie erwartet, zeigt sich bei jeder Partei ein eindeutiger Bereich im näheren Umfeld, in dem fast alle Wähler diese Partei wählen würden. Außerdem ein bisschen weiterer Bereich, in dem eine gewisse Wahrscheinlichkeit besteht, dass die Partei gewählt wird. Es gibt jedoch einige Besonderheiten. Zum Beispiel befinden sich die FDP-Wähler nicht in Richtung der AfD. Außerdem gibt es einzelne Wähler, die weit aus dem direkten Einflussbereich ihrer Partei herausfallen. Dies sind Wähler, die in den Umfragen zwar eine hohe Bewertung für eine Partei angeben, jedoch eine andere Partei wählen wird.
Insgesamt wird die Parteizuordnung also in dieser Hauptkomponentenanalyse mit dieser Wählerpositionszuordnung gut abgebildet. Eine Zuordnung der Wähler alleine aufgrund der Distanz zur Partei ist zwar keine perfekte Interpretation, stellt jedoch eine gut Annäherung dar.

\paragraph{Gesamtbeurteilung Wählerpositionierung}
Insgesamt erscheint die Wählerpositionierung sinnvoll und kosistent. Die Links-Rechts-Komponente wird gut abgebildet und ist konsistent mit der Interpretation der Hauptkomponente. Der Einflussbereich der Parteien ist ebenfalls sehr gut abgebildet.

\section{Dynamik agentenbasiert modelliert} \label{Sec:Dynamik-Anwendung}

Die in Kapitel \ref{Sec-Parteienpositionen} und Kapitel \ref{Sec-Wählerpositionen} berechneten Partei- und Wählerpositionen dienen nun als Grundlage für ein dynamisches Modell. Das Modell von \citet{laver2005policy} das in Kapitel \ref{sec:ABM-Dynamik} vorgestellt wurde wird umgesetzt.

\subsection{Umsetzung des Modells}

\paragraph{Mesa}
Zur Umsetzung des agentenbasierten Modells wird Mesa \citep{mesa2020} verwendet. Mesa ist ein open-source Projekt\footnote{https://gitlab.freedesktop.org/mesa/mesa}. 
Mesa bietet ein Gerüst, innerhalb dessen ein agentenbasiertes Modell einfach und effizient umgesetzt werden kann. Es bietet Werkzeuge zum Aufbau, der Analyse und der Visualisierung dieser Modelle. Außerdem ist eine ausführliche Dokumentation verfügbar\footnote{https://mesa.readthedocs.io/en/latest/}.

\paragraph{Einteilung Agenten und Modell}
Für Mesa wird das Modell in das äußere Modell und seine Agenten eingeteilt. In diesem Fall sind die Parteien die Agenten, da sie ihre Positionen dynamisch ändern. Die Wähler könnten auch als Agenten modelliert werden. Allerdings ist es hier einfacher sie als Teil des äußeren Modells zu betrachten, da sie lediglich ihre Parteipräferenz ändern. So können die Berechnungen effizienter durchgeführt werden, als wenn alle Wähler einzelne Agenten wären.

\paragraph{Implementierung}
Das Modell übernimmt die Berechnung der Wählerzuordnung und der daraus abgeleiteten Größen wie zum Beispiel den Wähleranteil einer Partei. Das Modell führt die Zeitschritte aus. Innerhalb eines Zeitschritts hat die Partei als Agent die Aufgabe, ihre neue Position für den nächsten Zeitschritt zu berechnen. Dabei kann sie auf die Ressourcen des Modells zurückgreifen und die Art und Weise hängt vom Typ der Partei ab. Wie in Kapitel \ref{sec:ABM-Dynamik} beschrieben teilt \citet{laver2005policy} die Parteien in die Typen Aggregator, Hunter, Predator und Sticker ein. Bei programmierten Modellen stecken wichtige Elemente auch in Implementierungsdetails.
Im Folgenden werden für das Modell von \citet{laver2005policy} die Implementierungsdetails diskutiert.

\paragraph{Einheitslänge}
\citet{laver2005policy} verwendet im Modell, insbesondere für Hunter und Predator eine Einheitslänge als Bewegung für den nächsten Zeitschritt. Allerdings ist nicht eindeutig geklärt wie diese Einheitslänge definiert ist.
Das die Länge nicht eindeutig geklärt ist, ist ein Problem insofern, dass sich die Ergebnisse je nach Einheitslänge erheblich unterscheiden können. Bei einer zu großen Einheitslänge sind die Positionsanpassungen der Parteien zu groß. Dadurch verlieren die Positionsanpassungen ihren Sinn und das Modell kann sich nicht stabilieren. Ist die Einheitslänge dagegen zu klein, braucht das Modell sehr viele Zyklen um sich zu ändern beziehungsweise zu stabilisieren. Im schlimmsten Fall funktioniert das Modell gar nicht, weil beispielsweise der Hunter aufgrund der Diskretität der Wähler mit einer kleinen Schrittlänge gar keine Stimmenanteiländerung feststellt.
Aufgrund einer Abbildung Lavers \citep[Abb.\,4]{laver2005policy} wird ein ungefährer Wert abgeschätzt. Im Modell hier wird ein Wert von 0,1 gewählt.

\paragraph{Hunter mit stagnierendem Stimmenanteil}
Direkt mit der Einheitslänge verbunden ist die Problematik, was der Hunter tut, wenn sein Stimmenanteil gleich bleibt. Wird die momentane Richtung beibehalten, dann verhält sich der Hunter eher explorativ. Dagegen kann diese Strategie katastrophal scheitern, wenn beispielsweise der Stimmenanteil bei Null liegt und der Hunter sich immer weiter von den Wählern entfernt. Die andere Möglichkeit ist, dass der Hunter seine Richtung wechselt. Dann ist die Strategie des Hunters stabiler. Allerdings besteht die Gefahr, dass zu wenig exploriert wird. Diese Eigenschaft muss also direkt mit der Einheitslänge abgestimmt werden.
Laver entscheidet sich für den Richtungswechsel, weshalb diese Arbeit diesen Fall ebenso handhabt \citep[S.\,280]{laver2005policy}.

\paragraph{Ausführungsreihenfolge}
Die Ausführungsreihenfolge zwischen den Parteien kann einen kleinen Unterschied machen. Da in dieser Version die Wählerzuordnungen erst aktualisiert werden, wenn alle Parteien ihre Positionen festgelegt haben, macht es für die meisten Parteitypen keinen Unterschied. Einzig der Predator macht seine Position von der Position der anderen Parteien abhängig. Da es im Normalfall keinen großen Unterschied macht, wird die Reihenfolge als zufällig gewählt.

\paragraph{Anzahl Zyklen}
Laver zeigt in seiner Arbeit, dass ein Modell mit bis zu 10 Parteien des Typs Aggregator mit bis zu 1000 Wählern nach spätestens 55 Zyklen statisch wird \citep[S.\,271-272]{laver2005policy}. Das dient als Orientierung, um abzuschätzen, wann ein Modell statisch wird.

\paragraph{Ausführungsdauer}
Die resultierende Implementierung ist effizient. So kann beispielsweise ein Beispielsystem mit 1847 Wählern, 6 Parteien und 1000 Zyklen unter 0,1 Sekunden berechnet werden\footnote{Systeminformation: 8 x Intel(R) Core(TM) i7-7700 CPU @ 3.60GHz, 16GB RAM}. Kleinere Systeme haben eine vernachlässigbare Ausführungsdauer. Für die nachfolgend aufgeführten Experimente ist die Ausührungsdauer also keine Schwierigkeit.

\subsection{Ergebnisse}

Nun wird das Modell auf die zuvor abgeleiteten Partei- und Wählerpositionen angewendet. Das Modell wird initialisiert mit den berechneten Positionen. Als Parteitypen werden die verschiedenen Konstellationen aus den Ergebnissen von \citet{laver2005policy} durchgespielt, die in Kapitel \ref{sec:ABM-Dynamik} vorgestellt wurden. Das Modell läuft für 1000 Zyklen, damit die Modelle, die statisch werden, statisch geworden sind. Im Folgenden werden die Ergebnisse für die verschiedenen Konstellationen präsentiert.

\paragraph{Nur Aggregatoren}

Bei nur Aggregatoren wird laut \citet{laver2005policy} eine gleichmäßige Verteilung über die Positionen der Wähler erwartet. Das Ergebnis auf den Daten für Deutschland 2017 ist in Abbildung \ref{fig:laver-aggregator6} zu sehen. Tatsächlich ist diese gleichmäßige Verteilung zu beobachten. Darüber hinaus ist bemerkenswert, dass die Parteipositionen sehr ähnlich zu den Ausgangspositionen sind. Das bedeutet, dass dieses Modell die tatsächlichen Parteipositionen relativ gut beschreibt.

\begin{figure}[htb]
	\centering
	\includegraphics[scale=1.0]{../../fig/fig_laver_aggregator6.png}
	\caption{TODO (Quelle: eigene Darstellung)}
	\label{fig:laver-aggregator6}
\end{figure}

\paragraph{Nur Hunter}

In Abbildung \ref{fig:laver-hunter6} ist das Ergebnis zu sehen, wenn alle sechs Parteien Hunter sind. Die Parteien bewegen sich in Richtung des Punktes der höchsten Konzentration im unteren Bereich. Allerdings haben die Parteien genügend Abstand, sodass jede Partei einen ausreichenden Anteil an Wählern besitzt. Im Gegensatz zum Ergebnis von \citet{laver2005policy} sammeln sich die Parteien nicht kreisförmig um einen Punkt, sondern haben ein komplexeres Muster.

\begin{figure}[htb]
	\centering
	\includegraphics[scale=1.0]{../../fig/fig_laver_hunter6.png}
	\caption{TODO (Quelle: eigene Darstellung)}
	\label{fig:laver-hunter6}
\end{figure}

\paragraph{Aggregatoren gegen einzelne Hunter oder Predatoren}

In dieser Arbeit wurden mehrere Szenarien durchgespielt, in denen vier oder fünf Aggregatoren gegen einzelne Hunter oder Predatoren antreten.
Repräsentativ dafür ist in Abbildung \ref{fig:laver-aggregator4-hunter-predator} die Situation von vier Aggregatoren mit einem Hunter und einem Predator dargestellt.
Die Hunter und Predatoren bewegen sich ausnahmslos in den unteren Bereich mit der hohen Wählerdichte.
Allerdings schaffen es nur die Hunter, wobei auch diese es nicht immer schaffen, den attraktivsten Platz ganz unten einzunehmen.
Ansonsten konzentrieren sich die Positionen der einzelnen Hunter und Predatoren auf den Bereich rechts davon.
Wie bei \citet{laver2005policy} bleiben die Aggregatoren sehr robust und behalten wesentliche Wähleranteile.
In der Situation auf den realen Daten sind die einzelnen Hunter und Predatoren jedoch erfolgreicher als bei \citet{laver2005policy}.
Dies liegt insbesondere an der unregelmäßigen Wählerverteilung, die sich von der symmetrischen Situation mit Normalverteilung bei \citet{laver2005policy} deutlich unterscheidet.

\begin{figure}[htb]
	\centering
	\includegraphics[scale=1.0]{../../fig/fig_laver_aggregator4_predator_hunter.png}
	\caption{TODO (Quelle: eigene Darstellung)}
	\label{fig:laver-aggregator4-hunter-predator}
\end{figure}

\paragraph{Mehrere Hunter gegen einzelnen Predator}

In Abbildung \ref{fig:laver-hunter5-predator} ist das Ergebnis von fünf Huntern mit einem einzelnen Predator dargestellt. Das Ergebnis ist sehr dynamisch und vom Zufall beeinflusst. Allen Ergebnissen ist jedoch gemein, dass der Predator tendenziell nicht die erfolgreichste Partei ist. Stattdessen pendelt er wie bei \citet{laver2005policy} zwischen den erfolgreichsten Huntern hin und her. Dadurch hält er sich zwar in der Region mit der größten Wählerdichte auf, kann dort aber nicht unbeschränkt dominieren.

\begin{figure}[htb]
	\centering
	\includegraphics[scale=1.0]{../../fig/fig_laver_hunter5_predator.png}
	\caption{TODO (Quelle: eigene Darstellung)}
	\label{fig:laver-hunter5-predator}
\end{figure}
