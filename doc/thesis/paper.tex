%%%%%%%%%%%%%%%%%%%%%%%%%%%%%%%%%%%%%%%%%%%%%%%%%%%%%%%%%%%%%%%%%%%%%%%%%%%%%
%%% LaTeX-Rahmen fuer das Erstellen von Masterarbeiten
%%%%%%%%%%%%%%%%%%%%%%%%%%%%%%%%%%%%%%%%%%%%%%%%%%%%%%%%%%%%%%%%%%%%%%%%%%%%%

%%%%%%%%%%%%%%%%%%%%%%%%%%%%%%%%%%%%%%%%%%%%%%%%%%%%%%%%%%%%%%%%%%%%%%%%%%%%%
%%% allgemeine Einstellungen
%%%%%%%%%%%%%%%%%%%%%%%%%%%%%%%%%%%%%%%%%%%%%%%%%%%%%%%%%%%%%%%%%%%%%%%%%%%%%

\documentclass[twoside,12pt,a4paper]{report}
%\usepackage{reportpage}
\usepackage{epsf}
\usepackage{graphics, graphicx}
\usepackage{latexsym}
\usepackage[margin=10pt,font=small,labelfont=bf]{caption}
\usepackage[utf8]{inputenc}
\usepackage[toc,page]{appendix}

% meine Pakete
\usepackage[authoryear]{natbib}
\citestyle{aysep={,}}
\usepackage{ngerman}
\usepackage{tikz}
\usetikzlibrary{calc}
\usepackage{amsmath}


\textwidth 14cm
\textheight 22cm
\topmargin 0.0cm
\evensidemargin 1cm
\oddsidemargin 1cm
%\footskip 2cm
\parskip0.5explus0.1exminus0.1ex

% Kann von Student auch nach persönlichem Geschmack verändert werden.
\pagestyle{headings}

\sloppy

\begin{document}

%%%%%%%%%%%%%%%%%%%%%%%%%%%%%%%%%%%%%%%%%%%%%%%%%%%%%%%%%%%%%%%%%%%%%%%%%%%%
%%% hier steht die neue Titelseite 
%%%%%%%%%%%%%%%%%%%%%%%%%%%%%%%%%%%%%%%%%%%%%%%%%%%%%%%%%%%%%%%%%%%%%%%%%%%%
 
\begin{titlepage}
 \begin{center}
  {\huge Bachelorarbeit\\Wirtschaftswissenschaften\\[1.5cm]}
  {\huge\bf Strategische Positionierung Politischer Parteien in Deutschland\\[3cm]}
  {\large Fakultät für Wirtschaftswissenschaft\\}
  {\large Lehrstuhl für Volkswirtschaftslehre, insbes.\,Mikroökonomie\\[1cm]}
  {\large FernUniversität in Hagen\\[3.5cm]}
  {\large Tim Schäfer\\[3cm]}
  {Hagen, \today} %TODO
  \end{center}
\end{titlepage}

%%%%%%%%%%%%%%%%%%%%%%%%%%%%%%%%%%%%%%%%%%%%%%%%%%%%%%%%%%%%%%%%%%%%%%%%%%%%
%%% Titelr"uckseite: Bibliographische Angaben
%%%%%%%%%%%%%%%%%%%%%%%%%%%%%%%%%%%%%%%%%%%%%%%%%%%%%%%%%%%%%%%%%%%%%%%%%%%%

\thispagestyle{empty}
\vspace*{\fill}
\begin{minipage}{\textwidth}
\parbox[t][2cm][t]{3.5cm}{Erstkorrektor:}\hfill
\parbox[t][2cm][t]{11cm}{
	Prof.\,Dr.\,Robert Schmidt \\
	Lehrstuhl für Volkswirtschaftslehre, insbes.\,Mikroökonomie\\
	Fakultät für Wirtschaftswissenschaft
}
\parbox[t][2cm][t]{3.5cm}{Zweitkorrektor:}\hfill
\parbox[t][2cm][t]{11cm}{ %TODO
	Prof.\,Dr.\,Joachim Grosser\\
	Lehrstuhl für Volkswirtschaftslehre, insbes.\,Wirtschaftspolitik\\
	Fakultät für Wirtschaftswissenschaft
}

Eidesstattliche Erklärung:
%TODO

\vskip 2cm

Ort, Datum	\hfill Unterschrift \hfill 
\end{minipage}
\newpage

%%%%%%%%%%%%%%%%%%%%%%%%%%%%%%%%%%%%%%%%%%%%%%%%%%%%%%%%%%%%%%%%%%%%%%%%%%%%

\pagenumbering{roman}
\setcounter{page}{1}

%%%%%%%%%%%%%%%%%%%%%%%%%%%%%%%%%%%%%%%%%%%%%%%%%%%%%%%%%%%%%%%%%%%%%%%%%%%%
%%% Seite I: Zusammenfassug, Danksagung
%%%%%%%%%%%%%%%%%%%%%%%%%%%%%%%%%%%%%%%%%%%%%%%%%%%%%%%%%%%%%%%%%%%%%%%%%%%%


\section*{Abstract}

Abstract
\newpage

\section*{Danksagungen}

Danksagungen

\cleardoublepage

%%%%%%%%%%%%%%%%%%%%%%%%%%%%%%%%%%%%%%%%%%%%%%%%%%%%%%%%%%%%%%%%%%%%%%%%%%%%%
%%% Table of Contents
%%%%%%%%%%%%%%%%%%%%%%%%%%%%%%%%%%%%%%%%%%%%%%%%%%%%%%%%%%%%%%%%%%%%%%%%%%%%%

\renewcommand{\baselinestretch}{1.3}
\small\normalsize

\tableofcontents

\renewcommand{\baselinestretch}{1}
\small\normalsize

\cleardoublepage

%%%%%%%%%%%%%%%%%%%%%%%%%%%%%%%%%%%%%%%%%%%%%%%%%%%%%%%%%%%%%%%%%%%%%%%%%%%%%
%%% Der Haupttext, ab hier mit arabischer Numerierung
%%% Mit \input{dateiname} werden die Datei `dateiname' eingebunden
%%%%%%%%%%%%%%%%%%%%%%%%%%%%%%%%%%%%%%%%%%%%%%%%%%%%%%%%%%%%%%%%%%%%%%%%%%%%%

\pagenumbering{arabic}
\setcounter{page}{1}

%% Introduction
%%%%%%%%%%%%%%%%%%%%%%%%%%%%%%%%%%%%%%%%%%%%%%%%%%%%%%%%%%%%%%%%%%%%
% Einleitung
%%%%%%%%%%%%%%%%%%%%%%%%%%%%%%%%%%%%%%%%%%%%%%%%%%%%%%%%%%%%%%%%%%%%

\chapter{Einleitung}\label{Kap-Einleitung}
Politische Partien müssen zu einer Vielfalt von Themen ihre Standpunkte bestimmen. Abhängig von ihren Standpunkten können sie mehr oder weniger Wähler und Wählerinnen überzeugen. Daher ist es naheliegend, dass eine rational agierende Partei ihre Standpunkte unter strategischen Gesichtspunkten auswählt. Andererseits sind Parteien auch idelogische Konstrukte, die ihre Standpunkte aus Überzeugung auswählen. Diese Arbeit soll die Hypothese überprüfen, dass politische Parteien ihre Standpunkte unter strategischen Gesichtspunkten wählen. Dabei zeigen bisherige Arbeiten, dass Parteien unter strategischen Gesichtspunkte ihre Wählerpotentiale nicht ausschöpfen \citep{schofield1998germany}.

Im ersten Teil der Arbeit sollen theoretische Modelle vorgestellt werden. Dazu werden  zunächst einfache Modelle des räumlichen Wettbewerbs betrachtet. Hotellings Gesetz besagt, dass Wettbewerber minimale Differenzierung betreiben \citep{hotelling1929}.

Aspremont et al. \citep{aspremont1979} kritisieren dieses Modell. Sie nehmen an, dass die Transportkosten anstatt linear quadratisch im Abstand sind. Das führt dazu, dass Wettbewerber zur Gewinnmaximierung zu maximaler Differenzierung neigen, also genau das Gegenteil von Hotellings Gesetz. Auch im politischen Kontext ist es denkbar, dass die Nutzenfunktion quadratisch im Abstand der politischen Position ist.

Als dritte Perspektive auf politischen Parteienwettbewerb sollen moderne Ansätze betrachtet werden, welche häufig agentenbasiert sind. Diese Ansätze sind in der Regel dynamisch, das heißt es finden mehrere Perioden statt. Marchi und Page \citep{marchi2014ABMs} geben einen Überblick über solche agentenbasierten Modell. Dabei gibt es sowohl Modelle die die Zahl der Parteien exogen festlegen, als auch Modelle, die die Zahl der Parteien flexibel halten \citep{laver2007endogenousParties}.

Abschließend wird das Wahlsystem in Deutschland vorgestellt. Außerdem sollen aus den Theorien konkrete überprüfbare Vorhersagen für deutsche Parteien und Wähler abgeleitet werden.

Im zweiten Teil der Arbeit sollen die abgeleiteten Hypothesen anhand von Daten getestet werden. 

Um Aussagen über die Verteilung der Parteienpositionen zu testen, werden die Wahl-O-Mat-Daten (Bundeszentrale für politische Bildung, 2021) verwendet. Diese Daten umfassen für alle Parteien eine Aussage (Ablehnung, neutral, Zustimmung) für eine Vielzahl von Positionen.

Um Aussagen über Wählerpositionen zu testen, werden die Daten des Politbarometers (Forschungsgruppe Wahlen, Mannheim, 2021) verwendet. Dies sind Wählerdaten, die generelle politische Präferenzen abfragen und einzelne zusätzliche Positionen erfragen.

Als dritte Überprüfung werden Daten über Wählerwanderung berücksichtigt. Außerdem werden Daten über die Meinung zu Spitzenpolitiker erfragt. Dies erlaubt Rückschlüsse darauf, ob die viele Wähler zwischen mehreren Parteien unentschlossen sind. Das kann entweder bedeuten, dass die Parteipositionen sehr nah beieinander liegen oder es kann bedeuten, dass die Nutzenfunktion der Wähler sehr unempfindlich gegenüber den Parteipositionen ist.

In einer abschließenden Diskussion soll bewertet werden, inwiefern welche Theorie einer praktischen Überprüfung standhält und welche theoretischen Grundlagen eine geeignete Beschreibung der politischen Realität liefern.

\cleardoublepage

%% 
\chapter{Theoretische Analyse}\label{Kap-Theorie}

\noindent
Im Folgenden werden unterschiedliche Modelle des räumlichen Wettbewerbs vorgestellt. Jedes Modell wird zunächst in seiner ursprünglichen Form gezeigt. Anschließend wird analysiert inwiefern sich das Modell auf politische Gegebenheiten anwenden lässt und welche Vorhersagen sich daraus ergeben.

\section{Hotellings Gesetz}\label{Sec-Hotelling}

\subsection{Modell}\label{Sec-Hotelling-Modell}

Hotelling \citep{hotelling1929} untersucht anhand eines einfachen Modells wie sich bei einem Duopol der räumliche Wettbewerb gestaltet. Dazu nimmt er an, dass die zwei Wettbewerber auf einer räumlichen Linie ihre Position wählen können. Abbildung \ref{Fig-Linearer-Wettbewerb} zeigt mögliche Positionen der Wettbewerber. Dabei sind die Konsumenten gleichmäßig über die gesamte Linie verteilt.

\begin{figure}[htb]
	\centering
	\begin{tikzpicture}
	\coordinate (start) at (0,0) {};
	\coordinate (end) at (10,0) {};
	\coordinate (left) at (2,0) {};
	\coordinate (right) at (7,0) {};
	\draw (start) -- (end);
	\draw ($(start)+(0,5pt)$) -- ($(start)-(0,5pt)$);
	\draw ($(left)+(0,5pt)$) -- ($(left)-(0,5pt)$);
	\draw ($(right)+(0,5pt)$) -- ($(right)-(0,5pt)$);
	\draw ($(end)+(0,5pt)$) -- ($(end)-(0,5pt)$);
	\node at ($(start)+(0,10pt)$) {$0$};
	\node at ($(end)+(0,10pt)$) {$l$};
	\node at ($(left)+(0,10pt)$) {$A$};
	\node at ($(right)+(0,10pt)$) {$B$};
	
	\coordinate (startdown) at ($(start)+(0,-20pt)$) {};
	\coordinate (enddown) at ($(end)+(0,-20pt)$) {};
	\coordinate (leftdown) at ($(left)+(0,-20pt)$) {};
	\coordinate (rightdown) at ($(right)+(0,-20pt)$) {};
	\coordinate (middledown) at (4,-20pt) {};
	
	\draw[|-|] (startdown) -- (leftdown) node[midway, below] {$a$};
	\draw[|-|] (rightdown) -- (enddown) node[midway, below] {$b$};
	\draw[|-|] (leftdown) -- (middledown) node[midway, below] {$x$};
	\draw[|-|] (middledown) -- (rightdown) node[midway, below] {$y$};
	\end{tikzpicture}
	\caption{Positionen der Wettbewerber $A$ und $B$ und Konsumenten auf einer Linie von $0$ bis $l$. Die Konsumenten im Bereich $a$ und $x$ kaufen bei $A$ ein und $y$ und $b$ bei $B$. (angelehnt an \cite{hotelling1929})}
	\label{Fig-Linearer-Wettbewerb}
\end{figure}

Eine grundlegende Annahme des Modells ist, dass die Konsumenten lineare Transportkosten in Höhe von $c$ pro Längeneinheit haben. Die Konsumenten kaufen somit bei demjenigen Duopolisten bei dem die Kosten bestehend aus Preis und Transportkosten am geringsten sind.

Nun nimmt Hotelling an, dass $A$ zu einem Preis $p_1$ und $B$ zu einem Preis $p_2$ verkauft. Die Verkaufsmengen sind jeweils $q_1$ und $q_2$.

Die erste wichtige Feststellung ist, dass das Duopol nur dann ein Duopol bleibt und nicht zu einem Monopol kollabiert, wenn der Preisunterschied nicht zu hoch ist. Genauer gesagt, darf der Preisaufschlag von $A$ nicht höher sein als die Transportkosten von Wettbewerber $B$ nach $A$:
\begin{equation}\label{eqn:Preisannahme}
\begin{split}
p_1 &\leq p_2 + c(l-a-b) \\
p_2 &\leq p_1 + c(l-a-b)
\end{split}
\end{equation}
So lange Bedingung \ref{eqn:Preisannahme} erfüllt ist, kaufen alle Konsumenten links bei Wettbewerber $A$ und alle rechts bei $B$. Die Konsumenten in der Mitte teilen sich in einen Abschnitt $x$, der bei $A$ kauft, und einen Abschnitt $y$, der bei $B$ kauft.

Daraus ergeben sich zwei Bedingungen. Erstens ergeben die Längen zusammen die gesamte Länge und zweitens soll der Konsument der die beiden Abschnitte $x$ und $y$ trennt indifferent zwischen den beiden Wettbewerbern sein. Dies ist in Gleichung \ref{eqn:Gleichgewichtsbedingungen} formuliert.
\begin{equation}\label{eqn:Gleichgewichtsbedingungen}
\begin{split}
l &= a + x + y + b \\
p_1 + cx &= p_2 + cy
\end{split}
\end{equation}

Löst man dieses lineare Gleichungssystem nach $x$ und $y$ auf ergibt sich
\begin{equation}
\begin{split}
x &= \tfrac{1}{2} \left(l-a-b+\frac{p_2-p_1}{c}\right)\\
y &= \tfrac{1}{2} \left(l-a-b+\frac{p_1-p_2}{c}\right)
\end{split}
\end{equation}

Ohne Beschränkung der Allgemeinheit werden die Kosten der Wettbewerber Null gesetzt und der Gewinn ist somit
\begin{equation}
\begin{split}
\pi_1 &= p_1 q_1 = p_1 (a+x) = \tfrac{1}{2} p_1 \left(l+a-b\right) -\frac{p_1^2}{c}+\frac{p_1 p_2}{c}\\
\pi_2 &= p_2 q_2 = p_2 (b+y) = \tfrac{1}{2} p_1 \left(l-a+b\right) -\frac{p_2^2}{c}+\frac{p_1 p_2}{c}
\end{split}
\end{equation}

Der gewinnmaximierende Preis, berechnet mithilfe der Gleichungen $\frac{\partial \pi_1}{\partial p_1}=0$ und $\frac{\partial \pi_2}{\partial p_2}=0$, ist
\begin{equation}
\begin{split}
p_1 &= \left(l+\frac{a-b}{c}\right) \\
p_2 &= \left(l-\frac{a-b}{c}\right)
\end{split}
\end{equation}
mit den verkauften Mengen
\begin{equation}
\begin{split}
q_1 &= a + x = \tfrac{1}{2} \left(l + \frac{a-b}{3}\right) \\
q_2 &= b + y = \tfrac{1}{2} \left(l - \frac{a-b}{3}\right)
.\end{split}
\end{equation}

Die resultierende Gewinnfunktion der Wettbewerber ist
\begin{equation}\label{eqn:Gewinnfunktion}
\begin{split}
\pi_1 &= \frac{c}{2} \left(l+\frac{a-b}{3}\right) \\
\pi_2 &= \frac{c}{2} \left(l-\frac{a-b}{3}\right)
.\end{split}
\end{equation}

\subsection{Schlussfolgerungen}

\subsection{Hypothesen}

\section{Modell von d‘Aspremont}\label{Sec-Aspremont}

d'Aspremont \citep{aspremont1979} kritisiert das Principle of Minimum Differentiation (TODO: kontrollieren, dass definiert) von Hotelling.
Die Kritik stützt sich auf zwei Aspekte: Erstens ist kein Gleichgewichtspreis definiert, wenn die beiden Wettbewerber sich am selben Punkt befinden. Zweitens hat das Modell drastisch andere Lösungen, wenn die Transportkostenfunktion sich ändert.

\subsection{Kritik an Hotelling}
Der Gleichgewichtspreis ergibt sich als Nash-Cournot-Gleichgewicht. Dabei müssen zwei Fälle unterschieden werden: Erstens wenn sich die Wettbewerber am selben Punkt befinden und zweitens wenn die Wettbewerber auseinanderliegen.

Im Fall, dass die Wettbewerber sich beide am selben Punkt (beispielsweise in der Mitte) befinden, so herrscht reiner Preiswettbewerb. Nach Betrand (TODO: zitieren) führt das zu Preisen in Höhe der Stückkosten, in diesem Fall zu einem Preis $p_1=p_2=0$.

Liegen die Wettbewerber auseinander, so stimmt \cite{aspremont1979} mit \cite{hotelling1929} überein und präzisiert das Marktgleichgewicht. Der Gleichgewichtspreis ist dann
\begin{equation}
\begin{split}
p_1 &= c \left(l+\frac{a-b}{3}\right) \\
p_2 &= c \left(l-\frac{a-b}{3}\right)
\end{split}
\end{equation}
unter der Bedingung
\begin{equation}
\begin{split}
\left(l+\frac{a-b}{3}\right)^2 &\geq \tfrac{4}{3} l (a+2b) \\
\left(l-\frac{a-b}{3}\right)^2 &\geq \tfrac{4}{3} l (2a+b)
\end{split}
\end{equation}

\paragraph{Das Gleichgewicht tendiert zur Mitte und kollabiert.} Da $\frac{\partial \pi_1}{\partial a}>0$ und $\frac{\partial \pi_2}{\partial b}>0$ bewegen sich beide Wettbewerber zur Gewinnoptimierung zur Mitte hin. Allerdings existiert in der Mitte nur noch das Gleichgewicht mit Nullpreisen. Da dies kein wirkliches Marktgleichgewicht ist, ist das laut \cite{aspremont1979} das gesamte Modell invalidiert.

\subsection{Modell}\label{Sec-Aspremont-Modell}
\cite{aspremont1979} schlägt stattdessen ein korrigiertes Modell vor mit einer veränderten Transportkostenfunktion vor. Diese ist dann quadratisch, also $cx^2$ statt $cx$.
Das Gleichgewicht berechnet sich identisch zum Fall oben. Die Gleichgewichtspreise sind dann
\begin{equation}
\begin{split}
p_1 &= c (l-a-b) \left(l+\frac{a-b}{b}\right) \\
p_2 &= c (l-a-b) \left(l-\frac{a-b}{b}\right)
\end{split}
\end{equation}

Die Gewinnfunktionen sind dann
\begin{equation}
TODO: Gewinnfunktionen
\end{equation}

Die Ableitungen $\frac{\partial \pi_1}{\partial a}<0$ und $\frac{\partial \pi_2}{\partial b}<0$ sind also negativ. Somit bewegen sich die beiden Wettbewerber so weit wie möglich voneinander weg. Das ist das Prinzip der maximalen Differenzierung.

\subsection{Schlussfolgerungen}

\subsection{Hypothesen}

\section{Agentenbasierte Ansätze}\label{Sec-ABM}

\section{Situation in Deutschland}\label{Sec-Deutschland}

\section{Vorhersagen aus Modellen}\label{Sec-Vorhersagen}

\cleardoublepage

%%
%%%%%%%%%%%%%%%%%%%%%%%%%%%%%%%%%%%%%%%%%%%%%%%%%%%%%%%%%%%%%%%%%%%%
% Ergebnisse
%%%%%%%%%%%%%%%%%%%%%%%%%%%%%%%%%%%%%%%%%%%%%%%%%%%%%%%%%%%%%%%%%%%%

\chapter{Empirische Analyse}\label{Kap-Empirische-Analyse}

\section{Beispiel von Schofield et al}

\blindtext[3]

\section{Positionen der Parteien}\label{Sec-Parteienpositionen}

[TODO grundsätzliche Erläuterungen warum Parteipositionen notwendig sind]

\subsection{Wahl-O-Mat Daten}
Wahl-O-Mat \citep{WahlOMat} ist ein Tool, das von der Bundeszentrale für politische Bildung zur Verfügung gestellt wird. Der Wahl-O-Mat wird wissenschaftlich begleitet von Prof.\,Dr.\,Stefan Marschall \citep{MarschallWahlOMat}.

Das Ziel des Wahl-O-Maten bestehe darin, über \glqq wesentliche und unterscheidbare Inhalte der Parteien\grqq zu informieren. Außerdem soll so das politische Interesse insbesondere vor den Wahlen aber auch nach den Wahlen geweckt werden. 
Die Thesen werden von von Schülern, Auszubildenden und Studenten entwickelt unter Begleitung des Wahl-O-Mat Teams. Deren ungefähr 80 Thesen werden den Parteien zur Beantwortung vorgelegt. Die Parteien geben an, ob sie zustimmen, nicht zustimmen oder neutral sind. Unter dem Kriterium der Unterscheidbarkeit der Parteien werden  38 Thesen ausgewählt und bilden den Wahl-O-Mat. %[TODO cite https://www.sozwiss.hhu.de/institut/abteilungen/politikwissenschaft/politik-ii/prof-dr-stefan-marschall/forschungsprojekte/wahl-o-mat-forschung/was-ist-der-wahl-o-mat]

[TODO wissenschaftliche Publikationen]

Die Daten umfassen ausdrücklich keine Daten von Wählern. Eine Anfrage bei Prof.\,Dr.\,Stefan Marschall hat ergeben, dass \glqq die Logfiles direkt nach dem Wahl-O-Mat-Einsatz aus Datenschutzgründen vernichtet [werden]. Sie stehen auch uns [wissenschaftliche Begleitung des Wahl-O-Mats] nicht zur Auswertung zur Verfügung.\grqq

Für diese Arbeit greife ich auf die GitHub-Datenbank Qual-O-Mat von Felix Bolte \citep{Bolte2022QualOMat} zurück. Diese Datenbank sammelt alle verfügbaren Wahl-O-Mat Daten mit den Antworten der Parteien und legt diese in einem strukturierten Format ab. Somit kann auf die Daten leichter zugegriffen werden.

\subsection{Hauptkomponentenanalyse Wahl-O-Mat}
\paragraph{Hauptkomponentenanalyse}
Die Wahl-O-Mat Daten stellen die Positionen der Parteien im Hinblick auf die Thesen dar. Angesichts dessen, dass 38 Thesen abgefragt werden ist der Raum der Positionen hochdimensional. Daher nehme ich eine Dimensionsreduktion vor. Mein Mittel der Wahl ist eine Hauptkomponentenanalyse vorzunehmen.

Die Hauptkomponentenanalyse hat nach Bishop und Nasrabadi \citep{bishop2006pattern} zwei verschiedene Motivationen. Die erste Motivation ist es, die Hauptkomponenten so zu wählen, dass die Varianz entlang der Hauptkomponenten maximiert wird. Das entspricht im vorliegenden Fall, dem Ziel, dass sich die Positionen der Parteien entlang der Hauptkomponenten maximal unterscheiden sollen. Die zweite Motivation ist es, die Länge der Projektion zu minimieren. Im Fall der Parteipositionen bedeutet das, dass die projizierte Position möglichst nah an der tatsächlichen Position liegen soll. Beide Ziele entsprechen auch den Eigenschaften, die eine geeignete Projektion der Parteipositionen darstellt.

\paragraph{Umsetzung}
Für die konkrete Umsetzung sind die Details zu beachten. Zunächst sind die Daten aufzubereiten. Dazu werden die Daten in numerische Werte konvertiert, wobei die Position \glqq stimmt nicht zu\grqq der $0$ entspricht und \glqq stimmt zu\grqq der $1$. Die neutrale Position entspricht der Mitte bei $0,5$.
Eine wichtige Entscheidung ist außerdem welche Parteien für die Hauptkomponentenanalyse verwendet werden sollen. In diesem Fall werden ausschließlich die Bundestagsparteien verwendet. Das schränkt zwar die Daten sehr ein, aber damit konzentriert sich die Analyse auf die Parteien die relevant sind.

Die Berechnung der Hauptkomponentenanalyse nehme ich mit scikit-learn \citep{scikit-learn} vor. Die Bibliothek implementiert eine Vielzahl an Algorithmen des maschinellen Lernens. Das garantiert eine einfach Handhabung und effiziente Implementierung.

Bei der Berechnung der Haupkomponentenanalyse muss außerdem bedacht werden, wie die Hauptkomponenten skaliert werden. In der Theorie stellen sie lediglich eine Richtung dar. Wenn jedoch eine Metrik verwendet wird, spielt die Skalierung eine Rolle. Es gibt im Wesentlichen zwei Optionen. Erstens, können die Hauptkomponenten so skaliert werden, dass die Varianz in jeder Hauptkomponente die gleich ist. Zweitens, können die Hauptkomponenten so skaliert werden, dass das Verhältnis der Varianzen zwischen den Hauptkomponenten erhalten bleibt. Ich entscheide mich dafür das Verhältnis der Varianzen zu erhalten, da somit eine Gewichtung zwischen den politischen Themen erhalten bleibt.

\paragraph{Positionen der Parteien in den Hauptkomponenten}
Die Positionen der Parteien auf die Hauptkomponenten projiziert sind in Abbildung \ref{fig:party-positions-pca} dargestellt. Parteien, die sich auf einer Achse nah beieinander befinden, haben auf dieser Achse tendenziell ähnliche Positionen. Somit haben beispielsweise \glqq DIE LINKE\grqq\ und \glqq Grüne\grqq\ tendenziell ähnliche Positionen in der ersten Hauptkomponente. Dagegen haben in der zweiten Hauptkomponente \glqq AfD\grqq\ und \glqq DIE LINKE\grqq\ und am anderen Ende des Spektrums \glqq CDU/CSU\grqq\ und \glqq SPD\grqq\ ähnliche Positionen. Dass das Zentrum leer ist, liegt an der Konstruktion der Hauptkomponentenanalyse. Da sie darauf abzielt, die Varianz zu maximieren, liegen nur wenige beziehungsweise keine Datenpunkte in der Projektion in der Mitte.

\begin{figure}[htb]
	\centering
	\includegraphics[scale=1.0]{../../fig/party_positions}
	\caption{Parteipositionen im zweidimensionalen Positionsraum: Der Positionsraum ergibt sich aus den 38 Thesen der Wahl-O-Mat-Daten \citep{WahlOMat,Bolte2022QualOMat} projiziert auf die ersten beiden Hauptkomponenten. (Quelle: eigene Darstellung)\\TODO Achsenbeschriftung}
	\label{fig:party-positions-pca}
\end{figure}

\paragraph{Analyse der Hauptkomponenten}
Die theoretischen Arbeiten, die in Kapitel \ref{Sec-ABM} betrachtet wurden, nehmen zumeist an, dass es sich bei den Achsen um politische Themenfelder handelt.
Um den dimensionslosen Hauptkomponenten Bedeutung zu verleihen wird berechnet, wie die Thesen in ihnen gewichtet sind. Der größte Absolutwert im Vektor einer Hauptkomponente repräsentiert gleichzeitig diejenige These, die diese Hauptkomponente ausmacht. Daher werden die Vektorelemente der Hauptkomponenten nach absteigendem Absolutwert sortiert.
In Tabelle \ref{tab:pca1} sind die 10 Thesen aufgelistet, die für die erste Hauptkomponente am bedeutendsten sind. In Tabelle \ref{tab:pca2} sind die 10 bedeutendsten Thesen der zweiten Hauptkomponente aufgelistet.

\begin{table}[htb]
	\centering
	\sisetup{round-mode=places,round-precision=3}
	\csvreader[
		head to column names,
		head to column names prefix=MY,
		tabular				= {|l|c|L{5cm}|L{7cm}|},
		table head			= \hline \bfseries Nr. & \bfseries Wert & \bfseries Titel & \bfseries Text \\\hline,
		late after line 	= \\\hline,
		late after last line=\\\hline,
		filter				= {\value{csvrow}<10},
	]{../../fig/statements_pca0.csv}{}{
		\MYindex & \num{\MYvalue} & \MYlabel & \MYtext
	}
	\caption{Zehn bedeutendsten Thesen der ersten Hauptkomponente der Parteipositionen: Sortiert nach absteigendem Absolutwert, was gleichbedeuted mit absteigender Relevanz ist. Für Parteien die einen positiven Wert in der Hauptkomponente haben gilt tendenziell: Sie lehnen Thesen mit positivem Wert ab und stimmen Thesen mit negativem Wert zu. (Quelle: eigene Darstellung)}
	\label{tab:pca1}
\end{table}

In der Spalte \glqq Wert\grqq ist der Wert im Vektor der Hauptkomponente gelistet. Je größer der Wert, desto größer ist die Bedeutung für diese Richtung. Ist der Wert positiv, so bedeutet das eine Ablehnung der These. Ist der Wert negativ bedeutet das eine Zustimmung zur These.
Zur Erläuterung ein kurzes Beispiel: Die \glqq AfD\grqq\ hat die beiden bedeutendsten Thesen der ersten Hauptkomponente, die These 29 mit 0 (Zustimmung) und die These 3 mit 1 (Ablehnung) beantwortet. In der Projektion werden diese Werte mit dem Wert in der Hauptkomponente multipliziert, was einen Wert von $0*0.259+1*(-0.259)=-0.259$ ergibt. Deshalb befindet sich die \glqq AfD\grqq\ eher im negativen Teil der ersten Hauptkomponente, also eher links. Parteien die diese Thesen genau gegenteilig beantwortet haben befinden sich eher im positiven Teil, also eher rechts, siehe dazu auch Abbilung \ref{fig:party-positions-pca}.

\begin{table}%[htb]
	\centering
	\sisetup{round-mode=places,round-precision=3}
	\csvreader[
	head to column names,
	head to column names prefix=MY,
	tabular				= {|l|c|L{5cm}|L{7cm}|},
	table head			= \hline \bfseries Nr. & \bfseries Wert & \bfseries Titel & \bfseries Text \\\hline,
	late after line 	= \\\hline,
	late after last line=\\\hline,
	filter				= {\value{csvrow}<10},
	]{../../fig/statements_pca1.csv}{}{
		\MYindex & \num{\MYvalue} & \MYlabel & \MYtext
	}
	\caption{Zehn bedeutendsten Thesen der zweiten Hauptkomponente der Parteipositionen: Sortiert nach absteigendem Absolutwert, was gleichbedeuted mit absteigender Relevanz ist. Für Parteien die einen positiven Wert in der Hauptkomponente haben gilt tendenziell: Sie lehnen Thesen mit positivem Wert ab und stimmen Thesen mit negativem Wert zu. (Quelle: eigene Darstellung)}
	\label{tab:pca2}
\end{table}

\paragraph{Interpretation der ersten Hauptkomponente}
Die Thesen der ersten Hauptkomponente in Tabelle \ref{tab:pca1} beinhalten mehrere Umweltschutzthemen, wie
\begin{itemize}
	\item Ausbau erneuerbarer Energien
	\item Besteuerung von Pkw-Diesel
	\item Tempolimit
\end{itemize}
Dabei bedeutet im positiven Teil der Hauptkomponente, dass Umweltschutz befürwortet wird.

Das zweite bedeutende Thema in der ersten Hauptkomponente ist Verteilungspolitik. Die zugehörigen Thesen sind
\begin{itemize}
	\item Freibetrag bei der Grunderwerbssteuer
	\item Gesetzliche Krankenversicherung
	\item Schuldenschnitt für Griechenland
	\item Vermögenssteuer
	\item Sachgrundlose Befristung
\end{itemize}
Dabei bedeutet der positive Teil der Hauptkomponente, dass eine soziale Umverteilung befürwortet wird.

Die Thesen \glqq Erhöhung der Verteidigungsausgaben\grqq\ und \glqq Abbau von Staatsschulden\grqq\ lassen sich nicht einwandfrei zuordnen.

Zusammenfassend gilt, dass die erste Hauptkomponente das Spektrum von \glqq rechts\grqq\ und \glqq braun\grqq\ nach \glqq links\grqq\ und \glqq grün\grqq\ abbildet.

\paragraph{Interpretation der zweiten Hauptkomponente}
Die zehn bedeutendsten Thesen der zweiten Hauptkomponente, die in Tabelle \ref{tab:pca2} zu sehen sind, sind nur schwer zuzuordnen. Beispielsweise stimmen Parteien, die sich im positiven Teil der zweiten Komponente befinden, also eher oben, der These \glqq Vorgezogener Renteneintritt\grqq\ zu, lehnen jedoch die These \glqq Sozialer Wohnungsbau\grqq\ ab. Dies sind beides linke Themen, die aber völlig entgegengesetzt beantwortet werden. Deshalb ist eine eindeutige Zuordnung nur schwer möglich.

\section{Positionen der Wähler}\label{Sec-Wählerpositionen}

\subsection{Politbarometer Daten}
Die Daten des Politbarometers \citep{politbarometer} werden von der Forschungsgruppe Wahlen Mannheim erhoben im Auftrag des ZDF. Die Erhebung erfolgt telefonisch und ist repräsentativ für das gesamte Bundesgebiet. Die Daten umfassen eine lange Historie von 1977 bis 2020. Für den Zweck in dieser Arbeit, werden diejenigen Daten ausgewählt, die am besten mit dem Zeitpunkt des Wahl-O-Maten übereinstimmen.

Der Daten des Politbarometers sind sehr umfangreich. Es werden viele verschiedene politische Ansichten erfasst. Für diese Arbeit werden folgende Variablen verwendet:
\begin{itemize}
	\item V6 Parteienwahl Absicht\\
	Die Frage lautete von 2010 bis 2020 (in anderen Jahren ähnlich): \glqq Und welche Partei würden Sie wählen?\grqq.\\
	Diese Antwort verwende ich um die Umfrageteilnehmer einer Partei zuzuordnen. Die Angaben dieser Frage sind außerdem zum jeweiligen Erhebungszeitpunkt aktueller als die Angaben bei der Variable \glqq V7 Wahl: Rückerinnerung\grqq.

	\item V8 bis V14 Skalometer Parteien\\
	Die Frage lautete von 1989 bis 2010 (in anderen Jahren ähnlich): \glqq Und nun noch etwas genauer zu den Parteien. Stellen Sie sich bitte einmal ein Thermometer vor, das von plus 5 bis minus 5 geht, mit einem Nullpunkt dazwischen. Sagen Sie mir mit diesem Thermometer, was Sie von den einzelnen Parteien halten. +5 bedeutet, dass Sie sehr viel von der Partei halten. -5 bedeutet, dass Sie überhaupt nichts von ihr halten. Mit den Werten dazwischen können Sie Ihre Meinung abgestuft sagen.\grqq\\
	Zu den konkreten Parteien wurde dann gefragt: \glqq Was halten Sie von der [Parteiname]?\grqq, wobei SPD, CDU, CSU, FDP, Grüne, AfD und Die Linken abgefragt werden.\\
	TODO Erläuterung Akkumulation CDU und CSU\\
	Diese Variable wird in dieser Arbeit verwendet, um die Wählerpositionen abzuleiten. Dabei werden die vorher berechneten Parteipositionen und die Bewertung der einzelnen Parteien berücksichtigt.

	\item V22 Links-Rechts-Kontinuum\\
	Die Frage lautete ab 2010: \glqq Wenn von Politik die Rede ist, hört man immer wieder die Begriffe 'links' und 'rechts'. Wir hätten gerne von Ihnen gewusst, ob Sie sich selbst eher links oder eher rechts einstufen. Stellen Sie sich	dazu bitte noch einmal ein Thermometer vor, das diesmal aber nur von 0 bis 10 geht. 0 bedeutet sehr links, 10 bedeutet sehr rechts. Mit den Werten dazwischen können Sie Ihre Meinung abgestuft sagen. Wo würden Sie sich einstufen?\grqq\\
	Diese Variable wird verwendet um die Qualität der Wählerpositionierung  und die Interpretation der Hauptkomponentenanalyse zu beurteilen.
\end{itemize}

[TODO Datenaufbereitung, Datenauswahl]

\subsection{Wählerpositionierung mithilfe der Parteipositionen}
\paragraph{Metrik}
Hier werden die Wähler mithilfe des Skalometers positioniert. Zu jeder Partei steht eine Bewertung des Wählers zwischen -5 und +5 zur Verfügung. Dies ist eine Metrik der Wähler, wobei ein hoher Wert einen geringen Abstand zur Partei und ein niedriger Wert einen großen Abstand zur Parteiposition bedeutet.
Fraglich ist jedoch wie diese Metrik umgesetzt werden kann. In dieser Arbeit wird eine exponentielle Gewichtung der Parteipositionen vorgenommen. Somit ist die Wählerposition $x_i$:
\begin{equation}
	x_i = \frac{\sum_j e^{r_{ij}} p_j}{\sum_j e^{r_{ij}}}
\end{equation}
berechnet aus den Parteipositionen $p_j$, und den Bewertungen des Wählers $r_{ij}$.
Diese Entscheidung hat Vor- und Nachteile:
\begin{itemize}
	\item Berechenbarkeit: Die Formel ist immer berechenbar, da der Nenner stets positiv ist.
	\item Positive Bewertungen werden stärker gewichtet. Aufgrund der Exponentialfunktion spielen fast ausschließlich die besten Bewertungen eine Rolle. Dies ist positiv in dem Sinn, dass bei einer Wahl der Wähler nur eine Stimme hat und es daher wichtiger ist, welche Partei der Wähler mag, als welche er nicht mag.
	\item Negative Bewertungen werden als positiv berechnet: Andererseits wird eine stark negative Gewichtung zwar als quasi Null gewichtet, aber immer noch positiv gewichtet was nicht der Realtität entspricht.
	\item Konvexkombination: Es handelt sich bei der Berechnung um eine Konvexkombination. Das bedeutet, dass Wählerpositionen außerhalb der konvexen Hülle der Parteipositionen gar nicht möglich sind. Dagegen ist es in der Realität durchaus wahrscheinlich, dass Wählerpositionen extremer sind als jede Parteiposition.
	\item Robustheit gegen Verschiebung: Es denkbar, dass Wähler unter einer genauen Zahl als Bewertung, zum Beispiel +3, etwas anderes verstehen. Die hier angewendete Formel hat den Vorteil, dass das Ergebnis invariant ist gegenüber einer pauschal bessereren oder schlechtereren Bewertung aller Parteien. Der einzig entscheidende Faktor ist der Abstand zwischen den Bewertungen.
	\item Nichtlinearität: TODO
\end{itemize}

TODO alternative Metriken

Das Ergebnis der Wählergewichtungen ist in Abbildung \ref{fig:voter-positions-pca} dargestellt.

\begin{figure}[htb]
	\centering
	\includegraphics[scale=1.0]{../../fig/voter_distribution}
	\caption{Wählerpositionen im zweidimensionalen Raum der Parteipositionen. Positionen sind mithilfe der Politbarometer-Daten \citep{politbarometer} berechnet. (Quelle: eigene Darstellung)\\TODO Achsenbeschriftung}
	\label{fig:voter-positions-pca}
\end{figure}

\paragraph{Analyse Rechts-Links}
Im Politbarometer stehen auch Daten zur Verfügung, wie sich die Befragten auf einer links-rechts-Skala einschätzen. Die Antworten sind in Abbildung \ref{fig:voter-positions-pca-left-right} veranschaulicht.

\begin{figure}[htb]
	\centering
	\includegraphics[scale=1.0]{../../fig/voter_distribution_left_right}
	\caption{Wähler auf einer links-rechts-Skala auf Grundlage der Politbarometer-Daten \citep{politbarometer}. (Quelle: eigene Darstellung)}
	\label{fig:voter-positions-pca-left-right}
\end{figure}

Nun stellt sich die Frage, ob dieses Ergebnis mit der Interpretation der ersten Hauptkomponente als links-rechts-Skala übereinstimmt. Tatsächlich ist es so, dass die Wähler im rechten Teil des Schaubilds sich tendenziell als links identifizieren und diejenigen im linken Teil des Schaubilds, sich als neutral beziehungsweise rechts identifizieren. Dies bestätigt also die Interpretation der Haupkomponente.

Dieses Muster ist zwar eindeutig aber eindeutig nicht homogen. Das liegt schlichtweg daran, dass Wähler nicht homogen sind und somit Abweichungen erwartbar sind.

In der Richtung der zweiten Hauptkomponente lässt sich hingegen kein eindeutiger Gradient feststellen.

\paragraph{Parteipräferenz}

Ein weiterer Indikator für die Qualität der Partei- und Wählerverteilung ist die Identifikation mit Parteien. Da die Wählerpositionen so gewählt sind, dass eine positive Bewertung der Partei auch als positionelle Nähe zur Partei gewertet wird, ist zu erwarten, dass die Wähler auch diese Partei wählen würden. Die Wahlabsicht ist in Abbildung \ref{fig:voter-positions-pca-party-affiliation} dargestellt.

\begin{figure}[htb]
	\centering
	\includegraphics[scale=1.0]{../../fig/voter_distribution_party_affiliation}
	\caption{Wahlabsicht der Wähler auf Grundlage der Politbarometer-Daten \citep{politbarometer}. (Quelle: eigene Darstellung)\\TODO Legende positionieren, Achsenbeschriftung}
	\label{fig:voter-positions-pca-party-affiliation}
\end{figure}

Wie erwartet, zeigt sich bei jeder Partei ein eindeutiger Bereich im näheren Umfeld, in dem fast alle Wähler diese Partei wählen würden. Außerdem ein bisschen weiterer Bereich, in dem eine gewisse Wahrscheinlichkeit besteht, dass die Partei gewählt wird. Es gibt jedoch einige Besonderheiten. Zum Beispiel befinden sich die FDP-Wähler nicht in Richtung der AfD. Außerdem gibt es einzelne Wähler, die weit aus dem direkten Einflussbereich ihrer Partei herausfallen. Dies sind Wähler, die in den Umfragen zwar eine hohe Bewertung für eine Partei angeben, jedoch eine andere Partei wählen wird.
Insgesamt wird die Parteizuordnung also in dieser Hauptkomponentenanalyse mit dieser Wählerpositionszuordnung gut abgebildet. Eine Zuordnung der Wähler alleine aufgrund der Distanz zur Partei ist zwar keine perfekte Interpretation, stellt jedoch eine gut Annäherung dar.

\paragraph{Gesamtbeurteilung Wählerpositionierung}
Insgesamt erscheint die Wählerpositionierung sinnvoll und kosistent. Die Links-Rechts-Komponente wird gut abgebildet und ist konsistent mit der Interpretation der Hauptkomponente. Der Einflussbereich der Parteien ist ebenfalls sehr gut abgebildet.

\section{Dynamik agentenbasiert modelliert}

Die in Kapitel [TODO] und Kapitel [TODO] erstellten Daten der Partei- und Wählerpositionen dienen nun als Grundlage für ein dynamisches Modell. Das Modell von \citet{laver2005policy} das in Kapitel [TODO] vorgestellt wurde wird hier umgesetzt.

\subsection{Umsetzung des Modells}

Zur Umsetzung des agentenbasierten Modells wird hier Mesa \citep{mesa2020} verwendet. Das Mesa Projekt ist ein open-source Projekt das auf github zu finden ist [TODO mesa repository]. 
Mesa bietet ein Gerüst innerhalb dessen ein agentenbasiertes Modell einfach und effizient umgesetzt werden kann. Es bietet Werkzeuge zum Aufbau, der Analyse und der Visualisierung dieser Modelle.
% TODO cite https://mesa.readthedocs.io/en/latest/overview.html

Zunächst wird das Modell in das allgemeine Modell und seine Agenten eingeteilt. In diesem Fall sind die Parteien die Agenten, da sie ihre Position dynamisch ändern. Die Wähler könnten auch als Agenten modelliert werden. Allerdings ist es hier einfacher sie als Teil des äußeren Modells zu betrachten, da sie lediglich ihre Parteipräferenz ändern. So können die Berechnungen effizienter durchgeführt werden, als wenn alle Wähler einzelne Agenten wären.

\paragraph{Implementierung}
Das Modell übernimmt die Berechnung der Wählerzuordnung und der daraus abgeleiteten Größen wie zum Beispiel den Wähleranteil einer Partei. Das Modell führt die Zeitschritte aus. Innerhalb eines Zeitschritts hat die Partei als Agent die Aufgabe, ihre neue Position für den nächsten Zeitschritt zu berechnen. Dabei kann sie auf die Ressourcen des Modells zurückgreifen und die Art und Weise hängt vom Typ der Partei ab. Wie in Kapitel [TODO] beschrieben teilt \citet{laver2005policy} die Parteien in die Typen Aggregator, Hunter, Predator und Sticker ein. Bei programmierten Modellen stecken wichtige Elemente auch in Implementierungsdetails. Im Folgenden wird insbesondere auf folgende Punkte eingegangen:
\begin{itemize}
\item Einheitslänge
\item Hunter mit stagnierendem Stimmenanteil
\item Ausführungsreihenfolge
\item Anzahl Zyklen
\end{itemize}

\paragraph{Einheitslänge}
\citet{laver2005policy} verwendet im Modell, insbesondere für Hunter und Predator eine Einheitslänge als Bewegung für den nächsten Zeitschritt. Allerdings ist nicht eindeutig geklärt wie diese Einheitslänge definiert ist.
Das die Länge nicht eindeutig geklärt ist, ist ein Problem insofern, dass sich die Ergebnisse je nach Einheitslänge erheblich unterscheiden können. Bei einer zu großen Einheitslänge sind die Positionsanpassungen der Parteien zu groß. Dadurch verlieren die Positionsanpassungen ihren Sinn und das Modell kann sich nicht stabilieren. Ist die Einheitslänge dagegen zu klein, braucht das Modell sehr viele Zyklen um sich zu ändern beziehungsweise zu stabilisieren. Im schlimmsten Fall funktioniert das Modell gar nicht, weil beispielsweise der Hunter aufgrund der Diskretität der Wähler mit einer kleinen Schrittlänge gar keine Stimmenanteiländerung feststellt.
Aufgrund einer Abbildung Lavers \citep[Abb.\,4]{laver2005policy} schätze ich einen ungefähren Wert ab. Daher wähle ich für mein Modell einen Wert von $0,1$.

\paragraph{Hunter mit stagnierendem Stimmenanteil}
Direkt mit der Einheitslänge verbunden ist die Problematik was der Hunter tut, wenn sein Stimmenanteil gleich bleibt. Wird die momentane Richtung beibehalten, dann verhält sich der Hunter eher explorativ. Dagegen kann diese Strategie katastrophal scheitern, wenn beispielsweise der Stimmenanteil bei Null liegt und der Hunter sich immer weiter von den Wählern entfernt. Die andere Möglichkeit ist, dass der Hunter seine Richtung wechselt. Dann ist die Strategie des Hunters stabiler. Allerdings besteht die Gefahr, dass zu exploriert wird. Diese Eigenschaft muss also direkt mit der Einheitslänge abgestimmt werden.
Laver entscheidet sich für den Richtungswechsel, weshalb diese Arbeit diesen Fall ebenso handhabt \citep[S.\,280]{laver2005policy}.

\paragraph{Ausführungsreihenfolge}
Die Ausführungsreihenfolge zwischen den Parteien kann einen kleinen Unterschied machen. Da in dieser Version die Wählerzuordnungen erst aktualisiert werden, wenn alle Parteien ihre Positionen festgelegt haben, macht es für die meisten Parteitypen keinen Unterschied. Einzig der Predator macht seine Position von der Position der anderen Parteien abhängig. Da es im Normalfall keinen großen Unterschied macht, wird die Reihenfolge als zufällig gewählt.

\paragraph{Anzahl Zyklen}
Laver zeigt in seiner Arbeit, dass ein Modell mit bis zu 10 Parteien des Typs Aggregator mit bis zu 1000 Wählern nach spätestens 55 Zyklen statisch wird \citep[S.\,271-2]{laver2005policy}. Das dient als Orientierung, um abzuschätzen, wann ein Modell statisch wird.

\paragraph{Ausführungsdauer}
Die resultierende Implementierung ist effizient. So kann beispielsweise ein System mit ??? Wählern und ??? Parteien ??? Zyklen innerhalb von ??? Sekunden berechnen [TODO]. Kleinere Systeme haben eine vernachlässigbare Ausführungsdauer. Für die nachfolgend aufgeführten Experimente ist die Ausührungsdauer keine Schwierigkeit.

\subsection{Ergebnisse}

Nun wird das Modell auf die zuvor abgeleiteten Partei- und Wählerpositionen angewendet. Das Modell wird initialisiert mit den berechneten Positionen. Als Parteitypen werden die verschiedenen Konstellationen aus den Ergebnissen von \citet{laver2005policy} durchgespielt, die in Kapitel [TODO] vorgestellt wurden. Das Modell läuft für 1000 Zyklen, damit die Modelle, die statisch werden statisch geworden sind. Im Folgenden werden die Ergebnisse für die verschiedenen Konstellationen präsentiert.

\paragraph{Nur Aggregatoren}

Bei nur Aggregatoren wird laut \citet{laver2005policy} eine gleichmäßige Verteilung über die Positionen der Wähler erwartet. Das Ergebnis auf den Daten für Deutschland 2017 ist in Abbildung \ref{fig:laver-aggregator6} zu sehen. Tatsächlich ist diese gleichmäßige Verteilung zu beobachten. Darüber hinaus ist bemerkenswert, dass die Parteipositionen sehr ähnlich zu den Ausgangspositionen sind. Das bedeutet, dass dieses Modell die tatsächlichen Parteipositionen relativ gut beschreibt.

\begin{figure}[htb]
	\centering
	\includegraphics[scale=1.0]{../../fig/fig_laver_aggregator6.png}
	\caption{TODO (Quelle: eigene Darstellung)}
	\label{fig:laver-aggregator6}
\end{figure}

\paragraph{Nur Hunter}

In Abbildung \ref{fig:laver-hunter6} ist das Ergebnis zu sehen, wenn alle sechs Parteien Hunter sind. Die Parteien bewegen sich in Richtung des Punktes der höchsten Konzentration im unteren Bereich. Allerdings haben die Parteien genügend Abstand, sodass jede Partei einen ausreichenden Anteil an Wählern besitzt. Im Gegensatz zum Ergebnis von \citet{laver2005policy} sammeln sich die Parteien nicht kreisförmig um einen Punkt, sondern haben ein komplexeres Muster.

\begin{figure}[htb]
	\centering
	\includegraphics[scale=1.0]{../../fig/fig_laver_hunter6.png}
	\caption{TODO (Quelle: eigene Darstellung)}
	\label{fig:laver-hunter6}
\end{figure}

\paragraph{Aggregatoren gegen einzelne Hunter oder Predatoren}

In dieser Arbeit wurden viele Szenarien durchgespielt, in denen vier oder fünf Aggregatoren gegen einzelne Hunter oder Predatoren antreten. Repräsentativ dafür ist in Abbildung \ref{fig:laver-aggregator4-hunter-predator} die Situation von vier Aggregatoren mit einem Hunter und einem Predator dargestellt. Die Hunter und Predatoren bewegen sich ausnahmslos in den unteren Bereich mit der hohen Wählerdichte. Allerdings schaffen es nur die Hunter und die auch nicht immer den attraktivsten Platz ganz unten einzunehmen. Ansonsten konzentrieren sich die Positionen der einzelnen Hunter und Predatoren auf den Bereich rechts davon. Wie bei \citet{laver2005policy} bleiben die Aggregatoren sehr robust und behalten wesentliche Wähleranteile. In der Situation auf den realen Daten sind die einzelnen Hunter und Predatoren jedoch erfolgreicher als bei \citet{laver2005policy}. Dies liegt insbesondere an der unregelmäßigen Wählerverteilung, die die symmetrische Situation wie \citet{laver2005policy} vermeidet.

\begin{figure}[htb]
	\centering
	\includegraphics[scale=1.0]{../../fig/fig_laver_aggregator4_predator_hunter.png}
	\caption{TODO (Quelle: eigene Darstellung)}
	\label{fig:laver-aggregator4-hunter-predator}
\end{figure}

\paragraph{Hunter gegen einzelnen Predator}

In Abbildung \ref{fig:laver-hunter5-predator} ist das Ergebnis von fünf Huntern mit einem einzelnen Predator dargestellt. Das Ergebnis ist sehr dynamisch und vom Zufall beeinflusst. Allen Ergebnissen ist jedoch gemein, dass der Predator tendenziell nicht die erfolgreichste Partei ist. Stattdessen pendelt er wie bei \citet{laver2005policy} zwischen den erfolgreichsten Huntern hin und her. Dadurch hält er sich zwar in der Region mit der größten Wählerdichte auf, kann dort aber nicht unbeschränkt dominieren.

\begin{figure}[htb]
	\centering
	\includegraphics[scale=1.0]{../../fig/fig_laver_hunter5_predator.png}
	\caption{TODO (Quelle: eigene Darstellung)}
	\label{fig:laver-hunter5-predator}
\end{figure}

\paragraph{Bewertung}

Die Ergebnisse sind im Großen und Ganzen wie in \citet{laver2005policy}. Es ergeben sich jedoch Unterschiede aus der anderen Verteilung der Wähler. Insbesondere die unsymmetrische Verteilung sorgt dafür, dass manche Gleichgewichte nicht zustande kommen und die Hunter und Predatoren erfolgreicher sind.

Die meisten Ergebnisse hier unterscheiden sich wesentlich von der Ausgangssituation. Eine Ausnahme davon bildet die Situation mit ausschließlich Aggregatoren. In dieser Situation bleiben die Parteien relativ stabil in der Nähe ihrer Ausgangsposition. Daher bildet dieses Modell die Daten dieser Arbeit am besten ab.

\cleardoublepage

%%
%%%%%%%%%%%%%%%%%%%%%%%%%%%%%%%%%%%%%%%%%%%%%%%%%%%%%%%%%%%%%%%%%%%%
% Diskussion und Ausblick
%%%%%%%%%%%%%%%%%%%%%%%%%%%%%%%%%%%%%%%%%%%%%%%%%%%%%%%%%%%%%%%%%%%%

\chapter{Diskussion}\label{Kap-Diskussion}

\blindtext[3]

\blindtext[3]

\blindtext[3]

\blindtext[3]

\cleardoublepage
\cleardoublepage

%%%%%%%%%%%%%%%%%%%%%%%%%%%%%%%%%%%%%%%%%%%%%%%%%%%%%%%%%%%%%%%%%%%%%%%%%%%%%
%%% Bibliographie
%%%%%%%%%%%%%%%%%%%%%%%%%%%%%%%%%%%%%%%%%%%%%%%%%%%%%%%%%%%%%%%%%%%%%%%%%%%%%

\addcontentsline{toc}{chapter}{Literaturverzeichnis}

\bibliographystyle{agsm}
\bibliography{mylit}
%% Obige Anweisung legt fest, dass BibTeX-Datei `mylit.bib' verwendet
%% wird. Hier koennen mehrere Dateinamen mit Kommata getrennt aufgelistet
%% werden.

%%%%%%%%%%%%%%%%%%%%%%%%%%%%%%%%%%%%%%%%%%%%%%%%%%%%%%%%%%%%%%%%%%%%%%%%%%%%%
%%% Ende
%%%%%%%%%%%%%%%%%%%%%%%%%%%%%%%%%%%%%%%%%%%%%%%%%%%%%%%%%%%%%%%%%%%%%%%%%%%%%

\end{document}

