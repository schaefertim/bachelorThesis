%%%%%%%%%%%%%%%%%%%%%%%%%%%%%%%%%%%%%%%%%%%%%%%%%%%%%%%%%%%%%%%%%%%%%%%%%%%%%
%%% LaTeX-Rahmen fuer das Erstellen von Masterarbeiten
%%%%%%%%%%%%%%%%%%%%%%%%%%%%%%%%%%%%%%%%%%%%%%%%%%%%%%%%%%%%%%%%%%%%%%%%%%%%%

%%%%%%%%%%%%%%%%%%%%%%%%%%%%%%%%%%%%%%%%%%%%%%%%%%%%%%%%%%%%%%%%%%%%%%%%%%%%%
%%% allgemeine Einstellungen
%%%%%%%%%%%%%%%%%%%%%%%%%%%%%%%%%%%%%%%%%%%%%%%%%%%%%%%%%%%%%%%%%%%%%%%%%%%%%

\documentclass[oneside,12pt,a4paper]{report}

% set Times New Roman
\usepackage{mathptmx}
\usepackage[T1]{fontenc}

%\usepackage{reportpage}
\usepackage{epsf}
\usepackage{graphics, graphicx}
\usepackage{latexsym}
\usepackage[margin=10pt,font=small,labelfont=bf]{caption}
\usepackage[utf8]{inputenc}
\usepackage[toc,page]{appendix}

% Literatur
\usepackage[
	backend=biber,
	natbib=true,
	language=ngerman,
	style=authoryear,
	maxcitenames=2,
]{biblatex}
\DefineBibliographyStrings{ngerman}{ 
	andothers = {{et\,al\adddot}},             
}
\addbibresource{mylit.bib}
\renewcommand*{\newunitpunct}{\addcomma\space}
% TODO checkout 
\DeclareFieldFormat[book, article, incollection, online, misc, inproceedings]{citetitle}{#1}
\DeclareFieldFormat[book, article, incollection, online, misc, inproceedings]{title}{#1} 
\DeclareFieldFormat[book]{number}{Nr.,\ #1} 

% meine Pakete
\usepackage{ngerman}
\usepackage{tikz}
\usetikzlibrary{calc}
\usepackage{amsmath}
\usepackage{setspace}

% don't reset counter of figures and tables
\usepackage{chngcntr}
\counterwithout{figure}{chapter}
\counterwithout{table}{chapter}
\counterwithout{equation}{chapter}

%% für die Tabellen
\usepackage{csvsimple}  %https://ftp.rrzn.uni-hannover.de/pub/mirror/tex-archive/macros/latex/contrib/csvsimple/csvsimple-l3.pdf
\usepackage{siunitx}
% left fixed width:
\newcolumntype{L}[1]{>{\raggedright\arraybackslash}p{#1}}
% center fixed width:
\newcolumntype{C}[1]{>{\centering\arraybackslash}p{#1}}
% flush right fixed width:
\newcolumntype{R}[1]{>{\raggedleft\arraybackslash}p{#1}}


\textwidth 14cm
\textheight 22cm
\topmargin 0.0cm
\evensidemargin 1cm
\oddsidemargin 1cm
%\footskip 2cm
\parskip0.5explus0.1exminus0.1ex

% Kann von Student auch nach persönlichem Geschmack verändert werden.
\pagestyle{headings}

\sloppy

\begin{document}

%%%%%%%%%%%%%%%%%%%%%%%%%%%%%%%%%%%%%%%%%%%%%%%%%%%%%%%%%%%%%%%%%%%%%%%%%%%%
%%% hier steht die neue Titelseite 
%%%%%%%%%%%%%%%%%%%%%%%%%%%%%%%%%%%%%%%%%%%%%%%%%%%%%%%%%%%%%%%%%%%%%%%%%%%%
 
\begin{titlepage}
	\raggedright
	{\huge\bf FernUniversität in Hagen\\[0.5cm]}
	{\LARGE\bf Fakultät für Wirtschaftswissenschaft\\[1.5cm]}
	{\large\doublespacing
		Bachelorarbeit\\
		zur Erlangung\\
		\singlespacing des Grades eines\\
		Bachelor of Science\\
		\doublespacing über das Thema\\[2cm]
	}
	\begin{center}
	{\huge\bf Strategische Positionierung Politischer Parteien in Deutschland\\[3cm]}
	\end{center}
	\vfill
	\begin{minipage}{\textwidth}
		\doublespacing
		\parbox[t][4.5cm][t]{3.5cm}{
			Eingereicht bei:\\
			Betreuer:\\
			Von cand.rer.oec.:\\
			Matrikelnummer:\\
			Anschrift:\\
			Telefon:\\
			E-Mail:\\
			Abgabedatum:\\
		}\hfill
		\parbox[t][2cm][t]{11cm}{
			Prof.\,Dr.\,Robert Schmidt \\
			Vitus Bühl\\
			Tim Schäfer\\
			6231080\\
			72622 Nürtingen, Teichstr.\,19\\
			015770295697\\
			tim@gruber-schaefer.de\\
			11.04.2022
		}
	\end{minipage}
\end{titlepage}


%%%%%%%%%%%%%%%%%%%%%%%%%%%%%%%%%%%%%%%%%%%%%%%%%%%%%%%%%%%%%%%%%%%%%%%%%%%%

\pagenumbering{roman}
\setcounter{page}{1}

%%%%%%%%%%%%%%%%%%%%%%%%%%%%%%%%%%%%%%%%%%%%%%%%%%%%%%%%%%%%%%%%%%%%%%%%%%%%
%%% Seite I: Zusammenfassug, Danksagung
%%%%%%%%%%%%%%%%%%%%%%%%%%%%%%%%%%%%%%%%%%%%%%%%%%%%%%%%%%%%%%%%%%%%%%%%%%%%


\section*{Abstract}

% formale Vorgaben:
% 100 bis 200 Wörter, Fragestellung und wichtigste Erkenntnisse/Ergebnisse

%Orientation in https://chemistrycommunity.nature.com/posts/43071-how-to-write-an-abstract
%1) Introduction (2 sentences):
%--> Sentence 1: Basic introduction to the field; accessible to scientists of any  discipline.
Um die strategischen Komponenten des politischen Wettbewerbs zu untersuchen, werden oft räumliche Modelle verwendet.
%--> Sentence 2: Background of the specific research question; comprehensible to scientists in the same or closely related fields of research.
Einfache räumliche Modelle sind analytisch lösbar, wohingegen komplexere Modelle agentenbasiert gelöst werden.
%2) Problem/objective (1 sentence):
%--> Explanation what is missing/unknown/problematic, i.e. why the current study happened. Typically, this sentence starts with “However”.
Diese agentenbasierten Modelle können in die Praxis umgesetzt werden.
%4) Main results and conclusions (~ 3 – 5 sentences)
%--> Summary of the most important findings of the study that are the foundation of the main conclusions. A few key bits of data are welcome but adding too many numbers is off-putting to the reader. Keep it focussed.
%--> Unless the method is new and/or a main part of the paper, there’s no need to include any details in the abstract. If you mentioned, it should be included in a sentence along the lines of “Using xyz, we show that (…)”.
Nach einer Vorstellung einiger relevanter Modelle, wird das Parteienmodell von \citet{laver2005policy} auf die Daten der Bundestagswahl 2017 angewandt.
Dazu wird der zweidimensionale Positionsraum mithilfe einer Hauptkomponentenanalyse der Parteipositionen berechnet und die Wähler werden aufgrund von Präferenzangaben positioniert.
Auf dieser Grundlage werden die Vorhersagen des agentenbasierten Modells getestet und großteils bestätigt.
%5) Implications (1 – 2 sentences)
%--> Some explanation on how your findings advance the field. Where does your work lead and what are the immediate implications? The word “immediate” is key here because being too creative or hyping the work are pitfalls that should be avoided. Rather, keep it realistic and explain which opportunities your work offer and/or what it leads to. 
Diese Arbeit demonstriert den Übergang von einer Modellsituation zur Arbeit an realen Daten. So können zukünftig einzelne Aspekte dieser Arbeit oder weitere agentenbasierte Modelle genauer untersucht werden.

\newpage

\section*{Danksagungen}

Ich danke Prof.\,Schmidt und Vitus Bühl für Ihre Unterstützung und Expertise. Ich danke insbesondere Vitus Bühl für die enge Zusammenarbeit und die anregenden Diskussionen.

\cleardoublepage

%%%%%%%%%%%%%%%%%%%%%%%%%%%%%%%%%%%%%%%%%%%%%%%%%%%%%%%%%%%%%%%%%%%%%%%%%%%%%
%%% Table of Contents
%%%%%%%%%%%%%%%%%%%%%%%%%%%%%%%%%%%%%%%%%%%%%%%%%%%%%%%%%%%%%%%%%%%%%%%%%%%%%

\renewcommand{\baselinestretch}{1.3}
\small\normalsize

\tableofcontents

\renewcommand{\baselinestretch}{1}
\small\normalsize

\cleardoublepage

%%%%%%%%%%%%%%%%%%%%%%%%%%%%%%%%%%%%%%%%%%%%%%%%%%%%%%%%%%%%%%%%%%%%%%%%%%%%%
%%% Der Haupttext, ab hier mit arabischer Numerierung
%%% Mit \input{dateiname} werden die Datei `dateiname' eingebunden
%%%%%%%%%%%%%%%%%%%%%%%%%%%%%%%%%%%%%%%%%%%%%%%%%%%%%%%%%%%%%%%%%%%%%%%%%%%%%

\pagenumbering{arabic}
\setcounter{page}{1}

%% Introduction
%%%%%%%%%%%%%%%%%%%%%%%%%%%%%%%%%%%%%%%%%%%%%%%%%%%%%%%%%%%%%%%%%%%%
% Einleitung
%%%%%%%%%%%%%%%%%%%%%%%%%%%%%%%%%%%%%%%%%%%%%%%%%%%%%%%%%%%%%%%%%%%%

\chapter{Einleitung}\label{Kap-Einleitung}
Politische Partien müssen zu einer Vielfalt von Themen ihre Standpunkte bestimmen. Abhängig von ihren Standpunkten können sie mehr oder weniger Wähler und Wählerinnen überzeugen. Daher ist es naheliegend, dass eine rational agierende Partei ihre Standpunkte unter strategischen Gesichtspunkten auswählt. Andererseits sind Parteien auch idelogische Konstrukte, die ihre Standpunkte aus Überzeugung auswählen. Diese Arbeit soll die Hypothese überprüfen, dass politische Parteien ihre Standpunkte unter strategischen Gesichtspunkten wählen. Dabei zeigen bisherige Arbeiten, dass Parteien unter strategischen Gesichtspunkte ihre Wählerpotentiale nicht ausschöpfen \citep{schofield1998germany}.

Im ersten Teil der Arbeit sollen theoretische Modelle vorgestellt werden. Dazu werden  zunächst einfache Modelle des räumlichen Wettbewerbs betrachtet. Hotellings Gesetz besagt, dass Wettbewerber minimale Differenzierung betreiben \citep{hotelling1929}.

Aspremont et al. \citep{aspremont1979} kritisieren dieses Modell. Sie nehmen an, dass die Transportkosten anstatt linear quadratisch im Abstand sind. Das führt dazu, dass Wettbewerber zur Gewinnmaximierung zu maximaler Differenzierung neigen, also genau das Gegenteil von Hotellings Gesetz. Auch im politischen Kontext ist es denkbar, dass die Nutzenfunktion quadratisch im Abstand der politischen Position ist.

Moderne Ansätze sind häufig agentenbasiert. Diese Ansätze sind in der Regel dynamisch, das heißt es finden mehrere Perioden statt. Marchi und Page \citep{marchi2014ABMs} geben einen Übrblick über solche agentenbasierten Modell. Dabei gibt es sowohl Modelle die die Zahl der Parteien exogen festlegen, als auch Modelle, die die Zahl der Parteien flexibel halten \citep{laver2007endogenousParties}.

Abschließend wird das Wahlsystem in Deutschland vorgestellt. Außerdem sollen aus den Theorien konkrete überprüfbare Vorhersagen für deutsche Parteien und Wähler abgeleitet werden.

Im zweiten Teil der Arbeit sollen die abgeleiteten Hypothesen anhand von Daten getestet werden. 

Um Aussagen über die Verteilung der Parteienpositionen zu testen, werden die Wahl-O-Mat-Daten (Bundeszentrale für politische Bildung, 2021) verwendet. Diese Daten umfassen für alle Parteien eine Aussage (Ablehnung, neutral, Zustimmung) für eine Vielzahl von Positionen.

Um Aussagen über Wählerpositionen zu testen, werden die Daten des Politbarometers (Forschungsgruppe Wahlen, Mannheim, 2021) verwendet. Dies sind Wählerdaten, die generelle politische Präferenzen abfragen und einzelne zusätzliche Positionen erfragen.

Als dritte Überprüfung werden Daten über Wählerwanderung berücksichtigt. Außerdem werden Daten über die Meinung zu Spitzenpolitiker erfragt. Dies erlaubt Rückschlüsse darauf, ob die viele Wähler zwischen mehreren Parteien unentschlossen sind. Das kann entweder bedeuten, dass die Parteipositionen sehr nah beieinander liegen oder es kann bedeuten, dass die Nutzenfunktion der Wähler sehr unempfindlich gegenüber den Parteipositionen ist.

In einer abschließenden Diskussion soll bewertet werden, inwiefern welche Theorie einer praktischen Überprüfung standhält und welche theoretischen Grundlagen eine geeignete Beschreibung der politischen Realität liefern.

\cleardoublepage

%% 
\chapter{Theoretische Analyse}\label{Kap-Theorie}

\noindent
Text...

\section{Hotellings Gesetz}\label{Sec-Hotelling}

Hotelling \citep{hotelling1929} untersucht anhand eines einfachen Modells wie sich bei einem Duopol der räumliche Wettbewerb gestaltet. Dazu nimmt er an, dass die zwei Wettbewerber auf einer räumlichen Linie ihre Position wählen können. Abbildung \ref{Fig-Linearer-Wettbewerb} zeigt mögliche Positionen der Wettbewerber. Dabei sind die Konsumenten gleichmäßig über die gesamte Linie verteilt.

\begin{figure}[htb]
	\centering
	\begin{tikzpicture}
	\coordinate (start) at (0,0) {};
	\coordinate (end) at (10,0) {};
	\coordinate (left) at (2,0) {};
	\coordinate (right) at (7,0) {};
	\draw (start) -- (end);
	\draw ($(start)+(0,5pt)$) -- ($(start)-(0,5pt)$);
	\draw ($(left)+(0,5pt)$) -- ($(left)-(0,5pt)$);
	\draw ($(right)+(0,5pt)$) -- ($(right)-(0,5pt)$);
	\draw ($(end)+(0,5pt)$) -- ($(end)-(0,5pt)$);
	\node at ($(start)+(0,10pt)$) {$0$};
	\node at ($(end)+(0,10pt)$) {$l$};
	\node at ($(left)+(0,10pt)$) {$A$};
	\node at ($(right)+(0,10pt)$) {$B$};
	\end{tikzpicture}
	\caption{Positionen der Wettbewerber $A$ und $B$ und Konsumenten auf einer Linie von $0$ bis $l$.}
	\label{Fig-Linearer-Wettbewerb}
\end{figure}

Die Konsumenten kaufen bei demjenigen Duopolisten, der ihnen den größten Nutzen gibt. Die Nutzenfunktion der Konsumenten ist dabei
\begin{equation}
U_{Hotelling} = 
\end{equation}

\section{Modell von d‘Aspremont}\label{Sec-Aspremont}

\section{Agentenbasierte Ansätze}\label{Sec-ABM}

\section{Situation in Deutschland}\label{Sec-Deutschland}

\section{Vorhersagen aus Modellen}\label{Sec-Vorhersagen}

\cleardoublepage

%%
%%%%%%%%%%%%%%%%%%%%%%%%%%%%%%%%%%%%%%%%%%%%%%%%%%%%%%%%%%%%%%%%%%%%
% Ergebnisse
%%%%%%%%%%%%%%%%%%%%%%%%%%%%%%%%%%%%%%%%%%%%%%%%%%%%%%%%%%%%%%%%%%%%

\chapter{Ergebnisse}\label{Kap-Ergebnisse}


\cleardoublepage

%%
%%%%%%%%%%%%%%%%%%%%%%%%%%%%%%%%%%%%%%%%%%%%%%%%%%%%%%%%%%%%%%%%%%%%
% Diskussion und Ausblick
%%%%%%%%%%%%%%%%%%%%%%%%%%%%%%%%%%%%%%%%%%%%%%%%%%%%%%%%%%%%%%%%%%%%

\chapter{Diskussion}\label{Kap-Diskussion}

\blindtext[3]

\blindtext[3]

\blindtext[3]

\blindtext[3]

\cleardoublepage
\cleardoublepage

%%%%%%%%%%%%%%%%%%%%%%%%%%%%%%%%%%%%%%%%%%%%%%%%%%%%%%%%%%%%%%%%%%%%%%%%%%%%%
%%% Bibliographie
%%%%%%%%%%%%%%%%%%%%%%%%%%%%%%%%%%%%%%%%%%%%%%%%%%%%%%%%%%%%%%%%%%%%%%%%%%%%%

\addcontentsline{toc}{chapter}{Literaturverzeichnis}

%\bibliographystyle{agsm}
%\bibliography{mylit}
%% Obige Anweisung legt fest, dass BibTeX-Datei `mylit.bib' verwendet
%% wird. Hier koennen mehrere Dateinamen mit Kommata getrennt aufgelistet
%% werden.

\renewcommand{\bibname}{Literaturverzeichnis}
\printbibliography

%%%%%%%%%%%%%%%%%%%%%%%%%%%%%%%%%%%%%%%%%%%%%%%%%%%%%%%%%%%%%%%%%%%%%%%%%%%%%
%%% Ende
%%%%%%%%%%%%%%%%%%%%%%%%%%%%%%%%%%%%%%%%%%%%%%%%%%%%%%%%%%%%%%%%%%%%%%%%%%%%%

\cleardoublepage

\addcontentsline{toc}{chapter}{Eidesstattliche Erklärung}

\chapter*{ Eidesstattliche Erklärung}

\noindent
Ich erkläre, dass ich die Bachelorarbeit selbstständig und ohne unzulässige Inanspruchnahme Dritter verfasst habe. Ich habe dabei nur die angegebenen Quellen und Hilfsmittel verwendet und die aus diesen wörtlich, inhaltlich oder sinngemäß entnommenen Stellen als solche den wissenschaftlichen Anforderungen entsprechend kenntlich gemacht. Die Versicherung selbstständiger Arbeit gilt auch für die Zeichnungen, Skizzen oder graphische Darstellungen. Die Arbeit wurde bisher in gleicher oder ähnlicher Form weder derselben noch einer anderen Prüfungsbehörde vorgelegt und auch noch nicht veröffentlicht. Mit der Abgabe der elekronischen Fassung der endgültigen Version der Arbeit nehme ich zur Kenntnis, dass diese mit Hilfe eines Plagiatserkennungsdienstes auf enthaltene Plagiate überprüft und ausschließlich für Prüfungszwecke gespeichtert wird.

\vskip 2cm

Ort, Datum	\hfill Unterschrift \hfill 

\end{document}

